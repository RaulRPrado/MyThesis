\chapter[Introduction]{Introduction}
\label{sec:introduction}

The Earth is constantly being reached by charged particles
coming from extraterrestrial sources, which are called Cosmic Rays.
For over a century, the study of these particles have been a very active field
that integrates aspects from both particle physics and astrophysics.
The extremelly wide energy range covered by the well know cosmic rays
energy spectrum make clear that a number of different astrophysical phenomena,
at distinct scales, are contributing to the production of these particles.

Of particular interest is the ultra-high energy range that includes
particles from $E=\E{17}$ up the end of the spectrum at $\sim\E{20.5}$. 
Althought the existence of these ultra-high energy particles is known
since the 1960's, due to detections done by Haverah Park experiment,
the precise measurements of their energy spectra and other features
became possible after the 1990's, with the experiments Fly's Eyes,
HiRes and AGASA. During the last decade, two modern experiments have
been running at the ultra-high energy range, Pierre Auger Observatory
and Telescope Array. 
Even though a large number of important results have been provided
by these experiments lately, the main question about the origin
of ultra-high energy cosmic rays are still unsolved. 
Apart from knowing the astrophysical sources and the acceleration
mechanisms, it is also unclear what is the exact nature of two evident structures
observed in the ultra-high energy spectrum, the \emph{ankle} and the \emph{suppression}.




Because of the very low flux of particles at the 



The compreesion of the astrophysical aspects behind the cosmic rays
is mainly complicated by the fact that they cannot be directly measured
at the high energy regime. Their detection is actually done by measuring
the cascade of particles created by the interaction of the cosmic ray particle
with atmospheric nuclei, the so called \emph{Extensive Air Showers}.  


Thus, the properties of the primary particles have to be infered from


The inferences of the properties of the primary cosmic ray have then
to be done by means of the measured features of the 





-cosmic rays->uhe->open questions

-cr->showers->understanding to infer

-muons->muon deficit problem

-augerprime->chap 3 and 4->simulation studies

-na61->pion-air






