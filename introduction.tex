\chapter[Introduction]{Introduction}
\label{sec:introduction}

The study of cosmic rays has been a very active field since the discovery
of these particles by Victor Hess, in 1912.~\cite{Hess1912}
Before the first accelerators were built during the 1950's,
the only way to study elementary particle physics was through the
detection of particles produced by cosmic ray interactions.
The discovery of positrons, muons and pions
are examples of the importance of cosmic rays for the development
of the particle physics field. 
One of the most striking events of the history of cosmic rays
was the discovery of Extensive Air Showers (EAS) by Pierre Auger,
in 1938.~\cite{Auger1938} By measuring coincidences on the time of signals
from detectors separated in space,
he could confirm the hypothesis that cosmic ray particles can interact
with atmospheric nuclei and start a cascade of secondary particles that
can eventually reach the ground. After Auger's discovery, the idea
of measuring the properties of cosmic rays through the detection of EAS
with arrays of particle detectors at ground became a central concept for
the experimental cosmic ray studies.

The advance of techniques to detect EAS and the construction
of large detector arrays allowed us to extend
cosmic ray studies up to the very high energy regions, where
the flux of particles is very low. Currently, the cosmic ray
energy spectrum is well known from \e{9} up to \E{21}.
Although this spectrum follows an overall power-law shape,
a number of features are observed and the  understanding of these features
in terms of astrophysical phenomena is one the motivations of the field.
The main interest of this thesis is on the Ultra-High Energy Cosmic Rays (UHECR)
region ($E>\E{17}$), where two spectrum features are observed: the \emph{ankle},
which is a softening of the flux around $4\times\E{18}$ and the \emph{suppression},
which is an abrupt reduction of the flux above $4\times\E{19}$.
Most questions about the nature of the UHECR are still opened,
and in particular, the actual nature of the ankle and the suppression is still
a matter of discussions.

Measuring the properties of UHECR is the main goal of the largest
cosmic ray experiment ever built, the Pierre Auger Observatory,
located in Malargue, Argentina, and operating since 2004.~\cite{\AugerPaper}
While its large collecting area (3000 km$^2$) guarantee a large
event statistics, its hybrid detection system that includes
surface detectors and fluorescence telescopes can provide a
precise measurements of the properties of the EAS.
An overview of the current status of the UHECR field
and an introduction to the Pierre Auger Observatory
will be given in~\cref{sec:uhecr} of this thesis. \newline

The precise inference of the properties of UHECR
from EAS measurements requires a deep understanding
of these objects. The most complete modeling of EAS is
achieved by means of Monte Carlo simulations.
Among the possible sources of uncertainties related to the
simulation of EAS, the main one is the modeling
of the hadronic interactions. 
Because of the phenomenological nature of the hadronic interaction models,
they are tuned to accelerator measurements
and a number of problems with this process originate a substantial
uncertainty on the predictions of the EAS properties by simulations.

The importance of EAS simulations comes mostly from the fact that
some properties of UHECR, including its mass composition,
can only be inferred by the comparison of EAS measurements
to the predictions from simulations.
The most reliable composition analysis at the UHECR regime
are based on the shower maximum parameter (\xmax) which is defined
as the atmospheric depth at which the shower reaches the maximum
number of particles. Apart from the \xmax, shower parameters related
to the muonic component of the showers are also sensitive to
the cosmic ray composition, the most common one being the number
of muons in the shower (\nmu). However, a number of measurements of the \nmu
have shown that the comparison with predictions from simulations leads
to inconsistent results. To be more precise, the \nmu observed is systematically
larger than the ones predicted by EAS simulations using hadronic interaction
models. Although this is traditionally known as \emph{muon deficit problem},
further EAS measurements have shown
that the simulations also fail to predict other properties of the muonic component.
As a direct consequence, measurements of muonic observables cannot be
properly used to infer cosmic ray composition.  
An overview of the EAS physics, simulations and a summary of measurements
of EAS observables at the UHECR regime will be given in~\cref{sec:showers}
of this thesis. \newline

In this thesis, the problem of describing the properties of the muonic component of EAS
by simulations is approached from an experimental perspective by three different fronts.
The first one is presented in~\cref{sec:interpretation},
where the Ref.~\cite{Prado:2016akv} is reproduced,
and it consists of the development of a method to interpret the energy evolution
of the measured \nmu moments in terms of mass composition which takes into account
the misprediction of the \nmu by the simulations. The method is tested by using
a set of simulations and it is expected to be applicable for the analysis
of future measurements of the \nmu by AugerPrime or other experiments.
Preliminary stages of this work can be found in Refs.~\cite{Prado:2015iaz,GAP2014,GAP2015}.

The second front is presented in~\cref{sec:observable} where
the Ref.~\cite{Prado:2017ijs} is reproduced. The main idea is to explore the potentialities
of using the measurements from muon detectors, similar to the ones of AugerPrime,
to further constrain hadronic interaction models. By considering
the operation of two distinct muon detectors, one at the surface
and one buried a few meters below the ground, we have proposed a new observable
that was shown to be sensitive to the properties of the energy spectrum of muons at ground.
This observable was studied by using simulations and it can be seen that its measurements
could provide an efficient way to constrain hadronic models.

The third front, presented in~\cref{sec:hadron},
is the data analysis of the fixed target experiment \NASixtyOne
at CERN Super Proton Synchrotron~\cite{\NASixtyOnePaper},
aiming the measurements of hadron production spectra
in $\pi$-C interactions. These measurements are valuable for the understanding of the muon
production in EAS (see~\cref{sec:showers}) and they are an important part of
the cosmic ray physics programee of \NASixtyOne experiment.
A preliminary stage of this work can be found in Ref.~\cite{\RaulICRC}.
Finally, in~\cref{sec:conclusions} the summary of the thesis is presented.
