%% Resumo.tex
% ---
% Resumo
% ---
\setlength{\absparsep}{18pt} % ajusta o espaçamento dos parágrafos do resumo		
\begin{resumo}[Resumo]
	\begin{flushleft} 
		        \setlength{\absparsep}{0pt} % ajusta o espaçamento da referência	
			\SingleSpacing 
			\imprimirautorabr~ ~\textbf{\imprimirtitleabstract}. \imprimirdata. \pageref{LastPage}p. 
			%Substitua p. por f. quando utilizar oneside em \documentclass
			%\pageref{LastPage}f.
			\imprimirtipotrabalhopt~-~\imprimirinstituicao, \imprimirlocal, \imprimirdata. 
 	\end{flushleft}
        \OnehalfSpacing 			

Raios C\'osmicos Ultra Energ\'eticos (Ultra-High Energy Cosmic Rays, UHECR) somente
podem ser medidos atrav\'es da detec\c{c}\~ao dos Chuveiros Atmosf\'ericos Extensos
(Extensive Air Showers, EAS) criados pela intera\c{c}\~ao do raio c\'osmico prim\'ario com
n\'ucleos atmof\'ericos. A infer\^encia de algumas propriedados dos UHECRs, como a composi\c{c}\~ao
de massa, \'e poss\'ivel somente atrav\'es da compara\c{c}\~ao entre medidas de observ\'aveis dos EASs
com predi\c{c}\~oes geradas por simula\c{c}\~oes de Monte Carlo. A fonte de incerteza mais importante
na descri\c{c}\~ao de EAS por simula\c{c}\~oes \'e a modelagem das intera\c{c}\~oes hadr\^onicas. Por muitos
anos \'e sabido que os modelos de intera\c{c}\~ao hadr\^onica falham na predi\c{c}\~ao de observ\'aveis
dos EASs relacionados a sua componente mu\^onica. A manifesta\c{c}\~ao mais evidente disso \'e 
chamada \emph{problema do d\'eficit de m\'uons} devido ao fato que o n\'umero de m\'uons em chuveiros
com energies acima de \E{18} predito por simula\c{c}\~oes \'e menor que os observados.
O objetivo desta tese \'e abordar este problema atrav\'es de tr\^es frentes. Primeiramente,
um m\'etodo \'e desenvolvido para interpretar as medidas do n\'umero de muons em termos
de composi\c{c}\~ao de raios c\'osmicos considerando o problema do d\'eficit de m\'uons.
Segundo, a proposta e o teste de um observ\'avel que \'e sens\'ivel ao espectro de energia dos m\'uons
na superf\'icie e, consequentemente, pode ser usado para discriminar entre
os modelos de intera\c{c}\~ao hadr\^onica.
Por \'ultima, a produ\c{c}\~ao de m\'uons em chuveiros \'e estudada atrav\'es
de medidas do espectro de produ\c{c}\~ao de hadrons em intera\c{c}\~oes do tipo p\'ion-carbono.



 \textbf{Palavras-chave}: Raios c\'osmicos de altas energies. Chuveiros atmsf\'ericos extensos. Componente mu\^onica de chuveiros atmosf\'ericos. F\'isica de chuveiros atmosf\'ericos.   
\end{resumo}
