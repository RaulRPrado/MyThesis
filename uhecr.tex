\chapter{Cosmic rays and the Pierre Auger Observatory}
\label{sec:uhecr}

In this chapter, an overview of cosmic rays physics
focusing on the UHECRs and an introduction
to the Pierre Auger Observatory will be given.
The goal of this chapter is to present the context and the motivation
for this thesis. More detailed material about the current status
of the cosmic ray field can be found
in Refs.~\cite{Aloisio:2017ooo,Mollerach:2017idb}.


%%%%%%%%%%%%%%%%%%%%%%%%%%%%%%%%%%%%
\section{Overview of cosmic rays}
\label{sec:uhecr:overview}

%%================================%%
\subsection{Energy spectrum}

%%%%%%%%%%%%%% SPEC SWORDY %%%%%%%%%%%%%%%
\begin{figure}
  \centering
  
  \begin{overpic}[clip, rviewport=0 0 1 0.98,width=0.85\textwidth]{spectrum_swordy}
    \put(48,91){
      \begin{tikzpicture}
        \filldraw[draw=black,color=white] (0,0) rectangle (4.2,0.8);
      \end{tikzpicture}
    }
  \end{overpic}
  
  \caption{Compilation of measurements of the cosmic ray energy spectrum from Ref.~\cite{SwordyPlot2001}.
  Two of the spectrum features are indicates: the knee and the ankle.}
  \label{fig:uhecr:overview:spec:swordy}
  \begin{center}
    \small Source: SWORDY.~\cite{SwordyPlot2001}
  \end{center}
\end{figure}

The cosmic ray flux as a function of energy is the so-called \emph{energy spectrum}
and it plays a central role in the understanding of the astrophysical aspects behind these particles.
A compilation of measurements of the cosmic ray
energy spectrum over about 13 orders of magnitude
in energy is shown in~\cref{fig:uhecr:overview:spec:swordy}.
One can see that from \appE{11} up to the highest energies
the spectrum can be described approximately 
by a power law function, $\text{d}\phi/\text{d}E \sim E^{-\gamma}$, where
the spectral index $\gamma$ is not exactly constant over all the energy range
but it changes only slightly, between 2.5 and 3.2.
Because the cosmic ray spectrum extends over a very large
energy range, it is expected that different astrophysical mechanisms
at distinct length scales  
contribute to the production of these particles. An illustration
of that comes from the fact that most of the modern models predict that the
cosmic ray flux is dominated by particles produced inside our galaxy up to around \e{17}-\E{18} and,
above these energies, they have extragalactic origin. The transition between
these two components is currently one of the most active topics of discussion.

Because of the steepness of the spectrum, the particle flux changes over 27 orders of magnitude
from \e{11} to \E{21}. As indicated in~\cref{fig:uhecr:overview:spec:swordy},
while at \E{11} the flux is about one particle/m$^2$/second,
at \E{19} it drops to only one particle/km$^2$/year.
As a consequence, different experimental techniques have to be applied to detect
cosmic ray particles at different energy ranges. Up to \appE{14}, the high flux
allows us to use small area instruments installed in balloons or satellites
to detects the particles before they interact with the Earth's atmosphere.
Above this energy, the interaction with the atmosphere turns to be
useful since we can measure the cosmic rays indirectly through the detection of
the extensive air showers (see~\cref{sec:showers}). For that, large arrays of detectors
are used and their areas can vary substantially depending on the energy range they are
intended to be sensitive. As an example, the KASCADE experiment~\cite{\KASCADEPaper}
is designed to measure particles from around \e{14} to \E{16}
and it has an area of 4000 m$^2$, while the Pierre Auger
Observatory~\cite{\AugerPaper} has an area of 3000 km$^2$ to measure particles above \E{18}.


A few features of the cosmic ray energy spectrum can be identified by the small changes on the
spectral index $\gamma$. To better visualize these changes, we show
in~\cref{fig:uhecr:overview:spec:pdg} a compilation of measurements of
the spectrum, starting at \appE{13}, in which the flux is scaled by a factor $E^{2.6}$.
As indicated in~\cref{fig:uhecr:overview:spec:pdg}, the lowest-energy feature
is the so-called \emph{first-knee} and it occurs around $3\times\E{15}$. At this energy
the index changes from $\gamma\approx 2.7$ to $3.0$. Increasing the energy, the next feature is
the so-called \emph{second-knee}, around \E{17}, where there is a further steepening
and the index goes to $\gamma\approx 3.3$. The spectrum then becomes harder again
at the \emph{ankle}, around $5\times\E{18}$, where the index changes to $\gamma\approx 2.6$.
The final feature occurs at the highest energies, above $4\times\E{19}$, where the
value of the spectral index becomes very high ($\gamma>4$), characterizing the so-called
\emph{suppression}.


%%%%%%%%%%%%%% SPEC PDG %%%%%%%%%%%%%%%
\begin{figure}
  \centering
  
  \begin{overpic}[clip, rviewport=0 0 1 1,width=0.8\textwidth]{spectrum_pdg}
    \put(73,48){
      \begin{tikzpicture}
        \filldraw[draw=black,color=white] (0,0) rectangle (1.7,0.4);
      \end{tikzpicture}
    }
    \put(55,62){
      \begin{tikzpicture}
        \filldraw[draw=black,color=white] (0,0) rectangle (1.7,0.4);
      \end{tikzpicture}
    }
    \put(38,70){
      \begin{tikzpicture}
        \filldraw[draw=black,color=white] (0,0) rectangle (1.7,0.4);
      \end{tikzpicture}
    }
    \put(35,71){\small \textbf{1st knee}}
    \put(52,64){\small \textbf{2nd knee}}
    \put(71,38){\small \textbf{ankle}}
    \put(79,49){\small \textbf{suppression}}
  \end{overpic}
  
  \caption{Compilation of measurements of the cosmic ray energy spectrum from Ref.~\cite{\PDGPaper}.
    The flux is scaled by a factor $E^{2.6}$ and the main features of the spectrum are indicated
    (see text for details). The references to the data can be found in Ref.~\cite{\PDGPaper}.}
  \label{fig:uhecr:overview:spec:pdg}
  \begin{center}
    \small Source: PATRIGNANI.~\cite{\PDGPaper}
  \end{center}
\end{figure}


A consistent interpretation of all these spectrum features requires the knowledge
of the cosmic rays composition. Particularly in the energy region comprehending
the first and second-knee, very efficient composition measurements were performed
by KASCADE~\cite{\KASCADEPaper} and KASCADE-Grande~\cite{\KASCADEGrandePaper} experiments.
By using an experimental setup
which includes surface detectors able to measure both the number of charged particles and
the number of muons in air showers, it was possible to measure the all particle spectrum
as well as to infer the spectra of individual groups of particles.
Results of both experiments are shown in~\cref{fig:uhecr:overview:spec:kascade}
where the plots on the top show spectra measurements from KASCADE~\cite{Antoni:2005wq}
around the first-knee energies and the ones on the bottom show
measurements from KASCADE-Grande~\cite{Fuhrmann:2013lgx} around the second-knee energies.
Although these results are strongly dependent on the choice
of the hadronic interaction model due to the unfolding algorithm applied
to estimate the individual spectra~\cite{Antoni:2005wq},
it is still possible to conclude that: (a) the first-knee is the result
of the suppression of the proton spectrum~\cite{Antoni:2005wq},
(b) the suppression of the components heavier than proton are consistent with
a rigidity dependent suppression~\cite{Antoni:2005wq} and (c) the second-knee coincides
with the suppression of the heaviest group of particles,
including iron nuclei.~\cite{Apel:2011mi,Apel:2013uni} 

%%%%%%%%%%%%%% SPEC KASCADE %%%%%%%%%%%%%%%
\begin{figure}
  \centering
  
  \begin{overpic}[clip, rviewport=0 0 1 1,width=0.8\textwidth]{kascade_spec}
    \put(18,60){}
  \end{overpic}
  
  \begin{overpic}[clip, rviewport=0 0 1 1,width=0.8\textwidth]{kascade_grande_spec}
    \put(18,60){}
  \end{overpic}
  
  \caption{Energy spectra of different components as measured by KASCADE~\cite{Antoni:2005wq} (top),
    around the first-knee energies, and by KASCADE-Grande~\cite{Fuhrmann:2013lgx} (bottom), around the
    second-knee energies. The hadronic interaction models used were \QGSJetOld and \QGSJetLong
    for the KASCADE and KASCADE-Grande analysis, respectively.}
  \label{fig:uhecr:overview:spec:kascade}
  \begin{center}
    \small Source: ANTONI.(top)~\cite{Antoni:2005wq}; FUHRMANN.(bottom)~\cite{Fuhrmann:2013lgx} 
  \end{center}
  
\end{figure}


Although the rigidity dependent suppression as an explanation for the first-knee
was a known hypothesis since suggested by Peters, in 1961~\cite{Peters1961},
the actual explanation is still a matter of discussion. The most simple
model would assume that the suppression is a consequence of the 
maximum attainable energy by the sources.~\cite{Gaisser:2013bla}  
Alternative models include explanations based on the effect of the
drifting of cosmic rays in the galactic magnetic field~\cite{Ptuskin1993,Candia:2002qd}
and on the escape of cosmic rays from the galaxy.~\cite{Giacinti:2014xya}
Most of the models, however, converge on the fact that
the cosmic ray flux is dominated by a galactic component up to second-knee energies.
The transition between this galactic and an extragalactic component would then occur
at energies around the second-knee and the ankle.
The measurements and the models concerning UHECRs
will be presented in~\cref{sec:uhecr:spectrum}.


%%================================%%
\subsection{Acceleration and sources}


The power-law shape of the energy spectrum indicates that
cosmic rays are not accelerated in thermal processes.
A stochastic acceleration mechanism was first proposed
by Fermi, in 1949.~\cite{Fermi:1949ee} The main idea is that
the multiple collisions of charged particles with moving magnetized regions
could accelerate them up to high energies. 
Although a power-law spectrum can be derived from this mechanism,
the overall acceleration efficiency is too low to explain the observed
energy density of cosmic rays.
The average energy gain is given by $\Delta E/E\sim \beta^2$,
where $\beta = v/c$ and $v$ is the velocity of magnetic cloud.
Because of this second power of $\beta$, this mechanism is
called \emph{second order Fermi mechanism}.

A similar but more efficient mechanism was developed almost
two decades later.~\cite{Axford1977,Krymsky1977,Bell:1978zc,Blandford:1978ky}
The main idea of this mechanism is that charged particles collide with multiple shock waves
and they gain energy by interacting with irregularities in the magnetic field.
The average energy gain is $\Delta E/E\sim \beta$ and again a power law energy
spectrum can be derived. The obtained spectral index is $\gamma=2-2.3$, which means that
propagation effects have to be responsible to account for the further steepening
of the observed spectrum.
This mechanism is called \emph{first order Fermi mechanism} or, alternatively,
\emph{diffusive shock acceleration} (DSA), and it is the basis of most of the models
that propose possible sources of cosmic rays.
A detailed approach to the DSA theory can be found
in Ref.~\cite{Drury:1983zz}.

The list of astrophysical objects usually considered as possible
acceleration sites for cosmic rays includes neutron stars,
active galactic nuclei (AGN), gamma ray bursts (GRB)
and others. A review of the candidates
for sources of cosmic rays can be found in Ref.~\cite{Torres:2004hk}.
Independently of the details of the acceleration mechanisms in different
types of astrophysical environments, it is possible to estimate the maximum energy
at which a certain type of object is able to accelerate particles. This was first done
by Hillas~\cite{Hillas1984} and the idea is simply to consider that
the Larmor radius of the particles must be smaller than the size of the site
in which they are confined to be accelerated.
As a result one obtain 
\begin{equation}
  \frac{E_\text{max}}{\E{18}} \approx \frac{\beta Z}{2} \frac{B}{\mu\text{G}}\frac{L}{\text{kpc}},
\end{equation}
where $\beta$ is the typical velocity of the shock waves in units of $c$, $Z$ is the charge
of the particles, $B$ is the magnetic field and $L$ is the typical size of the site.
The graphical illustration of this relation showing the astrophysical objects
in a plane of their size \emph{vs} their magnetic field is traditionally called \emph{Hillas plot}.
We show one version of it, focused on the candidate sources of UHECRs,
in~\cref{fig:uhecr:overview:hillas}.
The diagonal lines show the lower limits to accelerate protons and iron nuclei
up to \E{20}. 

%%%%%%%%%%%%%% HILLAS PLOT %%%%%%%%%%%%%%%
\begin{figure}
  \centering
  
  \begin{overpic}[clip, rviewport=0 0 1 1,width=0.8\textwidth]{hillas}
    \put(18,60){}
  \end{overpic}
  
  \caption{Hillas plot from Ref.~\cite{Mollerach:2017idb}. The sources shown are neutron stars (n-star),
    gamma-ray bursts (GRB), active galactic nuclei (AGN)
    split in radio-quiet Seyfert (Sy) and radiogalaxies (RG), supernova remnants (SNR),
    starburst galaxies and galaxy clusters.}
  \label{fig:uhecr:overview:hillas}
  \begin{center}
    \small Source: MOLLERACH.~\cite{Mollerach:2017idb} 
  \end{center}
\end{figure}


Regarding UHECRs, a famous family of models,
generically called \emph{top-down models}, assume that
the particles are not accelerated by astrophysical objects, but instead they are the products
of the decay of exotic super-massive particles. The origin of these exotic particles
could be, for example, topological defects from early Universe phase transitions
of dark matter. As a prediction of nearly all these models, a significant flux of
ultra-high energy photons should be observed. However, measurements  
of Pierre Auger Observatory have shown that the upper limits of the photon flux is significantly lower than
the predictions~\cite{Aglietta:2007yx} and, therefore, the top-down hypothesis can be ruled out.
A review of the top-down models can be found in Ref.~\cite{Olinto2000}.  


%%================================%%
\subsection{Propagation of UHECR}
\label{sec:uhecr:overview:propagation}


At the ultra-high energy regime, cosmic rays suffer from a number
of interaction processes during their propagation from the source to the Earth and
the consequent energy losses have a large influence on the measured
energy spectrum and composition.
Four interaction processes take place: photopion production, electron pair production,
photodisintegration and adiabatic losses due to the expansion of the universe.
Since the latter affects equally all types of particles with no dependence on
their energy, its influence on the energy spectra is the less relevant of all.
Concerning the other three remaining processes, it is helpful to
evaluate their effect separately for protons and nuclei.
In~\cref{fig:uhecr:propagation:attenuation} the energy losses of protons and iron nuclei
are quantified by evaluating the energy loss length,
which is defined as the average distance over
which the particle energy decreases by a factor $1/e$,
as a function of its energy.


%%%%%%%%%%%%%% ATTENUATION %%%%%%%%%%%%%%%
\begin{figure}
  \centering
  
  \begin{overpic}[clip, rviewport=0 0 1 1,width=0.8\textwidth]{att_lenght}
    \put(18,60){}
  \end{overpic}
 
  \caption{Energy loss lengths for protons and iron nuclei from Ref.~\cite{RafaelThesis}.}
  \label{fig:uhecr:propagation:attenuation}
  \begin{center}
    \small Source: BATISTA.~\cite{RafaelThesis} 
  \end{center}
\end{figure}

The photopion production process (indicated by orange curves in~\cref{fig:uhecr:propagation:attenuation})
occurs mainly because of the interaction of the cosmic ray particle with photons
from the cosmic microwave background (CMB). For protons, the processes that take place are
$p+\gamma\rightarrow p + \pi^0$ and $p+\gamma\rightarrow n+ \pi^+$, in which
the energy threshold is given by the condition that the center-of-mass energy 
has to be larger than the rest mass of the pion. From~\cref{fig:uhecr:propagation:attenuation} it is clear
that photopion production processes dominate the proton energy
losses at energies above $\approx 8\times\E{19}$,
and, as a consequence, it is expected the proton spectrum to be strongly
suppressed around this energies.
This effect was proposed in 1966 by Greisen~\cite{Greisen:1966jv},
Zatsepin and Kuzmin~\cite{Zatsepin:1966jv}, and it is known as the \emph{GZK suppression}.
Also from~\cref{fig:uhecr:propagation:attenuation}, one can see
that proton energy losses due to electron pair production
dominate at energies between $5\times\e{18}$ and
$5\times\E{19}$.

It can be seen that the combined effect of these two processes,
photopion production and electron pair production,
on the energy spectrum of cosmic rays composed purely by protons
can generate two evident features, a dip at $E\approx 5\times\E{18}$
and a sharp suppression above $E\approx 5\times\E{19}$, that
can explain the two features actually observed at the UHECR spectrum,
the ankle and the suppression.~\cite{Berezinsky:2002nc,Berezinsky:2005cq}
This scenario is called \emph{dip model} (see also discussion in~\cref{sec:uhecr:spectrum})
and the pure proton composition predicted by it has
two very relevant consequences to the possible identification of the sources
of UHECRs. The first one is the so-called
\emph{particle horizon}, meaning that protons above a certain energy
should be originated on average by sources inside a certain distance range.
In~\cref{fig:uhecr:propagation:horizon}
the particle horizon is quantified by curves that show the average energy
of protons as a function of the propagation distance for three cases in which the
particle left the source with \e{20}, \e{21} and \E{22}. As a conclusion from this plot,
one can see that protons reaching Earth with $E>\E{20}$ are approximately limited to
be originated at sources within 100 Mpc of distance. The second consequence is related
to the particle deflections by magnetic fields. It is possible to show that
heavier nuclei are so strongly deflected that their arrival direction does not contain
information about the direction of the astrophysical source. On the other hand,
since protons above $\approx\E{19}$ are only slightly deflected, in this pure-proton
scenario would be possible to use the arrival directions to infer the sources distribution.


%%%%%%%%%%%%%% GZK HORIZON %%%%%%%%%%%%%%%
\begin{figure}
  \centering
  
  \begin{overpic}[clip, rviewport=0 0 1 1,width=0.75\textwidth]{gzk_horizon}
    \put(18,60){}
  \end{overpic}
 
  \caption{Average proton energy as a function of the propagation distance
    through the CMB. The numbers above the curves indicate the proton initial energy.}
  \label{fig:uhecr:propagation:horizon}
  \begin{center}
    \small Source: CRONIN.~\cite{Cronin:2004ye} 
  \end{center}
\end{figure}

Unlike protons, nuclei also can suffer photodisintegration along their propagation.
This means that they can interact with photons from the extragalactic background light (EBL)
and be broken into lighter nuclei. In~\cref{fig:uhecr:propagation:attenuation} it is also
shown the energy loss due to this process for iron nuclei. 
One can see that photodisintegration dominates the energy losses over
electron pair production or photopion production at all the energies.
Because of photodisintegration, the resultant effect of the propagation of a
combination of different nuclei on the energy spectrum
turns to be more complicated to evaluate than in the proton case. However,
it is possible to show that for a mixed composition case,
the energy spectrum would also suffer a suppression at the highest energies,
mainly due to photodisintegration effects for heavy nuclei and photopion production for the light ones.
This suppression can also be considered to be a manifestation of the above mentioned GZK suppression,
since it would be a consequence of the propagation effects of cosmic ray particles.  


Concerning the other spectrum feature, the ankle, it is not possible to reproduce it
by considering nuclei propagation, instead of proton. The alternative explanation
for the ankle in the non-pure proton case is the effect of the
transition between galactic and extragalactic component. In~\cref{sec:uhecr:spectrum} we discuss
the recent measurements of the UHECR spectrum and composition
and how they constrain the above mentioned explanation for the
ankle and the suppression.



%%%%%%%%%%%%%%%%%%%%%%%%%%%%%%%%%%%%
\section{UHECR spectrum and composition}
\label{sec:uhecr:spectrum}

At the ultra-high energy range, two very evident features can be observed
in the spectrum: the ankle and the suppression. In~\cref{fig:uhecr:spectrum}
we show the most up to date spectrum as measured
by Pierre Auger Observatory~\cite{AugerSpec2017} and Telescope Array~\cite{TASpec2017},
where both features can be clearly seen. The ankle is a hardening of the spectrum
at $E\approx 5\times \E{18}$ which was first observed
in Haverah Park measurements in 1963~\cite{LinsleySpec1963}, and later confirmed
in measurements by AGASA~\cite{AgasaSpecPaper1995}, HiRes~\cite{\HiResSpecPaper},
Pierre Auger Observatory~\cite{Abraham:2008ru}
and Telescope Array.~\cite{AbuZayyad:2012qk} The suppression is an abrupt softening of the flux
at energies above $4\times \E{19}$ which was measured first by Hires~\cite{\HiResSpecPaper}
and later confirmed by Pierre Auger Observatory~\cite{Abraham:2008ru}
and Telescope Array.~\cite{AbuZayyad:2012qk}
The fact that the suppression was not observed on the AGASA spectrum~\cite{Takeda:1998ps}
was a source of intense discussions along years. However,
it is currently accepted that the AGASA measurements are not trustful because
of systematic uncertainties in the primary energy estimation.

%%%%%%%%%%%%%% SPEC UHECR %%%%%%%%%%%%%%%
\begin{figure}
  \centering
  
  \begin{overpic}[clip, rviewport=0 0 1 1,width=0.8\textwidth]{uhecr_spectrum}
    \put(18,60){}
  \end{overpic}
  
  \caption{UHECR spectrum as measured by Pierre Auger Observatory and Telescope Array.
    The Pierre Auger Observatory measurement combines several different data sets into
    one unique spectrum.~\cite{AugerSpec2017} The Telescope Array measurement includes
    only the analysis of the surface detectors data.~\cite{TASpec2017}}
  \label{fig:uhecr:spectrum}
  \begin{center}
    \small Source: By the author.
  \end{center}
\end{figure}


The most precise inferences of the composition of cosmic rays
at this energy range are based on the \xmax measurements
performed by fluorescence telescopes.
In~\cref{sec:shower:observables}, we present a detailed discussion about
the \xmax measurements, its composition interpretation, as well as
further measurements of air shower observables that could, in principle,
be used to infer composition. For now, it is only important to
point out that the cosmic ray composition
can be inferred by comparing the measured \xmax distributions to
predictions obtained from air shower simulations. This procedure can be done
either by comparing only the moments of the \xmax distributions, which
gives only the average composition, or by comparing the whole distributions,
which can provide the abundances of individual groups of primaries.
In all cases, the composition interpretation is dependent on the
hadronic interaction model used to generate the simulation predictions.

Although the energy spectrum with its both features can be successfully described
by multiple hypothesis, the composition measurements must be used to further constrain  
the models. Concerning the ankle, most of the models associate it to
the transition between the galactic and extragalactic flux, which
can take place in energies between the second-knee and the ankle.
The composition predictions by these models can vary substantially depending
on the exact position of the transition and on the composition content of
the extragalactic component. One historically relevant hypothesis to
describe the ankle without the galactic-to-extragalactic transition assumption is the
above mentioned dip model~\cite{Berezinsky:2002nc,Berezinsky:2005cq},
which proposes that the energy losses due to
electron pair production along the particle propagation would cause the hardening
of the spectrum around the ankle energies. It turns out that the success
of the dip model actually requires a nearly pure proton composition
around and above the ankle. Although at the time of the model proposal
this composition assumption seemed to be fulfilled by the \xmax measurements
of HiRes experiments~\cite{\HiResXmaxPaper},
recent measurements of Pierre Auger Observatory
have proven it wrong.~\cite{\AugerXmaxPRLPaper}

Concerning the suppression, the two currently most considered hypothesis
to explain it are the GZK effect (see~\cref{sec:uhecr:overview:propagation})
and the limit of the acceleration power by the sources. While in the GZK hypothesis
the composition can be either pure proton or mixed,
the source limit hypothesis imply in a rigidity dependent suppression and consequently an increasingly heavier
composition as the energy increases. Similarly to what happened to the dip model
interpretation of the ankle, the composition measurements by HiRes
supported the GZK hypothesis for many years,
which was also changed after the \xmax measurements by Pierre Auger Observatory were published. 

Many scenarios which propose to explain the UHECR spectrum and composition
can be found in the
literature.~\cite{Allard:2005cx,Aloisio:2011fv,Biermann:2011wf,Unger:2015laa,Globus:2015xga}
Two features are common to nearly all the recent models that are constrained by the
Auger \xmax measurements: the exact energy of the galactic-to-extragalactic transition
(the energy in which both fluxes become equivalent) is around $1-3 \times \E{18}$
and the suppression is mainly caused by a rigidity dependent cutoff due to the limit
of the acceleration power of the sources.
The observed mixed composition at the suppression energies and the confirmation 
that the suppression is not caused only by GZK effect with a pure proton composition  
has negative consequences, like the absence of correlation with nearby sources due to the deflection of
heavy nuclei in magnetic fields and the absence of the GZK horizon.
Those facts motivated the authors of Ref.~\cite{Aloisio:2011fv}
to call this generic scenario of \emph{disappointing model}.

In a recent analysis done by Pierre Auger Collaboration,
the measured spectrum and the \xmax distributions were
jointly fitted by using simulations which account for
all the propagation effects.~\cite{Aab:2016zth}
We show in~\cref{fig:uhecr:combined} the results
of the fit by using the hadronic interaction model \EposLHCLong~\cite{\EposLHCPaper}
to predict the \xmax distributions. The hypothesis of the limit of the acceleration power
by the sources as the reason of the suppression is favored by the results of this analysis.
However, the lack of information about the composition given by the \xmax measurements
at energies above $4\times\E{19}$ is a limiting factor for this type of analysis, since the
spectrum is measured up to above \E{20}. Therefore, precise composition measurements
at the highest energies are a requirement to refine the interpretation
of the end of the cosmic ray spectrum. 

%%%%%%%%%%%%%% COMBINED FIT %%%%%%%%%%%%%%%
\begin{figure}
  \centering
  
  \begin{overpic}[clip, rviewport=0 0 1 1,width=0.9\textwidth]{combined_fit}
    \put(18,60){}
  \end{overpic}
  
  \caption{Results of the combined fit of the spectrum and \xmax distributions
    performed by Pierre Auger Observatory.~\cite{Aab:2016zth} On the top plot
    the measured spectrum and the fitted contributions of different group of primaries are shown.
    On the bottom plots the measured mean (\xmaxmean) and standard deviation (\xmaxsig)
    of the \xmax distributions are shown together with the results of the fit. The color codes
    for the groups of the primaries are indicated on the right side of the top plot.
    The hadronic interaction model used to describe the \xmax distributions is
    \EposLHCLong.~\cite{\EposLHCPaper} The results by using other models
    can be found in Ref.~\cite{Aab:2016zth}.}
  \label{fig:uhecr:combined}
  \begin{center}
    \small Source: AAB.~\cite{Aab:2016zth} 
  \end{center}
\end{figure}

%%%%%%%%%%%%%%%%%%%%%%%%%%%%%%%%%%%%
\section{The Pierre Auger Observatory}
\label{sec:uhecr:auger}

The Pierre Auger Observatory was designed to measure
cosmic rays at the ultra-high energy range.
It is located near the town of Malargue, Argentina,
and its total area is approximately 3000 km$^2$.
The average altitude of the experiment is 1420 m, which
corresponds to a vertical atmospheric depth of approximately 880 g/cm$^2$.
The construction of the observatory started in 2002 and finished in 2008.
Its hybrid detection system, which combines a large surface detector 
and a set of fluorescence telescopes, is one of the most important experimental features
of the observatory. A detailed overview of the experiment
can be found in Ref.~\cite{\AugerPaper}.

%%%%%%%%%%%%%% AUGER ARRAY %%%%%%%%%%%%%%%
\begin{figure}
  \centering
  
  \begin{overpic}[clip, rviewport=0 0 1 1,width=0.6\textwidth]{auger_array}
    \put(18,60){}
  \end{overpic}
  
  \caption{Configuration of the Pierre Auger Observatory. The black dots represent
    stations of the standard surface detector, the gray dots represent the AMIGA stations
    and the blue and red lines indicate the
    field of view of each telescope which compose the fluorescence detector.
    The atmospheric monitoring devices (XLF, CLF and BLF) are represented by
    red dots. The blue circle show the approximate area of the AERA setup.}
  \label{fig:uhecr:auger:array}
  \begin{center}
    \small Source: STREICH.~\cite{AlexThesis}
  \end{center}
    
\end{figure}

%%HERE SOURCE

The standard surface detector (SD) is composed by a set of $\sim 1660$
water-Cherenkov detectors (WCD) which are regularly arranged in a triangular grid.
The nearest neighbor distance of the stations for the standard array,
which covers the full observatory area, is 1.5 km, while a small area of 23.5 km$^2$
called \emph{SD infill} contains stations separated by 750 m.
In~\cref{fig:uhecr:auger:array} we show the configuration of the Pierre Auger Observatory,
including the stations of the SD array indicated by black dots.
Each WCD consists of a cylindrical tank with a diameter of 3.6 m,
filled with 12 m$^3$ of ultra-high purity water, up to a height of 1.2 m. The particle
detection is done by collecting the Cherenkov light produced by the passage of charged
particle through the water. For that, three photomultiplier tubes (PMT) are installed
on the top of each station and their signal are extracted and digitized using
flash analog-to-digital converters (FADC)
with a sampling rate of 40 MHz. Since the SD operation does not
require any special environmental condition, its duty cycle turns to be very
close to 100\%.
More details about the SD can be found in Ref.~\cite{Allekotte:2007sf}.

The fluorescence detector (FD) is composed by 27 telescopes installed in
4 buildings around the edge of the array. Each of these buildings contains
a set of six telescopes with a field of view which covers \dg{30} horizontally
and goes from \dg{1.5} to \dg{30} vertically. The 24 telescopes of this kind
were part of the original project of the observatory and they are designed to measure
the fluorescence light emitted by the atmosphere during the development of air showers
with primary energies above approximately \E{18}. Another set of 3 telescopes
that can operate with a vertical field of view going from \dg{30} to \dg{60}
was later installed in one the buildings.
This enhancement is called HEAT~\cite{Mathes:2011zz} and it is
designed to extend the lower limit of the fluorescence measurements to approximately \E{17}.
In~\cref{fig:uhecr:auger:array}
blue and red lines are used to show the horizontal field of view of the standard telescopes
and the HEAT ones, respectively. Because of its sensitivity, the background light has to be avoided
during the FD operation. This means that moonless nights with proper atmospheric conditions
are required, implying in a duty cycle of about 13\% for the FD measurements. 
More details about the FD can be found in Ref.~\cite{Abraham:2009pm}.

The SD and FD measurements provide complementary information 
which can be combined to achieve a precise reconstruction of the
properties of the air shower and the primary particle.
However, because of the limited FD duty cycle, both information
are only available to a fraction of the events. Because of that,
two different event reconstruction modes are defined, the SD and
the hybrid reconstruction.

The hybrid reconstruction uses the measurements of the
fluorescence telescopes with only additional timing information
from one SD station. From the time structure of the FD signal it is
possible to reconstruct precisely the geometry of the shower and
from the amount of light collected it is possible to reconstruct the
longitudinal profile of the shower. A detailed procedure has to be applied
at this point to convert the number of photons collected in the telescopes
into energy deposit in the atmosphere by
taking into account all the attenuation effects.
One example of longitudinal profile as measured by Auger is shown
on the left plot of~\cref{fig:uhecr:auger:rec}.
After obtained the energy deposit as a function of slant depth,
a Gaisser-Hillas function~\cite{GaisserHillas1977} is fitted to that and the result
is used to determine the shower maximum \xmax and the calorimetric energy,
which is given by the integral of the fitted function. This calorimetric
energy is then corrected by the invisible energy due to neutrinos and high energy
muons and the final FD energy ($E_\text{FD}$) is then obtained.

%%%%%%%%%%%%%% AUGER REC %%%%%%%%%%%%%%%
\begin{figure}
  \centering
  
  \begin{overpic}[clip, rviewport=0 0 1 1,width=0.45\textwidth]{auger_long_profile}
    \put(18,60){}
  \end{overpic}
  \begin{overpic}[clip, rviewport=0 0 1 1,width=0.45\textwidth]{auger_lat_dist}
    \put(18,60){}
  \end{overpic}
  
  \caption{Examples of the energy deposit longitudinal profile
    as measured by the FD (left plot)~\cite{\AugerXmaxPRDPaper}
    and of the lateral distribution as measured by the SD (right plot).~\cite{\AugerPaper}}
  \label{fig:uhecr:auger:rec}
  \begin{center}
    \small Source: AAB.~\cite{\AugerXmaxPRDPaper,\AugerPaper} 
  \end{center}
\end{figure}


The SD reconstruction is based on the size and the times of signals registered
by the individual stations. The geometry of the shower is first determined
by fitting the time of the signals assuming a spherical shape for the shower front.
After obtaining the shower geometry it is possible to evaluate the lateral distribution,
which is basically the evolution of the station signals with the distance to the
shower axis. On the right plot of~\cref{fig:uhecr:auger:rec}
we show one example of the lateral distribution function
as measured by Auger. A function of the NKG type~\cite{Kamata1958,Greisen1956} is fitted
and the evaluation of fitted function at a certain distance is defined
as shower size. For the standard array, with stations at 1500 m distance from each other,
the signal is evaluated at 1000 m and the shower size is represented by $S(1000)$.
The determination of the energy of SD reconstructed showers is done
by means of an energy calibration procedure, in which a set of hybrid reconstructed
showers are used to relate the shower size estimated from SD to the calorimetric
energy estimated by FD.


The first step of the energy calibration is to correct the $S(1000)$ parameter
for the atmospheric attenuation effects. This is done by following the
constant instant cut (CIC) method~\cite{CIC1961} which, by assuming that the flux
of cosmic rays above a certain energy is isotropic, correct the zenith angle dependence
of $S(1000)$ by converting it to a certain reference zenith angle. For the standard array,
this reference value is \dg{38} and the corrected shower size is represented by $S_{38}$.
On the left plot of~\cref{fig:uhecr:auger:cal}
we show the evolution of $S(1000)$ with the zenith angle $\theta$
and the fitted CIC curve $f_\text{CIC}(\theta)$. The corrected shower size is then computed
as $S_{38} = S(1000)/f_\text{CIC}(\theta)$, where $\theta$ is the measured zenith angle of the event.
The next step is selecting a sample of high quality hybrid events and use them
to relate the $S_{38}$ and $E_\text{FD}$. This relation is parametrized by a power law function
of the type $E_\text{FD} = A (S_{38})^B$ which is fitted to the data.
On the right plot of~\cref{fig:uhecr:auger:cal}
we show one example of the $S_{38}$ vs $E_\text{FD}$ plot with the fitted
function. The parameters resulting from the fit are used to determine
the SD energy from the $S_{38}$ parameter. 

%%%%%%%%%%%%%% AUGER CAL %%%%%%%%%%%%%%%
\begin{figure}
  \centering
  
  \begin{overpic}[clip, rviewport=0 0 1 1,width=0.45\textwidth]{auger_cic}
    \put(18,60){}
  \end{overpic}
  \begin{overpic}[clip, rviewport=0 0 1 1,width=0.45\textwidth]{auger_calibration}
    \put(18,60){}
  \end{overpic}
  
  \caption{The attenuation curve derived from the SD data in which the reference
    zenith angle ($\theta = \dg{38}$) is indicated by a vertical dashed line (left)
    and the $S_{38}$-$E_\text{FD}$ plot used in the standard energy calibration (right).
    Both figures were obtained from Ref.~\cite{\AugerPaper}.}
  \label{fig:uhecr:auger:cal}

  \begin{center}
    \small Source: AAB.~\cite{\AugerPaper} 
  \end{center}
\end{figure}


More details about the standard event reconstruction
can be found in Ref.~\cite{\AugerPaper}. Besides the SD and hybrid
reconstruction, a special procedure is applied to highly inclined
air showers, with zenith angle larger than \dg{60}. The details about
this reconstruction procedure can be found in Ref.~\cite{Aab:2014gua}.

Apart from the SD and FD, a number of other detectors are currently
part of the Pierre Auger Observatory.
The muon detector called AMIGA~\cite{Suarez:2013ecb,Wundheiler:2011zz,Platino:2011zz}
(Auger Muons and Infill for the Ground Array) and the radio detector
called AERA (Auger Engineering Radio Array)~\cite{Kelley:2011zz} are already installed
and operating. Moreover, prototypes of a recently developed muon detector
named MARTA~\cite{Abreu:2017vsj} will be installed in the near future.


%%================================%%
\subsection{AugerPrime}

An example of the importance of the Auger spectrum
and mass composition measurements of UHECRs was
already given in~\cref{sec:uhecr:spectrum},
where the results of the combined fit were discussed.~\cite{Aab:2016zth} 
Besides that, further recent analysis by Auger have provided very
important results concerning the astrophysical origins of
UHECRs. Two of these results are worthwhile mentioning:
a large scale dipolar anisotropy
of $\sim 6\%$ found for energies above $8\times \E{18}$~\cite{Aab:2017tyv}
and evidences of anisotropy in the arrival directions
of cosmic rays with $E>\E{19}$ which indicates correlation with
nearby extragalactic sources.~\cite{Aab:2018chp} Even though these recent results
represent a great progress on the understanding of the nature and origin
of UHECRs, some of the main questions remain still opened.

An important piece of information to be provided experimentally that
is still missing is the mass composition of cosmic rays above $3 \times \E{19}$.
Although the composition can be inferred trustfully from \xmax measurements
(see~\cref{sec:shower:observables:xmax}),
the low duty cycle of the fluorescence telescopes prevents
it to be used above the suppression energy.
Overcoming this issue is the main goal of the upgrade of the Pierre Auger Observatory,
called AugerPrime.~\cite{Aab:2016vlz,Martello:2017pch}
Besides elucidating the mass composition at the highest energies,
the goals of AugerPrime also include searching for a flux contribution of protons up to the highest
energies and studying the properties of air showers and hadronic multiparticle production.
The latter one is intrinsically related
to the muon deficit problem (see~\cref{sec:shower:observables:nmu}),
which is the main topic of this thesis.

The experimental configuration of AugerPrime consists of four components.
The first one is the Surface Detector Electronics Upgrade (SDEU), intending to provide
a large number of improvements to the SD electronics, which includes the increasing of
analog-to-digital converter sampling rate from 40 MHz to 120 MHz.~\cite{Suomijarvi:2017ixb}
The second component is the changing
of the FD operation mode intending to increase its duty cycle up to 20\%.
The remaining two components are related to the detection of muons with surface detectors and
they are the installations of the Scintilator Surface Detectors (SSD) and
Underground Muon Detectors (UMD).

The SSD consists of scintilator detectors installed on the top of each existing WCD.
Since the scintilator and water-Cherenkov tanks respond differently to the muonic and
electromagnetic component of the shower, the combination of both measurements
can be used to estimate the muon content.~\cite{Gonzalez:2016ora} 
The UMD consists of scintilator buried 1.3 m below the ground, near the SD stations,
which are able to detect muons isolately with an energy threshold around 1 \GeV. The design
of the UMD will follow the same as the AMIGA detector~\cite{Suarez:2013ecb,Wundheiler:2011zz,Platino:2011zz}
that has already been developed during the last years. While the SSD will cover the full
standard array, with over 1600 stations, the UMD will be installed on the smaller area
of the SD infill.

The precise measurements of the muonic component is surely one of the most relevant
result to be provided by AugerPrime. As a consequence, it is expected a significant
improvement on the current understanding of the air shower physics, and in particular
the muon production. Both analysis in~\cref{sec:interpretation,sec:observable}
present methods that can be used to interpret and analyze data from
muon detectors similar to the ones of AugerPrime.


