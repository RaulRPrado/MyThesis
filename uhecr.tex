\chapter{Cosmic rays, air showers and the Pierre Auger Observatory}
\label{sec:uhecr}


\section{Ultra high energy cosmic rays}

\section{Extensive air showers}


In this chapter, a brief introduction to Extensive Air Showers (EAS)
will be given. Althought the physics behind EAS is very rich,
in this text we will give more focus on the selected topics
which are intrinsically related to the main topic of this thesis.
More detailed and extended materials about EAS can be found
in Refs.~\cite{GaisserBook,GriederBook}.


%%%%%%%%%%%%%%%%%%%%%%%%%%%%%%%%%%%%
\subsection{Extensive air shower phenomenology}
\label{sec:showers:phen}

An EAS start with the collision of the primary cosmic ray particle
with an atmospheric nucleus. After this first interaction, the secondary
particle produced also interact with the atmosphere producing more
particles and so on. This succession of interactions with particle production
create the cascade of particles that forms the air shower.
For the sake of clarity, the description of the cascade is tradionally
done by spliting it into different components: the \emph{electromagnetic},
the \emph{hadronic} and, eventually, the \emph{muonic component}. It is also traditional
to describe the EAS development in terms of its longitudinal and lateral profiles.
The former refers to the shower development along the direction of its axis
while the latter refers to the distribution of particles
along a perpendicular direction to shower axis.


The first interaction can produce tens or hundreds of
secondary particles, which are nearly all hadrons
with only an insignificant contribution of photons and leptons.
Among the produced hadrons, the dominant contribution is by far from pions ($>60\%$),
followed by kaons and nucleons at similar proportions ($\sim 10\%$). Other
types of mesons and baryons together count tipically for less than 5\% of
the particles~\cite{Calcagni:2017tws}.
Besides that, a very small fraction of nuclei can be produced
by the fragmentation of the target nucleus or the primary.
After the first interaction, the following air shower development is mostly driven
by the subsequent processes involving the pions. The three types of pions ($\pi^+$, $\pi^-$ and $\pi^0$)
are produced at similar proportions, which means that about 2/3 of them are charged and
1/3 is neutral. The charged ones will be responsible to create more
hadrons and eventually muon and the neutral ones to create photons and
to fed the electromagnetic component of the shower.
First we will focus on the latter case.

%%================================%%
\subsubsection[Electromagnetic component and the \xmax]{\boldmath Electromagnetic component and the \xmax}
\label{sec:showers:phen:em}


Neutral pions can decay via the electromagnetic interaction into two photons ($\pi^0\rightarrow 2\gamma$)
with a very short decay length ($c\tau_0=25$ nm). Since in the atmosphere this decay length
is much shorter than the interaction one, nearly all the neutral pions end up decaying into
photons. The high energy photons produced interact electromagnetically, being the dominant
process the pair production ($\gamma\rightarrow e^++e^-$). The electrons and positrons
produced, in turn, also interact electromagnetically, mainly through bremstralung, which
produces more photons. The succession of these electromagnetic interactions
will create a self-sustainable cascade of photons, electrons and positrons
which forms the electromagnetic component of the extensive air shower.
Since more neutral pions are produced in hadronic interactions of secondary
particles with atmospheric nuclei (again in the 1/3 proportion wrt the charged pions),
the same chain of electromagnetic interactions will take place again
and new lower energy electromagnetic cascades are produced.
This means that the energy carried by the hadronic particles
is partially, but constinuously, trasfered to the electromagnetic particles.
After a few generations of hadronic interactions ($\sim 6$), about 90\% of
the primary energy is carried by the electromagnetic component.


The lenght scale of the bremstralung interaction, responsible to produce photons out of
electrons, is given by the radiation lenght, $X_0$, that is the average distance needed
to electrons to lose all but $1/e$ of its energy. In the atmosphere $X_0\approx 37\text{g/cm}^2$.
On the other hand, the lenght scale of pair production is of the same order
of $X_0$, only slightly larger ($\approx 1.3 X_0$). Because of the short lenght scale
given by $X_0$, the electromagnetic shower develops very rapidly, which
implies a fast increase of the number of particles and a fast decrease of their
average energy. As lower the particle energies, more relevant the energy losses
due to ionization become. The critical energy in which the ionization losses
becomes equivalent to the bremstralung ones is $E_c = 85 \MeV$ in the
atmosphere. After the average energy of electrons becomes close to $E_c$,
the number of particles starts to decrease, which creates the peaked shape
of the longitudinal profile of the electromagnetic component.

In~\cref{} we show the average longitudinal profile of the number
of electrons ($N_e$) in air showers initiated by protons and iron nuclei.
It is also shown for comparison the longitudinal profile of the
deposited energy (\dEdX) in the atmosphere by all the charged particles.
Because the number of particles in the shower is totally dominated
by the electromagnetic component, one can see that the shape of the
$N_e$ and \dEdX profiles are basically the same.
The depth correspondent to the maximum of both profiles is called
the shower maximum and it is symbolized by \xmax. In air shower
experiments, the \dEdX profile can be accessed by telescopes which
measure the fluorescence light emitted by the interaction of the air shower
with the atmosphere. The \xmax of the shower can then be easily determined
from the \dEdX profile.

The dependences of \xmax with the primary energy and with the type
of the primary particle can be expressed by the relation
\begin{equation}
  \langle\xmax\rangle = \lambda_\text{I}(A)+D_{10}\;\log_{10}\frac{E_0}{A},
  \label{eq:shower:xmax}
\end{equation}
where $\lambda_\text{I}$ is the mean free path of the first interaction,
$E_0$ is the energy of the primary particle and $A$ is its nuclear mass.
The quantity $D_{10} = \frac{ \text{d} \xmaxmean }{ \text{d} \log_{10}E_0 }$
is called \emph{elongation rate} and it does not depend on $E_0$ and $A$.
Although the~\cref{eq:shower:xmax} can be derived by simplistic analtytic models~\cite{Matthews:2005sd},
it can also be certified by using full Monte Carlo simulations.

In~\cref{} we show the \xmaxmean as a function of the primary energy
for a set of simulatied air showers. The logaritmic energy dependece is clear,
as well as the separation between proton and iron initiated showers.
This dependence of \xmaxmean with the primary mass comes from both terms
on the right side of~\cref{eq:shower:xmax}, since $\lambda_\text{I}$
decreases with increasing $A$. The difference between \xmaxmean for
proton and iron initiated showers is $\sim 100 \text{g/cm}^2$, as can
be seen in~\cref{}. It is also clear from~\cref{} that the elongation rate
$D_{10}$ is nearly constant with energy and does not depend on the primary type.
For using \xmax measurements to infer the primary composition of
cosmic rays, both the differences on \xmax for different primaries and
the constancy of $D_{10}$ are important features. More about measurements
of \xmax will be given in~\cref{}. 


%%================================%%
\subsubsection{Hadronic component and muon production}
\label{sec:showers:phen:had}

Unlike the neutral pions, the charged ones can only decay weakly,
with a much longer decay length ($c\tau_0=7.8$ m)
than the electromagnetic $\pi^0$ decay.  
At high energies, however, their interaction length
is much smaller than the decay one. This means that most
of the charged pions actually interact with atmospheric nuclei
producing more particles at similar proportions as the ones produced
in the first interaction. While the neutral pions produced
fed the electromagnetic shower, as explained above, the charged ones
can keep interacting and producing more hadrons. The hadrons
produced by this chain of hadronic interactions compose the
hadronic component of the shower.

As lower the energy of the charged pions, larger the probability
that it decays instead of interacting again. The dominant decay channel
for charged pions is into muons and neutrinos ($\pi^+$ and $\pi^-$)
and the critical energy in which the probability of decaying is
equivalent to interacting is 85 \GeV. These decays are responsible
to produce the great majority of the muons of an air shower.
The tipical number of interactions of charged pions before
decaying into muons is 4--8. 


Apart from pion decays, a considerably fraction of the muons reaching
the ground is produced by the decay of charged kaons. In~\cite{} the muon
production in air showers was studyied in details and it was found
by using simulations that $\sim 90\%$ of the muons come from the decay
of charged pions and $\sim 10\%$ of charged kaons. While these particles
are called \emph{mother}, the particles which interacted hadronically
with an atmospheric nucleus to produce them is are called \emph{grandmother}.
Concerning then the properties of the grandmother particles,
it was found that more than 70\% of them are pions, $\approx 20\%$ are nucleons
and $\approx 6\%$ are kaons. In~\cref{} the energy distributions of the
grandmother particles are shown and we can see the contribution from these
different types of particles. The dominance of pions as gradmother is clear,
as well as the fact that most of the interactions of the grandmother particles
occur at energies of order of $\sim 100 \GeV$.


Although the interaction of the grandmother particle is very important,
it has to be noted that the whole chain of hadronic interactions, which are
mostly pion-air interactions, influence the muon production. This means
that any shower observable related to the muonic component is sensitive to the
properties of hadronic interactions which occur along the shower. The particle
production in these hadronic interactions is particularly important and,
in the context of this thesis, it is worthwhile to point out the effect of
(anti)baryon production. Most of the (anti)baryons produced are
nucleons (\proton, \antiproton, \neutron and \antineutron) and they
are all produced in similar proportions. Since these particles
can only interact again, their energy is surely use to fed the
hadronic component, which means that this energy is partially converted
in muons production. Because of that, the increase of (anti)baryon
production in hadron-air interactions has been considered a very important
mechanism to increase the final number of muons in the shower~\cite{\EposPaper,Drescher:2007hc}.
\note{spectrum of muons}


In~\cref{} we show the average longitudinal profile of
the number of muons, as well as the muon production,
for proton and iron initiated showers. The number of muons (\nmu) observed
at a given atmospheric depth is a tipical quantity of the shower.
From simplistic analytic models and full Monte Carlo simulations,
we can see that \nmu relates to the primary energy and mass as
\begin{equation}
  \nmumean \approx A^{1-\beta} E_0^\beta,
\end{equation}
where $\beta\approx 0.9$. The energy and primary mass dependence
of \nmumean can be seen in~\cref{}, where we show the \nmumean
as a function of $\log_{10}E_0$, also for proton and iron
initiated showers.
Results of recent measurements of \nmu, as well as other
observables related to the muonic component will be presented
in~\cref{sec:shower:observables}.


%%%%%%%%%%%%%%%%%%%%%%%%%%%%%%%%%%%%
\subsection{Extensive air shower simulations}
\label{sec:shower:simulations}

-softwares: CORSIKA, CONEX, etc

%%================================%%
\subsubsection{Hadronic interaction models}

-low energy

-high energy


%%%%%%%%%%%%%%%%%%%%%%%%%%%%%%%%%%%%
\section{The Pierre Auger Observatory}


%%%%%%%%%%%%%%%%%%%%%%%%%%%%%%%%%%%%
\subsection{Measurements of EAS observables and UHECR composition}
\label{sec:shower:observables}


%%================================%%
\subsubsection{Shower maximum (\xmax) and UHECR composition}

%%================================%%
\subsubsection{Number of muons (\nmu) and the muon deficit problem}

%%================================%%
\subsubsection{Muon production depth (\xmumax)}

%%================================%%
\subsubsection{Further measurements}

-risetime asymmetry

-delta method





