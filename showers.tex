\chapter[Extensive air showers]{Extensive air showers}
\label{sec:showers}

In this chapter, we give an introduction to Extensive Air Showers (EAS).
Although the physics behind these objects is very rich,
in this text we will give more focus on the selected topics
which are more relevant in the context of this thesis.
More detailed and extended materials about the topic can be found
in Refs.~\cite{GaisserBook,GriederBook}.

In~\cref{sec:showers:phen} we start by a phenomenological description
of the EAS development, focusing on the particle production in
both electromagnetic and hadronic component of the cascade. 
In~\cref{sec:shower:simulations} we present the most relevant aspects
of EAS simulations, which are currently necessary to study
its physics and to interpret the measurements.
Finally in~\cref{sec:shower:observables} we present a compilation
of measurements of EAS, focused on the UHECRs.
At this point we approach the currently inconsistency between measurements and
simulations of the muonic related observables, which is the main motivation
for this thesis.


%%%%%%%%%%%%%%%%%%%%%%%%%%%%%%%%%%%%
\section{Phenomenology of EAS}
\label{sec:showers:phen}

An EAS starts with the collision of the primary cosmic ray 
with an atmospheric nucleus. After this first interaction, the secondary
particles produced also interact with the atmosphere producing more
particles and so on. This succession of interactions with secondary production
creates the cascade of particles that forms the EAS.
For the sake of clarity, the description of the cascade is traditionally
done by splitting it into different components: the \emph{electromagnetic},
the \emph{hadronic} and, eventually, the \emph{muonic component}. It is also traditional
to describe the EAS development in terms of its longitudinal and lateral profiles.
The former refers to the shower development along the direction of its axis
while the latter refers to the distribution of particles
in a plane perpendicular to the shower axis.


The first interaction can produce tens to hundreds of
secondary particles, which are nearly all hadrons
with only an insignificant contribution of photons and leptons.
Among the produced hadrons, the dominant contribution is by far from pions ($>60\%$),
followed by kaons and nucleons at similar proportions ($\sim 10\%$). Other
types of mesons and baryons together count typically for less than 5\% of
the particles.~\cite{Calcagni:2017tws}
Besides that, a very small fraction of nuclei can be produced
by the fragmentation of the target atmospheric nucleus or the primary (in case it is also a nucleus).
After the first interaction, the following EAS development is mostly driven
by the subsequent processes involving the secondary pions.
The three types of pions ($\pi^+$, $\pi^-$ and $\pi^0$)
are produced at similar proportions, which means that about 2/3 of them are charged and
1/3 is neutral. The charged ones will be responsible to create more
hadrons and eventually muons, while the neutral ones will decay into photons,
feeding the electromagnetic component of the shower.
The latter case is described in~\cref{sec:showers:phen:em}
and the former is described later in~\cref{sec:showers:phen:had}. 

%%================================%%
\subsection[Electromagnetic component and the \xmax]{\boldmath Electromagnetic component and the \xmax}
\label{sec:showers:phen:em}


Neutral pions can decay via the electromagnetic interaction
into two photons ($\pi^0\rightarrow \gamma+\gamma$)
with a very short decay length ($c\tau_0=25$ nm). Since in the atmosphere this decay length
is much shorter than the interaction one, nearly all the neutral pions end up decaying into
photons. The high energy photons produced interact electromagnetically, being the dominant
process the pair production ($\gamma\rightarrow e^++e^-$).
The electrons\footnote{In this text the term \emph{electrons} actually refers to electrons and positrons.} 
produced also interact electromagnetically, mainly through bremstralung, which
produces more photons. The succession of these two electromagnetic processes
creates a self-sustainable cascade of photons and electrons
that forms the electromagnetic component of the EAS.
Since more neutral pions are produced in hadronic interactions of secondary
particles with atmospheric nuclei (again in the 1/3 proportion wrt the charged pions),
the same chain of electromagnetic interactions takes place again
and new lower energy electromagnetic cascades are produced.
This means that the energy carried by the hadronic particles
is partially, but continuously, transferred to electromagnetic particles.
After a few generations of hadronic interactions ($\sim 6$), about 90\% of
the primary energy is carried by the electromagnetic component.


The length scale of the bremstralung interaction, responsible to produce photons out of
electrons, is given by the radiation length, $X_0$, that is the average distance needed
to electrons to lose all but $1/e$ of its energy. In the atmosphere $X_0\approx 37\text{g/cm}^2$.
On the other hand, the length scale of pair production is of the same order
of $X_0$, only slightly larger ($\approx 1.3 X_0$). Because of the short length scale
given by $X_0$, the electromagnetic shower develops very rapidly, which
implies a fast increase of the number of particles and a fast decrease of their
average energy. As lower the energies of the particles, more relevant are the energy losses
due to ionization. The critical energy in which the ionization losses
becomes equivalent to the bremstralung ones is $E_c = 85 \MeV$ in the
atmosphere. After the average energy of electrons becomes of the order of $E_c$,
the number of particles starts to decrease, which creates the peaked shape
of the longitudinal profile of the electromagnetic component.


In the left plot of~\cref{fig:shower:phen:prof}
we show the average longitudinal profile of the number
of photons ($N_\gamma$) and electrons ($N_e$) in EAS initiated by protons and iron nuclei.
It is also shown for comparison the longitudinal profile of the
deposited energy (\dEdX) in the atmosphere by all the charged particles.
Because the number of particles in the shower is dominated
by the electromagnetic component, one can see that the shape of the
\dEdX profiles is basically the same as the $N_e$ one.
The depth correspondent to the maximum of these both profiles is called
the shower maximum and it is symbolized by \xmax. In EAS
experiments, the \dEdX profile can be accessed by telescopes which
measure the fluorescence light emitted by the interaction of the EAS
with the atmosphere. The \xmax can then be determined
from the \dEdX profile.

%%%%%%%%%%%%%% LONG PROFILES %%%%%%%%%%%%%%%
\begin{figure}[!ht]
  \centering
  
  \begin{overpic}[clip, rviewport=0 0 1 1,width=0.45\textwidth]{electron_profile}
    \put(18,60){}
  \end{overpic}
  \begin{overpic}[clip, rviewport=0 0 1 1,width=0.45\textwidth]{muon_profile}
    \put(18,60){}
  \end{overpic}
  
  \caption{Average longitudinal profiles of EAS obtained
    from simulations performed with \Conex~\cite{\ConexPaper} and
    the hadronic interaction models \Fluka~\cite{\FlukaPaper} and \EposLHCLong.~\cite{\EposLHCPaper}}
  \label{fig:shower:phen:prof}
  \begin{center}
    Source: By the author. 
  \end{center}
 
\end{figure}

The dependencies of \xmax with the primary energy ($E_0$) and with the nuclear
mass of the primary particle ($A$) can be expressed by the relation
\begin{equation}
  \langle\xmax\rangle = \lambda_\text{I}(A)+D_{10}\;\log_{10}\left(\frac{E_0}{A}\right),
  \label{eq:shower:xmax}
\end{equation}
where $\lambda_\text{I}$ is the mean free path of the first interaction.
The quantity $D_{10} = \text{d}\xmaxmean/\text{d} \log_{10}E_0$
is called \emph{elongation rate} and it does not depend on $E_0$ and $A$.
Although the~\cref{eq:shower:xmax} is derived by simplistic analytic models~\cite{Matthews:2005sd},
it can be certified by using full Monte Carlo simulations that this relation is valid
in very good approximation.

On the left plot of~\cref{fig:shower:phen:xmax}
we show the \xmaxmean as a function of the primary energy
for a set of simulated EASs. The logarithmic energy dependence is clear,
as well as the separation between proton and iron initiated showers.
This dependence of \xmaxmean on the primary mass comes from both terms
on the right side of~\cref{eq:shower:xmax}. The $\lambda_\text{I}$ term
decreases with increasing $A$, because the interaction cross section
is larger for heavier nuclei. The second term, $\log_{10}\left(\frac{E_0}{A}\right)$,
is a manifestation of the so called \emph{superposition principle},
which says that an air shower initiated by a nuclei of mass $A$ of energy $E_0$
is actually equivalent to a superposition of $A$ showers
initiated by protons with primary energies $E_0/A$.
Since \xmax is a longitudinal quantity, it is not affected, on average,
by the superposition of showers and it turns out that the term $\log_{10}\left(\frac{E_0}{A}\right)$
is actually equivalent to only one proton shower with energy $E_0/A$.


%%%%%%%%%%%%%% XMAX MOMENTS %%%%%%%%%%%%%%%
\begin{figure}[!ht]
  \centering
  
  \begin{overpic}[clip, rviewport=0 0 1 1,width=0.45\textwidth]{xmax_mean}
    \put(18,60){}
  \end{overpic}
  \begin{overpic}[clip, rviewport=0 0 1 1,width=0.45\textwidth]{xmax_sig}
    \put(18,60){}
  \end{overpic}
  
  \caption{\xmaxmean (left) and \xmaxsig (right) as a function
    of the primary energy obtained from air shower simulations performed with
    \Conex~\cite{\ConexPaper} and the hadronic interaction models \Fluka~\cite{\FlukaPaper}
    and \EposLHCLong.~\cite{\EposLHCPaper}}
  \label{fig:shower:phen:xmax}
  \begin{center}
    Source: By the author. 
  \end{center}
\end{figure}


The difference between \xmaxmean for
proton and iron initiated showers is $\sim 100 \text{g/cm}^2$, as can
be seen on the left plot of~\cref{fig:shower:phen:xmax}. It is also clear that the elongation rate
$D_{10}$ is nearly constant with energy and does not depend on the primary mass.
For using \xmax measurements to infer the primary composition of
cosmic rays, both the differences on \xmax for different primaries and
the constancy of $D_{10}$ are relevant features. By comparing the measured \xmaxmean
with simulation one can infer the average composition and by measuring
the $D_{10}$ one can verify if the composition is changing with energy or not. 


Besides the \xmaxmean, the fluctuations of \xmax also carry relevant information
about the primary cosmic rays. On the right plot of~\cref{fig:shower:phen:xmax}
we show the standard deviation of the \xmax distribution, \xmaxsig,
as a function of the primary energy for proton and iron initiated showers.
Concerning \xmaxsig, a consequence of the superposition principle
is that the fluctuations of the \xmax of showers initiated by nuclei
are smaller than the fluctuations of proton showers. It is also necessary
to take into account the smaller fluctuations on the depth of the first
interactions in nuclei showers. As a result,
we can see on the right plot of~\cref{fig:shower:phen:xmax}   
that the differences between \xmaxsig of proton and iron initiated showers is
about $\sigma^p[X_\text{max}]-\sigma^{Fe}[X_\text{max}] \approx 40 \text{g/cm}^2$.

The study of cosmic rays composition by using the \xmaxmean and \xmaxsig
is usually called \xmax moments analysis. The \xmax measurements
and their composition interpretation are approached in~\cref{sec:shower:observables:xmax}. 



%%================================%%
\subsection{Hadronic component and muon production}
\label{sec:showers:phen:had}

Unlike the neutral pions, the charged ones can only decay via weak interaction,
with a much longer decay length ($c\tau_0=7.8$ m)
than the electromagnetic $\pi^0$ decay.  
Thus, at high energies, their interaction length
is much smaller than the decay one. This means that most
of the charged pions actually interact with atmospheric nuclei
producing more particles at similar proportions as the ones produced
in the first interaction. While the neutral pions produced
fed the electromagnetic shower, as explained above, the charged ones
can keep interacting and producing more hadrons. The hadrons
produced by this chain of hadronic interactions compose the
hadronic component of the shower.

As lower the energy of the charged pions, larger the probability
that they decay instead of interacting again. The dominant decay channel
for charged pions is into muons and neutrinos
($\pi^+\rightarrow\mu^++\nu_\mu$ and $\pi^-\rightarrow\nu^-+\bar{\nu}_\mu$)
and the critical energy in which the probability of decaying is
equivalent to the probability of interacting is 85 \GeV. These decays are responsible
to produce the great majority of the muons in EAS.
The typical number of interactions of charged pions before
decaying into muons is 4-8.~\cite{Meurer:2005dt} 

Apart from pion decays, a considerably fraction of the muons reaching
the ground is produced by the decay of charged kaons.
In Refs.~\cite{Meurer:2005dt,MeurerThesis} the muon
production in EAS was studied in details and it was found
by using simulations that $\sim 90\%$ of the muons come from the decay
of charged pions and $\sim 10\%$ of charged kaons. While these particles
are called \emph{mother}, the particles which interacted hadronically
with an atmospheric nucleus to produce them are called \emph{grandmother}.
Thus, concerning the properties of the grandmother particles,
it was found that more than 70\% of them are pions, $\approx 20\%$ are nucleons
and $\approx 6\%$ are kaons. In~\cref{fig:shower:phen:ioana} the energy distributions of the
grandmother particles are shown and we can see the contribution from these
different types of particles.~\cite{\IoanaICRC}
The dominance of pions as grandmother particle is clear,
as well as the fact that most of the interactions of the grandmother particles
occur at energies of order of $\sim 100 \GeV$.

%%%%%%%%%%%%%% IOANA PLOT %%%%%%%%%%%%%%%

\begin{wrapfigure}{r}{0.5\textwidth}
  \centering
  \includegraphics[width=0.5\textwidth]{energyDistAuger}
  \caption{Energy distributions of the grandmother particles
    of muons that reaches the ground with lateral distances between 1000 and 2000 m  
    in a proton initiated shower with primary energy \E{19}.  
    Plot reproduced from Ref.~\cite{\IoanaICRC}}
  \label{fig:shower:phen:ioana}
  \begin{center}
    Source: MARI\c{S}.~\cite{\IoanaICRC} 
  \end{center}
\end{wrapfigure}

Although the interaction of the grandmother particle is very important,
it has to be noted that the whole chain of hadronic interactions, which are
mostly pion-air interactions, influence the muon production. This means
that any EAS observable related to the muonic component is sensitive to the
properties of hadronic interactions which occur along the shower. The particle
production in these hadronic interactions is particularly important and,
in the context of this thesis, it is worthwhile to point out the effect of
(anti)baryon production. Most of the (anti)baryons produced are
nucleons (\proton, \antiproton, \neutron and \antineutron) and they
are all produced in similar proportions. Since these particles
are stable and can only interact again, their energy surely fed the
hadronic component, which means that it is partially converted
into muon production. Because of that, the increase of (anti)baryon
production in hadron-air interactions has been considered a very important
mechanism to increase the total number of muons in the EAS.~\cite{\EposPaper,Drescher:2007hc}
Then, in view of the muon deficit problem, which is presented in~\cref{sec:shower:observables:nmu},
understanding the particle production in pion-air interactions turns out to be
very important. Therefore, the~\cref{sec:hadron} of this thesis is dedicated to the measurement of
the hadron production in pion-carbon interactions.

Besides increasing the number of muons, increasing the (anti)baryon
production in hadron-air interactions also affects the energy spectra
of the produced muons. It was argued in Ref.~\cite{\EposPaper} that
larger (anti)baryon production implies in shifting the muon energy spectra
to lower energies. Since accessing the energy spectra of muons at the ground
is very challenging experimentally, in~\cref{sec:observable} we propose
a new observable to be used for that considering experiments with
two different muon detectors.


On the left plot of~\cref{fig:shower:phen:prof}
we show the average longitudinal profile of
the number of muons, as well as the muon production,
for proton and iron initiated showers. First we can see that the
number of muons in iron initiated showers is substantially larger than
in proton showers. The reason for that is the fact that the first interaction
of a heavier primary produces a larger number of secondary hadrons, which
means that more hadronic sub-showers are generated and consequently
more muons are produced.
Second, we can see by comparing both plots of~\cref{fig:shower:phen:prof} 
that, after reaching its maximum, the muonic component is much weakly attenuatted
than the electromagnetic counterpart.
Thus, the number of muons after the maximum depth decreases very slowly
and after a certain atmospheric depth, only the muonic component survives to the attenuation.

%%%%%%%%%%%%%% NMU MOMENTS %%%%%%%%%%%%%%%
\begin{wrapfigure}{R}{0.5\textwidth}
  \centering
  \includegraphics[width=0.5\textwidth]{nmu_mean}
  \caption{Average \nmu as a function of the primary energy
    for a set of showers simulated with \Conex~\cite{\ConexPaper} and
    the hadronic interaction models
    \Fluka~\cite{\FlukaPaper} and \EposLHCLong.~\cite{\EposLHCPaper}
    The \nmu here is defined as all the muons which reach an observation
    level of 875 g/cm$^2$ with energy above 1 \GeV.}
  \label{fig:shower:phen:nmu}
  \begin{center}
    Source: By the author. 
  \end{center}
\end{wrapfigure}

The number of muons (\nmu) observed
at a given atmospheric depth is a characteristic quantity of the shower.
A relation between \nmu and the primary energy ($E_0$) and mass ($A$)
can also be obtained from simplified analytic models~\cite{Matthews:2005sd}
and confirmed by full Monte Carlo simulations.~\cite{AlvarezMuniz:2002ne}
This relation is 
\begin{equation}
  \nmumean \approx A^{1-\beta} E_0^\beta,
  \label{eq:shower:nmu}
\end{equation}
where $\beta\approx 0.9$.~\cite{AlvarezMuniz:2002ne}
The energy and primary mass dependence
of \nmumean can be seen in~\cref{fig:shower:phen:nmu},
where we show the \nmumean as a function of energy,
again for proton and iron initiated showers.
Results of measurements of \nmu, as well as other
observables related to the muonic component will be presented
in~\cref{sec:shower:observables}.


%%%%%%%%%%%%%%%%%%%%%%%%%%%%%%%%%%%%
\section{Simulations of EAS}
\label{sec:shower:simulations}

In this section we give a short overview of the topic of EAS simulations,
including a description of the most used simulation codes 
(\cref{sec:shower:simulations:codes}) and
an overview of the hadronic interaction models used by them (\cref{sec:shower:simulations:models}). 
Besides the hadronic interaction models, the EAS simulation codes also
make use of electromagnetic interaction models. The commonly used one is \EGS~\cite{Nelson:1985ec}
Since the electromagnetic processes are assumed to be
properly described by quantum electrodynamics (QED),
their modeling is not considered to be a significant
source of uncertainties on the EAS simulations.
Because of that, we will only focus here on the hadronic interaction models,
which, in turn, are the most relevant source of uncertainties.
More detailed approaches to the topic can be found
in Refs.~\cite{Knapp:2002vs,Engel:2011zzb,Allen:2013ofa}.


%%================================%%
\subsection{Simulation strategies}
\label{sec:shower:simulations:codes}

To simulate an EAS taking into account all the
relevant aspects is surely a very complex task.
All the particles produced have to be propagated 3-dimensionally
through the atmosphere and their interactions with atmospheric atoms
have to be properly modeled. The effect of the
structure of the atmosphere with changing density
and the earth magnetic field also have to be taken into account.
Furthermore, the extremely large number of particles at high energies makes
the computational cost of an EAS simulation extremely large.
Because of that, the full simulation turns to be
impossible for practical purposes and, thus, special strategies
have to be developed by simulation softwares to reduce the complexity
of the problem.


The most complete softwares apply Monte Carlo methods to simulate
the 3-dimensional propagation of the particles and their interactions
along the whole shower development.
This class of software includes \Mooca~\cite{\MoocaPaper},
\Aires~\cite{\AiresPaper} and \Corsika~\cite{\CorsikaPaper},
being the latter the one used in this thesis.
To reduce the computing time, an algorithm called \emph{statistical thinning}
is usually applied.~\cite{HillasThinning1,HillasThinning2}
Instead of following all the particles, by using the
thinning algorithm only a number of them are selected
to be followed and a weight is given to these particles
which corresponds to the number of particles that
are not being followed anymore.


A few parameters is usually needed
to set up the thinning configuration. As an example, in \Corsika
the maximum particle energy in which the thinning algorithm is applied is
set by the $\varepsilon_\text{th}$ parameter, which is actually the fraction 
with relation to the energy of the primary. This parameter can be set separately for
electromagnetic and hadronic particles and typical values are between $10^{-2}-10^{-7}$.
As smaller the values of $\varepsilon_\text{th}$, more precise the simulation will be,
but with the cost of longer computing time. The maximum weight allowed is also a parameter to be set
and its typical values lie between 100-10000. Since the thinning algorithm removes
the important information of the geometrical spread of the particles, for some analysis
a \emph{dethinning} algorithm has to be applied.~\cite{Stokes:2011wf}

A second class of simulation softwares, called \emph{hybrid}, makes use of Monte Carlo techniques
to simulate only the interactions at the highest energies and, below a certain
energy threshold, cascade equations are applied~\cite{Bossard:2000jh} to compute the
number of each type of particle as a function of the atmospheric depth.
This means that the solution of the cascade equation gives only a 1-dimensional
description of the shower and all the information about the lateral distributions of
the particles is lost. Since for many applications only the longitudinal profile
of the shower is needed, this class of software turns to be very useful.
The very short computing time required to solve the cascade equations
makes these softwares very fast. One of the most used code of this kind
is \Conex~\cite{\ConexPaper},
which is used in this thesis. The \Seneca code~\cite{\SenecaPaper}  
also makes use of cascade equations but includes a re-sampling
algorithm to recover the lateral particle distributions in the final stages
of the shower. It is worthwhile mentioning that the current
version of the \Corsika software can also simulate showers in the hybrid mode,
similar to \Conex or \Seneca.


%%================================%%
\subsection{Hadronic interaction models}
\label{sec:shower:simulations:models}

The models used to describe the hadronic interactions at high energies 
are by far the most relevant source of uncertainties on the prediction of
EAS observables by simulations. Although the strong interactions can be
successfully described by the \emph{quantum cromodynamics} theory (QCD),
we are still not able to compute the bulk of hadronic particle production processes
from first principles. 

The hadronic interactions are described by first considering hadrons as constituted
by point-like particles called \emph{partons}. Then, the interactions between partons
are treated differently depending on the transfer momentum: \emph{hard} and \emph{soft}
processes, in which there is a large and small momentum transfer, respectively.
While the properties of hard processes can be properly computed by means of
perturbative QCD, the soft ones require either lattice calculations
or phenomenological models to be described. Because of the long processing time
required by lattice calculations, distinct phenomenological approaches 
are applied by hadronic interaction models.

All the currently used hadronic models for high-energy interactions
are based on Gribov's Reggeon Field Theory~\cite{Gribov1968,Drescher:2000ha}, in which
hadron-hadron, hadron-nucleus and nucleus-nucleus scattering are described by the exchanging of objects
called \emph{reggeons} and \emph{pomerons}. Another examples of common features
of all models are the application of minijet and string fragmentation models.
An intrinsic feature of these phenomenological models is the large number of parameters
that cannot be determined theoretically, but they have to be tuned instead.
For that, a large number of accelerator measurements are used. The problem with this
strategy is that the accelerator measurements cover only a small fraction
of the energy range and particle production phase space that is relevant
for describing EASs. 

The problem with the model tuning can be split in three parts.
First, the highest energy collisions studied in accelerators
($\sqrt{s} = 13$ \TeV reached by LHC) is still much lower
energy than the interaction of a high energy cosmic ray ($\sqrt{s}\approx 400$ for
a \E{20} primary proton). Second, there is a lack of measurements
of particle production at the forward region, which is the most relevant
to understand particle production in soft interactions and also
the most relevant for the EAS development. Finally,
there is a lack of accelerator data concerning some colliding systems
that are very important for EAS physics,
e.g. hadron-nucleus interactions.

As a consequence of the phenomenological nature of the models
in combination with the problems with accelerator data listed above,
the predictions of EAS simulations with different hadronic
interaction models turns out to be discordant. Because of that,
it is a common approach to compare the EAS measurements to
simulations with a number of different models and to treat
the differences between them as a systematic uncertainty
of theoretical nature.

For EAS studies, the most currently used models
for high energy interactions are
\QGSJetLong~\cite{\QGSJetPaper}, \EposLong~\cite{\EposPaper},
\EposLHCLong~\cite{\EposLHCPaper}, \SibyllLong~\cite{\SibyllPaper},
\SibyllNewLong~\cite{\NewSibyllPaper} and \DPMJetLong.~\cite{\DPMJetPaper}
Among these, \QGSJetLong, \EposLHCLong and \SibyllNewLong are updated
to take into account recent LHC data.  


Below a certain energy threshold ($\sim 100\;\GeV$), the hadronic interactions
in EAS simulations are described by the so called \emph{low energy models}, which
are much simpler than the high energy ones. The particle production, for example,
is usually modeled by means of empirical parametrizations.
The most used low energy models are \Urqmd~\cite{\UrqmdPaper},
\Gheisha~\cite{Gheisha1985} and \Fluka~\cite{\FlukaPaper}.
A review of these models can be found in Ref.~\cite{Heck:2004rq}.
The influence of the low energy models on the number of muons in EAS
have been shown in Ref.~\cite{Maris:2009uc}.
We show in~\cref{sec:observable} that they are also relevant
for the spectrum of the muons at ground.


%%%%%%%%%%%%%%%%%%%%%%%%%%%%%%%%%%%%
\section{Measurements of EAS observables}
\label{sec:shower:observables}

In this section, a compilation of measurements of EAS observables
is presented. The main focus is on the high energy cosmic rays range ($E>10^{17}$ eV).
We start in~\cref{sec:shower:observables:xmax}
by the \xmax measurements which are the most reliable
observable to be used to infer composition of cosmic rays.
In~\cref{sec:shower:observables:nmu}, measurements of the
\nmu are presented and the muon deficit problem
is introduced.
In~\cref{sec:shower:observables:further}
we briefly mention further observables recently measured.
A summary is finally presented in~\cref{sec:shower:observables:summary}. 


%%================================%%
\subsection{Shower maximum and UHECR composition}
\label{sec:shower:observables:xmax}

The \xmax (see~\cref{sec:showers:phen:em}) can be measured directly
by fluorescence telescopes, which can measure the energy deposited
by the EAS along the atmosphere. This technique was first successfully applied
by Fly's Eyes experiment~\cite{\FlysEyesPaper},
posteriorly followed by HiRes~\cite{\HiResPaper} and recently it
has been used by Telescope Array~\cite{\TAXmaxPaper} and
Pierre Auger Observatory.~\cite{\AugerPaper,\AugerXmaxPRDPaper}
Another technique commonly used in the past to measure the \xmax
are the non-imaging Cherenkov detectors. From the lateral profile
of the Cherenkov light measured in surface detectors, it is possible
to infer the longitudinal position of the shower maximum.~\cite{Hillas:1982wz,Patterson:1983qj}
This method was used by Tunka~\cite{\TunkaPaper,\TunkaXmaxPaper},
Yakutsk~\cite{\YakutskPaper,\YakutskXmaxPaper} and CASA-BLANCA.~\cite{\CasaBlancaXmaxPaper}
In~\cref{fig:shower:observables:xmax:all}
we show a compilation of \xmaxmean measurements from \E{15.0}
up to the highest energies and
in~\cref{fig:shower:observables:xmax:auger,fig:shower:observables:xmax:ta}
we show the most up to date
\xmax results of Pierre Auger Observatory~\cite{\AugerXmaxICRC2017Paper}
and Telescope Array.~\cite{TAComp2017}
The \xmaxsig is also shown together with the \xmaxmean.

%%%%%%%%%%%%%% XMAX ALL %%%%%%%%%%%%%%%
\begin{figure}[!ht]
  \centering
  
  \begin{overpic}[clip, rviewport=0 0 1 1,width=0.7\textwidth]{xmax_kampert}
    \put(18,60){}
  \end{overpic}

  \caption{Compilation of \xmaxmean results reproduced from Ref.~\cite{Kampert:2012mx}.}
  \label{fig:shower:observables:xmax:all}
  \begin{center}
    Source: KAMPERT.~\cite{Kampert:2012mx} 
  \end{center}
\end{figure}

The colored curves in~\cref{fig:shower:observables:xmax:all}
show the \xmaxmean predictions from
Monte Carlo simulations. It is traditional to show the predictions for
the lightest primary expected, protons, and for the heaviest, iron nuclei.
By comparing the \xmax measurements with the Monte Carlo predictions,
one can obtain information about the average composition of the primary cosmic rays.
Besides the comparison of the \xmax moments,
the full \xmax distributions can be compared and the fractions of different
groups of primaries can be fitted. This procedure has been applied by Pierre Auger
Observatory in Ref.~\cite{Aab:2014aea}.

%%%%%%%%%%%%%% XMAX AUGER %%%%%%%%%%%%%%%
\begin{figure}[!ht]
  \centering
  
  \begin{overpic}[clip, rviewport=0 0 1 1,width=0.8\textwidth]{xmax_auger_icrc}
    \put(18,60){}
  \end{overpic}
  
  \caption{\xmaxmean and \xmaxsig as a function of the primary energy
    as measured by Pierre Auger Observatory.}
  \label{fig:shower:observables:xmax:auger}

  \begin{center}
    Source: BELLIDO.~\cite{\AugerXmaxICRC2017Paper} 
  \end{center}
\end{figure}

%%%%%%%%%%%%%% XMAX TA %%%%%%%%%%%%%%%
\begin{figure}[!ht]
  \centering
  
  \begin{overpic}[clip, rviewport=0 0 1 1,width=0.4\textwidth]{xmax_mean_ta}
    \put(18,60){}
  \end{overpic}
  \begin{overpic}[clip, rviewport=0 0 1 1,width=0.4\textwidth]{xmax_sig_ta}
    \put(18,60){}
  \end{overpic}

  \caption{\xmaxmean and \xmaxsig as a function of the primary energy
    as measured by Telescope Array experiment.}
  \label{fig:shower:observables:xmax:ta}

  \begin{center}
    Source: HANLON.~\cite{TAComp2017} 
  \end{center}
\end{figure}


The composition interpretation of the \xmax measurements
is only possible because this observable is well described
by Monte Carlo simulations using hadronic interaction models.
From~\cref{fig:shower:observables:xmax:all},
we can first see that the measured \xmaxmean lies
always in between the predictions for protons and iron nuclei.
Furthermore, it is important the fact that
the Monte Carlo predictions with different hadronic interaction models
do not diverge substantially, which reduces the model dependencies
on the composition interpretations of the measurements.

Another important feature of the \xmax analyses is
that the detector biases can be fully removed and the measured values
of the \xmaxmean and \xmaxsig can be directly compared to
the ones from shower simulations, without the requirement of simulating
the detection process. The Pierre Auger Observatory remove most of the
detector biases by using the so called \emph{fiducial volume cuts},
which reduces significantly the number of events.~\cite{\AugerXmaxPRDPaper}
On the other hand, because of the lack of statistics,
Telescope Array opt for not to perform any strategy to unbias the \xmax moments.
Instead, the measurements are compared to
simulation that include all the detector effects. As a consequence,
the \xmax moments as measured by both experiments cannot
be compared to each other directly.

From~\cref{fig:shower:observables:xmax:auger,fig:shower:observables:xmax:ta},
the composition interpretation
of both Auger and Telescope Array \xmax data seem to be incompatible.
The Auger data shows a clear changing of the average composition from a very light
composition around \E{18.3} to an intermediate one around \E{19.5}. This trending is confirmed
by the fits of \xmax distributions shown in Ref.~\cite{Aab:2014aea}.
The Telescope Array data, on the other hand, shows a constant composition
which is dominantly light, between proton and helium nuclei on average, from \E{18.2}
up to the highest energies. This apparent inconsistency has been studied
by a working group formed of members of both collaboration and the most recent
conclusion is that, taking into account the detector effects intrinsic of the Telescope Array
measurements and all the systematic uncertainties, the results of Telescope Array
are actually compatible with the composition inferred from Auger data up to \E{19}.~\cite{VitorICRC2017}
Above this energy it was not possible to take any conclusion
because of the lack of statistics of the Telescope Array measurements. 


%%================================%%
\subsection{Number of muons and the muon deficit problem}
\label{sec:shower:observables:nmu}

In this section we present measurements of both the number of muons (\nmu)
and the muon density ($\rho_\mu$) in EAS.
Measurements of \nmu or $\rho_\mu$ at ground can
be done with several different types of surface detector arrays.
However, the muon measurements are strongly affected by specific features
of each experimental setup and, therefore, it is not possible in general
to compare the measurements from different experiments. The experimental features
which affect the most the muon measurements are the muon energy threshold,
the lateral distance interval in which the muons are detected and the observation
level. 

To be able to detect isolated muons, the detectors are usually shielded so the
electromagnetic component is absorbed before reaching being detected. This can
be done either by placing a layer of a heavy material on the top of the detector
or by burying it few meters below the ground level. It turns out that low energetic muons
also end up being absorbed, and as a consequence, each experiment measures
muons above a different energy threshold and the \nmu is different.
The lateral distance range in which the muons are detected is also
a specific feature of each experiment that affects strongly the \nmu. Since
the experiments are always made of granular arrays, the distance between the detectors
usually determine in which lateral distance the \nmu is estimated. Lastly,
the observation level of the experiment affects the \nmu because of the atmospheric attenuation,
which affects much strongly the electromagnetic component, but it is still relevant for
muon detection.

Among the results of measurements of \nmu and $\rho_\mu$ in EAS experiments,
we have selected three to present here. The first one are the
measurements by HiRes/MIA experiments of the density of muons at the ground
in EAS initiated by cosmic rays of energy approximately in the range
between \E{17} and \E{18}. The experimental setup consisted of a set of fluorescence
telescopes of the HiRes experiment~\cite{\HiResPaper} combined with a detector array of
MIA.~\cite{\MIAPaper} The surface detectors were formed by scintilator counters
buried about 3 m below the ground level, which implies in an energy threshold for muons
of around 850 \MeV. The density of muons at a lateral
distance of 600 m ($\rho_\mu(600)$) was reconstructed and compared to simulations.
On the left plot of~\cref{fig:shower:observables:nmu} we reproduce
the plot of $\rho_\mu(600)$ as a function of the primary energy as published
in Ref.~\cite{\HiResMIAMuonPaper}.
By comparing the measured muon density with predictions from Monte Carlo simulations,
it was found that the $\rho_\mu(600)$ in data was larger than the predictions for
the heaviest primaries expected, iron nuclei. This result was clearly inconsistent
in terms of composition and the conclusion that could be taken is that
the Monte Carlo simulations are deficient in muon production.
This was the first manifestation of what is usually called \emph{muon deficit problem},
in 1999.

%%%%%%%%%%%%%% NMU %%%%%%%%%%%%%%%
\begin{figure}[!ht]
  \centering
  
  \begin{overpic}[clip, rviewport=0 0 1 1,width=0.3\textwidth]{hires_mia_muons}
    \put(18,60){}
  \end{overpic}
  \begin{overpic}[clip, rviewport=0 -0.12 1 1,width=0.35\textwidth]{icetop_muons}
    \put(18,60){}
  \end{overpic}
  \begin{overpic}[clip, rviewport=0 -0.06 1 1,width=0.3\textwidth]{has_nmu}
    \put(18,60){}
  \end{overpic}
  
  \caption{Muon density as measured by HiRes/MIA (left)~\cite{\HiResMIAMuonPaper}
    and IceTop (middle)~\cite{\IceTopMuonPaper} and number of muons as measured by Pierre Auger Observatory
    in highly inclined showers (right).~\cite{\AugerHASMuonPaper}}
  \label{fig:shower:observables:nmu}

  \begin{center}
    Source: ABU-ZAYYAD.(left)~\cite{\HiResMIAMuonPaper};DEMBINSKI.(middle)~\cite{\IceTopMuonPaper};AAB.(right)~\cite{\AugerHASMuonPaper} 
  \end{center}
\end{figure}

The second measurement to be presented was performed
by IceTop detector~\cite{\IceTopPaper}, part of the IceCube experiment~\cite{\IceCubePaper},
by using ice-Cherenkov surface detectors.
The primary energy range of these measurements goes from \E{15} to \E{17}
and the density of muons at 600 and 800 m from the shower axis was estimated
as muon content observable. On the middle plot of~\cref{fig:shower:observables:nmu}
we reproduce the result from the Ref.~\cite{\IceTopMuonPaper}
which shows the $\rho_\mu$ as a function of energy compared to simulation predictions
by using the \SibyllNewLong as hadronic interaction model.
The muon deficit can be again seen at energies around \E{17}. Although these are still
preliminary results, it is worthwhile to present it here because they show that
the muon deficit problem is still existing at relatively low energies (<\E{18}),
even by using modern hadronic interaction models like \SibyllNewLong.


The third and final measurement presented here is
the one performed by Pierre Auger Observatory.~\cite{\AugerPaper}
Even that its standard surface array
does not contain muon detectors, the number of muons can be measured by Auger
in highly inclined showers ($\theta > 60^\circ$). For this type of showers
the electromagnetic component is almost totally attenuated by the atmosphere
and the signal at the water-Cherenkov tanks is dominated by muons.
On the right plot of~\cref{fig:shower:observables:nmu} we show the
average \nmu as measured by Auger compared
to simulations as published in Ref.~\cite{\AugerHASMuonPaper}.
The primary energy range here is above \E{18.6} and the muon deficit is clear
at the whole energy range. It is important to point out that both models
which are shown for comparison are recent versions, updated with LHC results.
In Ref.~\cite{\AugerTopDownPaper} another Pierre Auger result
that confirms the muon deficit at energies
around \E{19} can be found, in which showers with $\theta < 60^\circ$ 
were used.

In conclusion, the muon deficit problem is known for almost two decades
and at the moment it prevents us to do composition studies based on the
measurements of the \nmu in EAS. In~\cref{sec:interpretation}
we present a method to interpret the energy evolution of the moments
of the \nmu measurements in terms of cosmic rays
composition, taking into account the muon deficit problem.


%%================================%%
\subsection{Further measurements}
\label{sec:shower:observables:further}

In this section we briefly mention four further
measurements of EAS observables, the first one
made by KASCADE-Grande experiment~\cite{\KASCADEGrandePaper}
and the remaining three by the Pierre Auger Observatory.

The KASCADE-Grande experiment can measure both
number of charged particles and number of muons
in EAS with primary energy between
\E{16.0} and about \E{17.5}. In a recent analysis, the
KASCADE-Grande collaboration has studied the evolution
of the \nmu with the zenith angle of the shower by using
a parameter called muon attenuation length ($\Lambda_\mu$). 
Since the atmospheric attenuation of the muonic component
depends on the energy of muons, the parameter $\Lambda_\mu$ turns to
be sensitive to the muon energy spectra. Thus, by comparing the measurements
to Monte Carlo predictions it is possible to test muon production
properties of the hadronic interaction models. On the left plot of~\cref{fig:shower:observables:further1}
we reproduce the plot from Ref.~\cite{Apel:2017thr} that compares the measured value of 
$\Lambda_\mu$ with simulations and what can be seen is that none of
the currently used models can describe properly the muon attenuation.
The primary energy range of this analysis is similar to the one of IceTop shown
in the previous section.


%%%%%%%%%%%%%% FURTHER 1 %%%%%%%%%%%%%%%
\begin{figure}[!ht]
  \centering
  
  \begin{overpic}[clip, rviewport=0 0 1 1,width=0.48\textwidth]{kascade_attenuation}
    \put(18,60){}
  \end{overpic}
  \begin{overpic}[clip, rviewport=0 -0.1 1 1,width=0.48\textwidth]{mpd_auger}
    \put(18,60){}
  \end{overpic}
  
  \caption{Muon attenuation length as measured by KASCADE-Grande experiments (left)~\cite{Apel:2017thr}
    and the maximum muon production depth as measured by Pierre Auger Observatory
    (right).~\cite{\AugerMPDPaper}}
  \label{fig:shower:observables:further1}
  \begin{center}
    Source: APEL.(left).~\cite{Apel:2017thr}; AAB.(right)~\cite{\AugerMPDPaper} 
  \end{center}

\end{figure}

The second measurement to be presented is the longitudinal depth
of the maximum of the muon production profile (\xmumax),
which can be measured by using the temporal structure of the
water-Cherenkov detectors of Pierre Auger Observatory
in highly inclined showers.~\cite{\AugerMPDPaper}
The main results is the energy evolution of \xmumaxmean that is show
on the right plot of~\cref{fig:shower:observables:further1}.
Although the composition interpretation of these measurements
depends strongly on the hadronic interaction model, it can be seen that
for both \QGSJetLong and \EposLHCLong the inferred composition would be
much heavier than the expected from the \xmaxmean measurements. The \EposLHCLong
case is even worst because it implies in a composition even heavier than iron nuclei.
In Refs.~\cite{Ostapchenko:2016bir,Pierog:2015ifw}
it was shown that \xmumax is very sensitive to the properties of
the pion-air interactions and therefore its measurements
can be effectively used to constrain hadronic models.

The third and forth observables to be mentioned are related to the quantity called
risetime that is defined as the time difference between the water-Cherenkov detectors
reach 10 and 50\% of their total signal. This quantity is sensitive to both the
electromagnetic and muonic component of the EAS.
In Ref.~\cite{\AugerAsymmetryPaper} an analysis of the azimutal asymmetry of the risetime
was published where a parameter called \emph{risetime asymmetry},
$\sec(\theta)_\text{max}$, was defined.
In a second analysis, published in Ref.~\cite{\AugerDeltaPaper}, the so called
\emph{delta method} was applied in which the risetime dependence
on the lateral distance is used to define
a parameter called $\Delta_\text{s}$. 
In~\cref{fig:shower:observables:further2}
we show the main results of both analysis. 
The conclusions from both analysis are similar, being that the composition
extracted directly from $\sec(\theta)_\text{max}$ and $\Delta_\text{s}$
are not compatible with the one inferred from \xmax measurements.
However, the inconsistency is not
as strong as in the case of \nmu and \xmumax. This intermediate behavior can
be understood by the fact that the risetime parameter is not sensitive to
only the muonic content of EASs, but instead, to a combination
of the electromagnetic and muonic components.
It was shown in Ref.~\cite{\AugerDeltaPaper} that the $\Delta_\text{s}$
parameter can be calibrated by using the \xmax measurements and as a result
consistent composition inferences can be done.



%%%%%%%%%%%%%% FURTHER 2 %%%%%%%%%%%%%%%
\begin{figure}[!ht]
  \centering
  
  \begin{overpic}[clip, rviewport=0 -0.03 1 1,width=0.44\textwidth]{asymmetry_auger}
    \put(18,60){}
  \end{overpic}
  \begin{overpic}[clip, rviewport=0 0 1 1,width=0.48\textwidth]{delta_auger}
    \put(18,60){}
  \end{overpic}

  
  \caption{Risetime asymmetry (left)~\cite{\AugerAsymmetryPaper}
    and the delta parameter (right)~\cite{\AugerDeltaPaper}
    as measured by Pierre Auger Observatory.}
  \label{fig:shower:observables:further2}

  \begin{center}
    Source: AAB.~\cite{\AugerAsymmetryPaper,\AugerDeltaPaper} 
  \end{center}
\end{figure}


%%================================%%
\subsection{Summary}
\label{sec:shower:observables:summary}

We can divide the EAS measurements presented here
in three categories: pure electromagnetic, pure muonic and mixed.
The only pure electromagnetic measurement discussed is
the \xmax one, which can be used to study cosmic rays
composition because, among other reasons,
the predictions from EAS simulations are consistent
with the data.

The pure muonic measurements include number (or density)
of muons (\nmu), maximum of the muon production depth (\xmumax)
and the muon attenuation length ($\Lambda_\mu$). The conclusion concerning
these measurements is that the Monte Carlo simulations fail
on describing properly the muonic component. As a consequence,
observables like \nmu and \xmumax cannot be used to infer cosmic rays composition.
One particular manifestation of this inconsistency between data and simulations
is the so called muon deficit problem, which is observed from \E{17}
up to the highest energies.

The mixed measurements presented are the ones related to the risetime
parameter, the risetime asymmetry ($\sec(\theta)_\text{max}$) and the delta
parameter ($\Delta_S$). In these cases, the composition interpretation
of the measurements does not imply in non-physical conclusions
(average nuclear mass heavier than iron nuclei), however,
it is not, in general, compatible with the composition inferred
from the \xmax measurements. This behavior is expected also because
of the failure of the simulation to predict the muonic component properly.

An usual method to compare the composition interpretation from
several measurements is to convert their average value to the average
of the logarithm of the nuclear mass, $\langle\ln A\rangle$. 
This conversion can be performed for observables that correlates
linearly with $\ln A$, through the relation
\begin{equation}
  \langle\ln A\rangle = \frac{\langle p_\text{data}\rangle - \langle p_\text{MC}^\text{proton}\rangle}{\langle p_\text{MC}^\text{iron}\rangle - \langle p_\text{MC}^\text{proton}\rangle}, 
  \label{eq:shower:lna}
\end{equation}
where $p$ is an arbitrary observable and $p_\text{MC}$ is obtained from Monte Carlo simulations.
In~\cref{fig:shower:observables:lna} we show the $\langle \ln A\rangle$
as a function of primary energy
for all the observables measured by Pierre Auger Observatory presented in this section. 
The upper plot refers to the hadronic model \EposLHCLong, while the lower one to \QGSJetLong.


%%%%%%%%%%%%%% LNA ALL %%%%%%%%%%%%%%%
\begin{figure}[!ht]
  \centering
  
  \begin{overpic}[clip, rviewport=0 0 1 1,width=0.48\textwidth]{lna_epos}
    \put(18,60){}
  \end{overpic}
  \begin{overpic}[clip, rviewport=0 0 1 1,width=0.48\textwidth]{lna_qgsjet}
    \put(18,60){}
  \end{overpic}
  
  \caption{$\langle\ln A\rangle$ as a function of energy for a set of
    observables measured by Pierre Auger Observatory. Only statistical uncertainties
    are shown. The hadronic interaction
    models \EposLHCLong (left) and \QGSJetLong (right) to perform the convertion
    represented by~\cref{eq:shower:lna}.}
  \label{fig:shower:observables:lna}

  \begin{center}
    Source: By the author. 
  \end{center}
\end{figure}


