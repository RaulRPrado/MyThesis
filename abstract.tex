%% Abstract.tex
% ---
% Abstract
% ---
\autor{Prado, R. R.}
\begin{resumo}[Abstract]        
 \begin{otherlanguage*}{english}
	\begin{flushleft} 
		\setlength{\absparsep}{0pt} % ajusta o espaçamento dos parágrafos do resumo		
 		\SingleSpacing 
 		\imprimirautorabr~ ~\textbf{\imprimirtitulo}. \imprimirdata.  \pageref{LastPage}p. 
		%Substitua p. por f. quando utilizar oneside em \documentclass
		%\pageref{LastPage}f.
		\imprimirtipotrabalho~-~\imprimirinstituicao, \imprimirlocal, 	\imprimirdata. 
 	\end{flushleft}
	\OnehalfSpacing 

Ultra-High Energy Cosmic Rays (UHECR) can only be measured by the detection of
Extensive Air Showers (EAS) created by the interaction of the cosmic ray particle
with an atmospheric nuclei. The inference of some of the properties of UHECR,
like their mass composition, is only possible by the comparison of measurements
of EAS observables to predictions from Monte Carlo simulations.
The most important source of uncertainties on the description of EAS by the simulations
is the modeling of hadronic interactions.
For many years it has been known that the hadronic interaction models fail on predicting the EAS
observables related to their muonic component. The most evident manifestation of that is
called \emph{muon deficit problem} due to the fact that the number of muons in EAS
with energies above \E{18} predicted by simulations is smaller than the observed ones.
The aim of this thesis is to approach this problem in three distinct fronts.
First, a method is developed to interpret measurements of number of muons in terms
of cosmic rays composition in despite of the muon deficit problem.
Second, an EAS observable which is sensitive to the muon energy spectrum at ground
and, consequently, can be used to constrain hadronic interaction models is proposed
and tested. Third and final, the muon production in air showers is studied through
measurements of hadron production spectra in pion-carbon interactions.  

   \vspace{\onelineskip}
 
   \noindent 
   \textbf{Keywords}:Ultra-high energy cosmic rays. Extensive air showers. Muonic component of air showers. Air shower physics. 
 \end{otherlanguage*}
\end{resumo}
