%\documentclass[12pt,openright,twoside,a4paper,chapter=TITLE,english]{USPSC}
\documentclass[12pt,openright,twoside,a4paper,english,sumario=tradicional]{USPSC}

% ---
% Pacotes básicos - Fundamentais 
% ---
\usepackage[T1]{fontenc}		% Seleção de códigos de fonte.
\usepackage[utf8]{inputenc}		% Codificação do documento (conversão automática dos acentos)
\usepackage{lmodern}			% Usa a fonte Latin Modern
\usepackage{lastpage}			% Usado pela Ficha catalográfica
\usepackage{indentfirst}		% Indenta o primeiro parágrafo de cada seção.
\usepackage{color}				% Controle das cores
\usepackage{graphicx}			% Inclusão de gráficos
\usepackage{float} 				% Fixa tabelas e figuras no local exato
\usepackage{microtype} 			% para melhorias de justificação
\usepackage{pdfpages}
%\usepackage{makeidx}            % para gerar índice remissimo

\usepackage[num,overcite,abnt-emphasize=bf, abnt-thesis-year=both, abnt-repeated-author-omit=yes, abnt-last-names=abnt, abnt-etal-cite,abnt-etal-list=0, abnt-etal-text=default, abnt-and-type=e, iso-abbreviation=standard]{abntex2cite}

\renewcommand{\thefootnote}{\fnsymbol{footnote}}  %Comando para inserção de símbolos em nota de rodapé
\renewcommand{\footnotesize}{\small} %Comando para diminuir a fonte das notas de rodapé

\usepackage{cite}	

% pacotes de tabelas
\usepackage{multicol}	% Suporte a mesclagens em colunas
\usepackage{multirow}	% Suporte a mesclagens em linhas
\usepackage{longtable}	% Tabelas com várias páginas
\usepackage{threeparttablex}    % notas no longtable
\usepackage{array}

% ---
% DADOS INICIAIS - Define sigla com título, área de concentração e opção do Programa 
% Consulte a tabela referente aos Programas, áreas e opções de sua unidade contante do
% arquivo USPSC-Siglas estabelecidas para os Programas de Pós-Graduação por Unidade.xlsx 
% ou nos APÊNDICES A-G
\siglaunidade{IFSC} 
\programa{DFA}
\definecolor{blue}{RGB}{41,5,195}

% informações do PDF
\makeatletter
\hypersetup{
	%pagebackref=true,
	pdftitle={\@title}, 
	pdfauthor={\@author},
	pdfsubject={\imprimirpreambulo},
	pdfcreator={LaTeX with abnTeX2},
	pdfkeywords={abnt}{latex}{abntex}{USPSC}{trabalho acadêmico}, 
	colorlinks=true,       		% false: boxed links; true: colored links
	linkcolor=blue,          	% color of internal links
	citecolor=blue,        		% color of links to bibliography
	filecolor=magenta,      		% color of file links
	urlcolor=blue,
	bookmarksdepth=4
}
\makeatother
% --- 
% --- 
% Espaçamentos entre linhas e parágrafos 
% --- 

% O tamanho do parágrafo é dado por:
\setlength{\parindent}{1.3cm}

% Controle do espaçamento entre um parágrafo e outro:
\setlength{\parskip}{0.2cm}  % tente também \onelineskip


%%%%%%%%%%%%% RAUL %%%%%%%%%%%%%
\usepackage{xspace}
\usepackage{enumitem}
\usepackage{amsmath}
\usepackage{rviewport}
\usepackage{comment}
\usepackage{cleveref}
\crefname{appendix}{App.}{Apps.}
\crefname{section}{Sec.}{Secs.}
\crefname{paragraph}{Sec.}{Secs.}
\crefname{table}{Tab.}{Tabs.}
\crefname{figure}{Fig.}{Figs.}
\crefname{equation}{Eq.}{Eqs.}
\crefname{item}{item}{items}

%%%%%%% units
\newcommand{\kg}{\ensuremath{\mbox{kg}}\xspace}
\newcommand{\eV}{\ensuremath{\mbox{e\kern-0.1em V}}\xspace}
\newcommand{\EeV}{\ensuremath{\mbox{Ee\kern-0.1em V}}\xspace}
\newcommand{\TeV}{\ensuremath{\mbox{Te\kern-0.1em V}}\xspace}
\newcommand{\PeV}{\ensuremath{\mbox{Pe\kern-0.1em V}}\xspace}
\newcommand{\GeV}{\ensuremath{\mbox{Ge\kern-0.1em V}}\xspace}
\newcommand{\MeV}{\ensuremath{\mbox{Me\kern-0.1em V}}\xspace}
\newcommand{\GeVc}{\ensuremath{\mbox{Ge\kern-0.1em V}\kern-0.1em/\kern-0.05em c}\xspace}
\newcommand{\GeVcc}{\ensuremath{\mbox{Ge\kern-0.1em V}\kern-0.1em/\kern-0.05em c^2}\xspace}
\newcommand{\AGeV}{\ensuremath{A\,\mbox{Ge\kern-0.1em V}}\xspace}
\newcommand{\AGeVc}{\ensuremath{A\,\mbox{Ge\kern-0.1em V}\kern-0.1em/\kern-0.05em c}\xspace}
\newcommand{\MeVc}{\ensuremath{\mbox{Me\kern-0.1em V}\kern-0.1em/\kern-0.05em c}\xspace}
\newcommand{\MeVcc}{\ensuremath{\mbox{Me\kern-0.1em V}\kern-0.1em/\kern-0.05em c^2}\xspace}
\newcommand{\T}{\ensuremath{\mbox{T}}\xspace}
\newcommand{\cmsq}{\ensuremath{\mbox{cm}^2}\xspace}
\newcommand{\msq}{\ensuremath{\mbox{m}^2}\xspace}
\newcommand{\cm}{\ensuremath{\mbox{cm}}\xspace}
\newcommand{\mm}{\ensuremath{\mbox{mm}}\xspace}
\newcommand{\mb}{\ensuremath{\mbox{mb}}\xspace}
\newcommand{\micron}{\ensuremath{\mu\mbox{m}}\xspace}
\newcommand{\mrad}{\ensuremath{\mbox{mrad}}\xspace}
\newcommand{\ns}{\ensuremath{\mbox{ns}}\xspace}
\newcommand{\m}{\ensuremath{\mbox{m}}\xspace}
\newcommand{\s}{\ensuremath{\mbox{s}}\xspace}
\newcommand{\ms}{\ensuremath{\mbox{ms}}\xspace}
\newcommand{\ps}{\ensuremath{\mbox{ps}}\xspace}
\newcommand{\pT}{\ensuremath{p_\text{T}}\xspace}
\newcommand{\pL}{\ensuremath{p_\text{L}}\xspace}
\newcommand{\xF}{\ensuremath{x_\text{F}}\xspace}
\newcommand{\xpF}{\ensuremath{x'_\text{F}}\xspace}
\newcommand{\minv}{\ensuremath{m_\text{inv}}\xspace}

%%%%%%%%%%%%% some software programs and generators
%----- NA61 software
\def\Offline{\mbox{$\overline{\text%
{Off}}$\hspace{.05em}\raisebox{.4ex}{\underline{line}}}\xspace}
\def\SHOE{\mbox{SHO\hspace{-1.34ex}\raisebox{0.2ex}{\color{green}\textasteriskcentered}\hspace{0.25ex}E}\xspace}
\def\DSHACK{\mbox{DS\hspace{0.15ex}$\hbar$ACK}\xspace}
\def\SHINE{\mbox{\textsc{S\hspace{.05em}\raisebox{.4ex}{\underline{hine}}}}\xspace}
%----- event generators
\def\Glissando{\textsc{Glissando}\xspace}
\newcommand{\FlukaLong}{{\scshape Fluka\,2008}\xspace}
\newcommand{\FlukaEleven}{{\scshape Fluka\,2011}\xspace}
\newcommand{\Fluka}{{\scshape Fluka}\xspace}
\newcommand{\UrqmdLong}{{\scshape U}r{\scshape qmd\,1.3.1}\xspace}
\newcommand{\Urqmd}{{\scshape U}r{\scshape qmd}\xspace}
\newcommand{\GheishaLong}{{\scshape Gheisha\,2002}\xspace}
\newcommand{\GheishaOld}{{\scshape Gheisha\,600}\xspace}
\newcommand{\Gheisha}{{\scshape Gheisha}\xspace}
\newcommand{\Corsika}{{\scshape Corsika}\xspace}
\newcommand{\Venus}{{\scshape Venus}\xspace}
\newcommand{\VenusLong}{{\scshape Venus\,4.12}\xspace}
\newcommand{\GiBUU}{{\scshape GiBUU}\xspace}
\newcommand{\GiBUULong}{{\scshape GiBUU\,1.6}\xspace}
\newcommand{\FlukaNewLong}{{\scshape Fluka\,2011.2\_17}\xspace}
\newcommand{\Root}{{\scshape Root}\xspace}
\newcommand{\Geant}{{\scshape Geant}\xspace}
\newcommand{\GeantThree}{{\scshape Geant3}\xspace}
\newcommand{\GeantFour}{{\scshape Geant4}\xspace}
\newcommand{\QGSJet}{{\scshape QGSJet}\xspace}
\newcommand{\DPMJet}{{\scshape DPMJet}\xspace}
\newcommand{\Epos}{{\scshape Epos}\xspace}
\newcommand{\EposLong}{{\scshape Epos\,1.99}\xspace}
\newcommand{\QGSJetLong}{{\scshape QGSJet\,II-04}\xspace}
\newcommand{\QGSJetOldLong}{{\scshape QGSJet\,II-03}\xspace}
\newcommand{\DPMJetLong}{{\scshape DPMJet\,3.06}\xspace}
\newcommand{\SibyllLong}{{\scshape Sibyll\,2.1}\xspace}
\newcommand{\SibyllNewLong}{{\scshape Sibyll\,2.3}\xspace}
\newcommand{\EposLHCLong}{{\scshape Epos\,LHC}\xspace}
\newcommand{\Hsd}{{\scshape Hsd}\xspace}
\newcommand{\Ampt}{{\scshape Ampt}\xspace}

%%%%%%%%%%%%%%%%%%%%%%%% misc
\def\red#1{{\color{red}#1}}
\def\avg#1{\langle{#1}\rangle}
\def\sci#1#2{#1{\times}10^{#2}}
\newcommand{\Fi}[1]{Fig.~\ref{#1}}
\newcommand{\NASixtyOne}{NA61\slash SHINE\xspace}%this seems to work properly to me. aa
\newcommand{\NAFortyNine}{NA49\xspace}%this seems to work properly to me. aa
\newcommand{\CernVM}{\textsc{Cern\-\kern-0.05emVM}\xspace}

%%%%%%%%%%%%%%%%%%%%%%%% RAUL

\newcommand{\E}[1]{$10^{#1}$ \eV\xspace}


\def \nmu{$N_{\mu}$\xspace}
\def \xmax{$X_\text{max}$\xspace}
\def \xmumax{$X^\mu_\text{max}$\xspace}

\def \pions{$\pi^\pm$\xspace}
\def \kaons{K$^\pm$\xspace}
\def \proton{p\xspace}
\def \antiproton{$\bar{\text{p}}$\xspace}
\def \protons{p($\bar{\text{p}}$)\xspace}
\def \protonpm{p$^\pm$\xspace}
\def \lamb{$\Lambda$\xspace}
\def \antilamb{$\bar{\Lambda}$\xspace}
\def \lambs{$\Lambda(\bar{\Lambda})$\xspace}
\def \kzeros{K$_{S}^{0}$\xspace}
\def \pipi{$\pi^+\pi^-$\xspace}
\def \vzero{$V^0$\xspace}
\def \vzeros{$V^0$'s\xspace}


\newcommand{\eps}{\ensuremath{\mbox{$\varepsilon$}}\xspace}
\newcommand{\meaneps}{\ensuremath{\mbox{$\langle\varepsilon\rangle$}}\xspace}
\newcommand{\meanepsbb}{\ensuremath{\mbox{$\langle\varepsilon\rangle^\text{BB}$}}\xspace}
\newcommand{\dedx}{\ensuremath{\mbox{\text{d}$E$/\text{d}$x$}}\xspace}
\newcommand{\meandedx}{\ensuremath{\mbox{$\langle$\text{d}$E$/\text{d}$x\rangle$}}\xspace}
\newcommand{\ncl}{\ensuremath{\mbox{$n_{\text{cl}}$}}\xspace}
\newcommand{\chisq}{\ensuremath{\mbox{$\chi^2$}}\xspace}
\newcommand{\redchisq}{\ensuremath{\mbox{$\chi^2/$ndf}}\xspace}

\newcommand{\ipart}{\ensuremath{i}\xspace}
\newcommand{\npart}{\ensuremath{I}\xspace}
\newcommand{\ich}{\ensuremath{j}\xspace}
\newcommand{\ieps}{\ensuremath{k}\xspace}
\newcommand{\neps}{\ensuremath{K}\xspace}
\newcommand{\pp}{\ensuremath{p}\xspace}
\newcommand{\iz}{\ensuremath{m}\xspace}
\newcommand{\nz}{\ensuremath{M}\xspace}
\newcommand{\iq}{\ensuremath{l}\xspace}
\newcommand{\nq}{\ensuremath{L}\xspace}
\newcommand{\gen}{\text{gen}\xspace}
\newcommand{\sel}{\text{sel}\xspace}
\newcommand{\cmc}{\ensuremath{\mbox{$C_\text{MC}$}}\xspace}

\newcommand{\imass}{\ensuremath{i}\xspace}
\newcommand{\nmass}{\ensuremath{I}\xspace}

\newcommand{\impact}{\ensuremath{\mbox{$b_r$}}\xspace}
\newcommand{\impactmax}{\ensuremath{\mbox{$b_r^\text{cut}$}}\xspace}
\newcommand{\decaydist}{\ensuremath{\mbox{$L_\text{decay}$}}\xspace}
\newcommand{\decaydistmin}{\ensuremath{\mbox{$L_\text{decay}^\text{cut}$}}\xspace}
\newcommand{\decaydistopt}{\ensuremath{\mbox{$L_\text{decay}^\text{cut, opt}$}}\xspace}





  
  


%aliases for bib entries

%%EXPERIMENTS
\def\NASixtyOnePaper{Abgrall:2014xwa} 
\def\NAFortyNinePaper{Afanasev:1999iu}
\def\T2KPaper{Abe:2011ks}
\def\AlicePaper{Aamodt:2008zz}
\def\CMSPaper{Chatrchyan:2008aa}
\def\AugerPaper{ThePierreAuger:2015rma}
\def\KASCADEGrandePaper{Apel:2010zz}
\def\KASCADEPaper{Antoni:2003gd}

%%MODELS
\def\QGSJetPaper{Ostapchenko:2010vb}
\def\DPMJetPaper{Roesler:2000he}
\def\EposPaper{Pierog:2006qv}
\def\EposLHCPaper{Pierog:2013ria}
\def\SibyllPaper{Ahn:2009wx}
\def\NewSibyllPaper{Riehn:2015oba}
\def\UrqmdPaper{Bleicher:1999xi}
\def\FlukaPaper{Ferrari:2005zk}

%%SOFTWARES
\def\MoocaPaper{Hillas:1997tf}
\def\AiresPaper{Sciutto:1999jh}
\def\CorsikaPaper{Heck:1998vt}
\def\ConexPaper{Bergmann:2006yz}
\def\SenecaPaper{Drescher:2002cr}


%%CR RESULTS
\def\AugerHASMuonPaper{Aab:2014pza}
\def\AugerMPDPaper{Aab:2014dua}
\def\AugerTopDownPaper{Aab:2016hkv}
\def\AugerXmaxPRDPaper{Aab:2014kda}
\def\AugerXmaxICRC2017Paper{Bellido:2017cgf}
\def\AugerAsymmetryPaper{Aab:2016enk}
\def\AugerDeltaPaper{Aab:2017cgk}
\def\TAXmaxPaper{Jui:2011vm}
\def\HiResPaper{AbuZayyad:2000uu}
\def\FlysEyesPaper{Baltrusaitis:1985mx}
\def\MIAPaper{Borione:1994iy}
\def\HiResMIAMuonPaper{AbuZayyad:1999xa}
\def\IceTopMuonPaper{Dembinski:2017zkb}
\def\IceCubePaper{Aartsen:2016nxy}
\def\IceTopPaper{IceCube:2012nn}
\def\TunkaPaper{Antokhonov:2011zz}
\def\TunkaXmaxPaper{Budnev:2009mc}
\def\YakutskXmaxPaper{Knurenko:2011zz}
\def\YakutskPaper{Knurenko:2010eu}
\def\CasaBlancaXmaxPaper{Fowler:2000si}


%%NA61
\def\RhoPaper{Aduszkiewicz:2017anm}


%%OTHERS
\def\APP16{Prado:2016akv}
\def\APP17{Prado:2017ijs}
\def\IoanaICRC{IoanaICRC2009}
\def\MichaelICRC{UngerICRC2011}
\def\HansICRC{HansICRC2013}
\def\AlexICRC{Herve:2015lra}
\def\RaulICRC{Prado:2017hub}
\def\PDGPaper{Patrignani:2016xqp}  





\newcommand{\warning}[1]{\textcolor{red}{\textbf{(#1)}}}
\newcommand{\note}[1]{\textcolor{blue}{\textbf{(#1)}}}

\graphicspath{{./figures/}}
%%%%%%%%%%%%%%%%%%%%%%%%%%%%%%%%




% ---
% compila o sumário e índice
\makeindex
% ---

% ----
% Início do documento
% ----
\begin{document}

% Seleciona o idioma do documento (conforme pacotes do babel)
\selectlanguage{english}

% Retira espaço extra obsoleto entre as frases.
\frenchspacing 

% --- Formatação dos Títulos
\renewcommand{\ABNTEXchapterfontsize}{\fontsize{12}{12}\bfseries}
\renewcommand{\ABNTEXsectionfontsize}{\fontsize{12}{12}\bfseries}
\renewcommand{\ABNTEXsubsectionfontsize}{\fontsize{12}{12}\bfseries}
\renewcommand{\ABNTEXsubsubsectionfontsize}{\fontsize{12}{12}\bfseries}
\renewcommand{\ABNTEXsubsubsubsectionfontsize}{\fontsize{12}{12}\bfseries}%\normalfont


% ----------------------------------------------------------
% ELEMENTOS PRÉ-TEXTUAIS
% ----------------------------------------------------------
% ---
% Capa
% ---
\imprimircapa
% ---
% Folha de rosto
% (o * indica impressão em anverso (frente) e verso )
% ---
\imprimirfolhaderosto*
%\imprimirfolhaderosto
% ---
% ---
% Inserir a ficha catalográfica em pdf
% ---
% A biblioteca da sua Unidade lhe fornecerá um PDF com a ficha
% catalográfica definitiva. 
% Quando estiver com o documento, salve-o como PDF no diretório
% do seu projeto como fichacatalografica.pdf e iclua o arquivo
% utilizando o comando abaixo:
%\includepdf{USPSC-Tese-pre-textual/fichacatalografica.pdf} 
% Se você optar por elaborar a ficha catalográfica, deverá 
% incluir uma % antes da linha % antes
% do comando \include{USPSC-Tese-pre-textual/USPSC-fichacatalografica} 
% e retirar o % do comando abaixo
%\include{USPSC-Tese-pre-textual/USPSC-fichacatalografica}
% As informações que compõem a ficha catalográfica estão 
% definidos no arquivo USPSC-pre-textual-UUUU.tex
% ---


% ---
% Inserir folha de aprovação
% ---
\includepdf{folhadeaprovacao.pdf}

\includepdf{PaginaEmBranco.pdf}

% ---
% Dedicatória
% ---
%\include{USPSC-Dedicatoria}
% ---

% ---
% Agradecimentos
% ---
%% Agradecimentos.tex
% ---
% Agradecimentos
% ---=====
\begin{agradecimentos}
  This thesis is dedicated to my family and specially to the memory of my grandfather Ernesto.

  I am very grateful to:
  \begin{itemize}
  \item My advisor Prof. Vitor de Souza for all the guidance along over 9 years of my science journey.
  \item My colleagues and friends from the Astroparticle Physics group at IFSC for providing a great and fun working environment along all these years. I would like to thank specially Rodrigo L. and Luan A. for reviewing this thesis.  
  \item The LIP group (specially Prof. Mário Pimenta and Ruben Conceição) and the KIT group (specially Michael Unger, Alex Herve and Darko Veberic) for the great reception and collaboration. 
  \item All my friends, specially Jessica D., Natália K.B., Thiago M.A., Milena C., Rodrigo G.L., Guilherme T., Edmilson R.S., Diogo L.B., Pedro I.B., André M., Gilson C., David S., Darko V..   
  \item FAPESP, for the financial support (grant No. 2014/10460-1 and 2016/12735-3).
  \end{itemize}

\end{agradecimentos}
% ---

% ---

% ---
% Epígrafe
% ---
\begin{epigrafe}
    \vspace*{\fill}
	\begin{flushright}
		\textit{``O estudo, a busca da verdade e da beleza são domínios \\
		em que nos é consentido sermos crianças por toda a vida.''\\
		Albert Einstein}
	\end{flushright}
\end{epigrafe}
% ---

% A T E N Ç Ã O
% Se o idioma do texto for em inglês, o abstract deve preceder o resumo
% resumo em português
%
% Resumo
% ---
%% Abstract.tex
% ---
% Abstract
% ---
\autor{Prado, R. R.}
\begin{resumo}[Abstract]        
 \begin{otherlanguage*}{english}
	\begin{flushleft} 
		\setlength{\absparsep}{0pt} % ajusta o espaçamento dos parágrafos do resumo		
 		\SingleSpacing 
 		\imprimirautorabr~ ~\textbf{\imprimirtitleabstract}.	\imprimirdata.  \pageref{LastPage}p. 
		%Substitua p. por f. quando utilizar oneside em \documentclass
		%\pageref{LastPage}f.
		\imprimirtipotrabalho~-~\imprimirinstituicao, \imprimirlocal, 	\imprimirdata. 
 	\end{flushleft}
	\OnehalfSpacing 
   This is the english abstract.

   \vspace{\onelineskip}
 
   \noindent 
   \textbf{Keywords}: .
 \end{otherlanguage*}
\end{resumo}

% ---

% Abstract
% ---
%% Resumo.tex
% ---
% Resumo
% ---
\setlength{\absparsep}{18pt} % ajusta o espaçamento dos parágrafos do resumo		
\begin{resumo}[Resumo]
	\begin{flushleft} 
		        \setlength{\absparsep}{0pt} % ajusta o espaçamento da referência	
			\SingleSpacing 
			\imprimirautorabr~ ~\textbf{\imprimirtitleabstract}. \imprimirdata. \pageref{LastPage}p. 
			%Substitua p. por f. quando utilizar oneside em \documentclass
			%\pageref{LastPage}f.
			\imprimirtipotrabalhopt~-~\imprimirinstituicao, \imprimirlocal, \imprimirdata. 
 	\end{flushleft}
        \OnehalfSpacing 			

Raios C\'osmicos Ultra Energ\'eticos (Ultra-High Energy Cosmic Rays, UHECR) somente
podem ser medidos atrav\'es da detec\c{c}\~ao dos Chuveiros Atmosf\'ericos Extensos
(Extensive Air Showers, EAS) criados pela intera\c{c}\~ao do raio c\'osmico prim\'ario com
n\'ucleos atmof\'ericos. A infer\^encia de algumas propriedados dos UHECRs, como a composi\c{c}\~ao
de massa, \'e poss\'ivel somente atrav\'es da compara\c{c}\~ao entre medidas de observ\'aveis dos EASs
com predi\c{c}\~oes geradas por simula\c{c}\~oes de Monte Carlo. A fonte de incerteza mais importante
na descri\c{c}\~ao de EAS por simula\c{c}\~oes \'e a modelagem das intera\c{c}\~oes hadr\^onicas. Por muitos
anos \'e sabido que os modelos de intera\c{c}\~ao hadr\^onica falham na predi\c{c}\~ao de observ\'aveis
dos EASs relacionados a sua componente mu\^onica. A manifesta\c{c}\~ao mais evidente disso \'e 
chamada \emph{problema do d\'eficit de m\'uons} devido ao fato que o n\'umero de m\'uons em chuveiros
com energies acima de \E{18} predito por simula\c{c}\~oes \'e menor que os observados.
O objetivo desta tese \'e abordar este problema atrav\'es de tr\^es frentes. Primeiramente,
um m\'etodo \'e desenvolvido para interpretar as medidas do n\'umero de muons em termos
de composi\c{c}\~ao de raios c\'osmicos considerando o problema do d\'eficit de m\'uons.
Segundo, a proposta e o teste de um observ\'avel que \'e sens\'ivel ao espectro de energia dos m\'uons
na superf\'icie e, consequentemente, pode ser usado para constrair os modelos de intera\c{c}\~ao hadr\^onica.
Por \'ultima, a produ\c{c}\~ao de m\'uons em chuveiros \'e estudada atrav\'es
de medidas do espectro de produ\c{c}\~ao de hadrons em intera\c{c}\~oes do tipo p\'ion-carbono.



 \textbf{Palavras-chave}: Raios c\'osmicos de altas energies. Chuveiros atmsf\'ericos extensos. Componente mu\^onica de chuveiros atmosf\'ericos. F\'isica de chuveiros atmosf\'ericos.   
\end{resumo}

% ---

% ---
% inserir lista de figurass
% ---
\pdfbookmark[0]{\listfigurename}{lof}
\listoffigures*
\cleardoublepage
% ---

% ---
% inserir lista de tabelas
% ---
\pdfbookmark[0]{\listtablename}{lot}
\listoftables*
\cleardoublepage
% ---

% ---
% inserir lista de quadros
% ---
%\pdfbookmark[0]{\listofquadroname}{loq}
%\listofquadro*
%\cleardoublepage
% ---

% ---
% inserir lista de abreviaturas e siglas
% ---
%\begin{siglas}
%    \item[ABNT] Associação Brasileira de Normas Técnicas
%    \item[abnTeX] ABsurdas Normas para TeX
%	\item[EESC] Escola de Engenharia de São Carlos
%	\item[IAU] Instituto de Arquitetura e Urbanismo
%	\item[IBGE] Instituto Brasileiro de Geografia e Estatística
%	\item[ICMC] Instituto de Ciências Matemáticas e de Computação
%	\item[IFSC] Instituto de Física de São Carlos
%	\item[IQSC] Instituto de Química de São Carlos
%	\item[PDF] Portable Document Format
%	\item[TCC] Trabalho de Conclusão de Curso
%	\item[USP] Universidade de São Paulo
%	\item[USPSC] Campus USP de São Carlos
%\end{siglas}
% ---

% ---
% inserir lista de símbolos
% ---
%\begin{simbolos}
%  \item[$ \Gamma $] Letra grega Gama
%  \item[$ \Lambda $] Lambda
%  \item[$ \zeta $] Letra grega minúscula zeta
%  \item[$ \in $] Pertence
%\end{simbolos}
% ---
% ---
% inserir o sumario
% ---
\pdfbookmark[0]{\contentsname}{toc}
\tableofcontents*
\cleardoublepage
% ---
% ----------------------------------------------------------
% ELEMENTOS TEXTUAIS
% ----------------------------------------------------------
\textual

\chapter[Introduction]{Introduction}
\label{sec:introduction}



\chapter{Cosmic rays and the Pierre Auger Observatory}
\label{sec:uhecr}


\cite{Mollerach:2017idb}

%%%%%%%%%%%%%%%%%%%%%%%%%%%%%%%%%%%%
\section{Overview of cosmic rays}
\label{sec:uhecr:overview}

%%================================%%
\subsection{History}

-history: HESS, Auger and UHE

%%================================%%
\subsection{Energy spectrum}

%%%%%%%%%%%%%% SPEC SWORDY %%%%%%%%%%%%%%%
\begin{wrapfigure}{r}{0.55\textwidth}
  \centering
  \includegraphics[width=0.55\textwidth]{spectrum_swordy}
  \caption{\cite{SwordyPlot2001}}
  \label{fig:uhecr:overview:spec:swordy}
\end{wrapfigure}

The cosmic rays flux as a function of its energy, the so called \emph{energy spectrum},
plays a central hole in the understanding of the astrophysical aspects behind these particles.
A compilation of measurements of the cosmic rays
energy spectrum over about 13 orders of magnitude
in energy is shown in~\cref{fig:uhecr:overview:spec:swordy}
One can see that from about \E{11} up to the highest energies
the spectrum can be described approximately 
by a power law $\text{d}\phi/\text{d}E \sim E^{-\gamma}$, where
the spectral index $\gamma$ is not exactly constant over all the energy range
but it changes only slightly between 2.5 and 3.2.
Because the cosmic rays spectrum extends over a very large
energy range, it is expected that different astrophysical mechanisms,
occuring at distinct scales, contribute to the origin
of the cosmic rays from different regions of the spectrum. As an illustration
we can point out that most of the modern models assume that the cosmic rays flux
is dominated by particles produced inside our galaxy up to around \e{17}-\E{18} and
an extragalactic origin for particles above this energy. The transition between
these two components is currently one of the most relevant topics of discussion.


Because of the spectrum steepness, the particle fluxes change over 27 orders of magnitude
from \e{11} and \E{21}. As indicated in~\cref{fig:uhecr:overview:spec:swordy},
while at \E{11} the flux is about 1 particle/m$^2$/second,
at \E{19} it drops to only 1 particle/km$^2$/year.
As a consequence, different experimental techniques have to be used to detect
cosmic ray particles at different energy ranges. Up to around \E{14}, the low flux
allows us to use small area instruments installed in ballons or satellites
to detects the particles before they interact with the earth's atmosphere.
Above this energy, the interaction with the atmosphere turns to be
useful since we can measure the cosmic rays indirectly through the detection of
the extensive air showers (see~\cref{sec:showers}). For that, large arrays of detectors
are used and their areas can vary substantially depending on the energy range they are
intended to study. As an example, the KASCADE experiment~\cite{\KASCADEPaper}
that is designed to measure particles from around \e{14} to \E{16}
has an area of 4000 m$^2$, while Pierre Auger
Observatory~\cite{\AugerPaper} has an area of 3000 km$^2$ to measure particles above \E{18}.

%%================================%%
\subsubsection{Main features of the energy spectrum}

The features of the cosmic ray spectrum are identified by the changes on the
spectral index $\gamma$. To better visualize these changes, we show
in~\cref{fig:uhecr:overview:spec:pdg} a compilation of measurements of
the spectrum from \E{13} in which the flux is scaled by a factor $E^{2.6}$.
As indicated in~\cref{fig:uhecr:overview:spec:pdg}, the first feature
is the so called \emph{first-knee} and it occurs around $3\times\E{15}$. At this energy
the index changes from $\gamma\approx 2.7$ to $3.0$. The next feature is
the \emph{second-knee}, around \E{17}, where there is a further steepening
and the index goes to $\gamma\approx 3.3$. The spectrum then becomes harder again
at the \emph{ankle}, around $5\times\E{18}$, where the index changes to $\gamma\approx 2.6$.
The final feature occurs at the highest energies, above $4\E{19}$, where the
value of the spectral index becomes very high ($>4$), caracterizing the so called
\emph{suppression}.


%%%%%%%%%%%%%% SPEC PDG %%%%%%%%%%%%%%%
\begin{figure}
  \centering
  
  \begin{overpic}[clip, rviewport=0 0 1 1,width=0.8\textwidth]{spectrum_pdg}
    \put(18,60){}
  \end{overpic}
  
  \caption{\cite{\PDGPaper}.}
  \label{fig:uhecr:overview:spec:pdg}
\end{figure}

The consistent interpretation of all these spectrum features requires the knowledge
about the composition of the cosmic rays. Particularly in the energy region compreending
the first and second-knee, very efficient composition measurements were performed
by KASCADE~\cite{\KASCADEPaper} and
KASCADE-Grande~\cite{\KASCADEGrandePaper} experiments. By using an experimental setup
that included surface detectors able to measured both number of charged particles and
number of muons in air showers, it was possible to measure the all particle spectrum
as well as to infer the spectra of individual groups of particles. In~\cref{}
we show the results of both experiments. Althought the final spectra are
strongly dependend on the hadronic interaction models
(see~\cref{sec:shower:simulations:models}) used in the analysis,
it is still possible to conclude that: (a) the first-knee is the result
of the suppression of the proton component of the spectrum~\cite{Antoni:2005wq},
(b) the suppression of the heavier components is consistent with
a rigidity dependent suppression~\cite{Antoni:2005wq} and (c) the second-knee coincides
with the suppression of the heaviest group of particles,
including iron nuclei~\cite{Apel:2011mi,Apel:2013uni}. 

Although the rigidity dependent suppression as an explanation for the knee
was a known hypothesis since it was suggested by Peters, in 1961~\cite{Peters1961},
the actual explanation is still a matter of discussion. The most simple
model would assume that the suppression is a consequence of a limit
in the maximum energy reachable by the sources~\cite{Gaisser:2013bla}.  
Alternative models include explanations based on the effect of the
drifting of cosmic rays in the galactic magnetic field~\cite{Ptuskin1993,Candia:2002qd}
and on the escape of cosmic rays from the galaxy~\cite{Giacinti:2014xya}.
Most of the models, however, converge on the fact that up to the second-knee
the cosmic ray flux is dominated by a galactic component. The transition
between this galactic and an extra-galactic component would then occur
at energies around the second-knee and the ankle.
The measurements and the models concerning this energy range (above \E{17}),
which is the range of interest of this thesis,
will be presented in~\cref{sec:uhecr:spectrum}.


%%================================%%
\subsection{Acceleration and sources}

The power-law shape of the energy spectrum indicates that
cosmic rays are not accelerated in thermal processes.
A stochastic acceleration mechanism was first proposed
by Fermi, in 1949~\cite{Fermi:1949ee}. The main idea is that
the multiple collisions of charged particles with moving magnetized regions
could accelerate them up to high energies. 
Althought a power law spectrum can be derived from this mechanism,
the overall acceleration efficiency is too low to explain the observed
energy density of cosmic rays. The average energy gain is given by $\Delta E/E\sim \beta^2$,
where $\beta = v/c$ and $v$ is the velocity of magnetic cloud.
Because of this second power of $\beta$, this mechanism is
called \emph{second order Fermi mechanism}.

A similar but more efficient mechanism was developed almost
two decades later~\cite{Axford1977,Krymsky1977,Bell:1978zc,Blandford:1978ky}.
The main idea now is that charged particles collide with multiple shock waves
and they gain energy by interacting with irregularities in the magnetic field.
The average energy gain is $\Delta E/E\sim \beta$ and again the power law energy
spectrum can be derived. The obtained spectral index is $\gamma=2-2.3$, which means that
propagation effects have to be responsible to account for the further speepning
of the observed spectrum.
This mechanism is called \emph{first order Fermi mechanism} or, alternatively,
\emph{diffusive shock acceleration}, and it is the basis of most of the models
that propose possible sources of cosmic rays.
A detailed approach on the diffusive shock acceleration theory can be found
in Ref.~\cite{Drury:1983zz} and a review on alternative acceleration
models in Ref.\cite{}.

The list of astrophysical objects usually considered as possible
acceleraton sites for cosmic rays includes neutron stars~\cite{Fang:2012rx},
active galactic nuclei~\cite{}, gamma ray burst~\cite{Vietri1995,Waxman:2004ez}
and others. A review of the candidates
for sources of cosmic rays can be found in Ref.~\cite{Torres:2004hk}.
Indenpendently of the details about the acceleration mechanisms in different
types of astrophysical enviroments, it is possible to estimate the maximum energy
that a certain type of object is able to accelerate particles. This was first done
by Hillas~\cite{Hillas1984} and the idea is simply consider that
the Larmos radius of the particles must be smaller than the size of the site
in which the particles are confined to be accelerated. As a reults one obtain that
\begin{equation}
  \frac{E_\text{max}}{\E{18}} \approx \frac{\beta Z}{2} \frac{B}{\mu\text{G}}\frac{L}{\text{kpc}},
\end{equation}
where $\beta$ is the tipical velocity of the shock waves in units of $c$, $Z$ is the charge
of the particles, $B$ is the magnectic field and $L$ is the tipical size of the site.
The graphical illustration of this relation is usually called \emph{Hillas-plot}.
We show one version of it, focused on the candidate sources of ultra-high energy
cosmic rays, in~\cref{fig:uhecr:overview:hillas}.
The diagonal lines show the cases of \E{20} proton and iron nuclei.

%%%%%%%%%%%%%% HILLAS PLOT %%%%%%%%%%%%%%%
\begin{figure}
  \centering
  
  \begin{overpic}[clip, rviewport=0 0 1 1,width=0.8\textwidth]{hillas}
    \put(18,60){}
  \end{overpic}
  
  \caption{\cite{Mollerach:2017idb}}
  \label{fig:uhecr:overview:hillas}
\end{figure}


Regarding cosmic rays of ultra-high energy, a famous family of models,
generically called \emph{top-down models}, assume that
they are not accelerated by astrophysical objects, but intead they are the products
of the decay of exotic super-massive particles. The origin of these exotic particles
could be, for example, topological defects from early Universe phase transition
of even dark matter. As a prediction of nearly all these models, a significant flux of
ultra-high energy photons should be observed. However, measurements
of Pierre Auger Observatory have shown that the actual flux is much lower than
the predictions~\cite{Aglietta:2007yx} and, therefore, the top-down hypothesis can be ruled out.
A review on the top-down models can be found in Ref.\cite{Olinto2000}.  




%%================================%%
\subsection{Propagation}




%%%%%%%%%%%%%%%%%%%%%%%%%%%%%%%%%%%%
\section{Energy spectrum and composition at ultra-high energies}
\label{sec:uhecr:spectrum}

-hypothesis for ankle and GZK

-combined fit

-composition at the highest energies


%%%%%%%%%%%%%%%%%%%%%%%%%%%%%%%%%%%%
\section{The Pierre Auger Observatory}
\label{sec:uhecr:auger}


%%================================%%
\subsection{AugerPrime}








\chapter[Extensive air showers]{Extensive air showers}
\label{sec:showers}


\section{EM component and the \xmax}

\section{Hadronic component and the \nmu}


\cite{Pierog:2013dya,Knapp:2002vs,Meurer:2005dt,Pierog:2017nes,Espadanal:2016jse}

average pt against spectra of muons at the ground

\section{Observables}


\subsection{\xmax}

\subsection{\nmu}


\subsection{\xmumax}





\chapter[UHECR detection and the Pierre Auger Observatory]{UHECR detection and the Pierre Auger Observatory}
\label{sec:auger}


%\chapter[\NASixtyOne experiment]{\NASixtyOne experiment}
\label{sec:shine}


\chapter[Interpretation of measurements of the number of muons in extensive air shower experiments]{Interpretation of measurements of the number of muons in extensive air shower experiments}
\label{sec:interpretation}

\includepdf[pages=-, pagecommand={}]{paper_interpretation.pdf}




\chapter[A new air-shower observable to constrain hadronic interaction models]{A new air-shower observable to constrain hadronic interaction models}
\label{sec:observable}

\includepdf[pages=-]{paper_observable.pdf}



\chapter[Hadron production in pion-carbon interactions]{Hadron production in pion-carbon interactions}
\label{sec:hadron}

\note{Introduction}

\note{Make it clear what was previously done}

%%%%%%%%%%%%%%%%%%%%%%%%%%%%%%%%%%%%%%%%
\section{Dataset and simulations}
\label{sec:hadron:data}

\note{DONE}

The $\pi^-$-C data were collected by \NASixtyOne in 2009 at two beam energies:
158 and 350 \GeVc. The $\pi^-$ beam was a secondary one
produced by the collisions of a 400 \GeVc proton beam against
a 10 cm long beryllium target. The carbon target consisted of
an isotropic graphite plate with 2 cm thickness along the beam axis.
For more details about the $\pi^-$-C dataset see Ref.\cite{\RhoPaper}.

Two trigger modes are relevant for the present analysis: the beam and
interaction trigger, which are denominated by T1 and T2 respectively.
The definition of T1 is
$\text{S1}\wedge\text{S2}\wedge\overline{\text{V0}}\wedge\overline{\text{V1}}\wedge\overline{\text{V1}'}$
and T2 is
$\text{T1}\wedge\overline{\text{S4}}$.
While the T1 assures that a beam particle
reached the target position,
the T2 is supose to eliminate events in which a beam particle
crossed the target without interacting. Because of the position of
the S4 detector, it can also be reached 
by high energy particles produced by the inelastic interaction at the target,
causing the removal of events which are desirable for the analysis.
It was verified that the rate of this events is very small and they do not
produce a significant bias on our results.
For more details about the trigger modes see Ref.~\cite{MartinThesis}.
The standard calibration algorithm applied to \NASixtyOne
data is described in Ref.~\cite{Abgrall:2008zz}.

To estimate and remove from the particle spectra the contribution
of interactions that do not occur at the target, a set of data
were also taken with the target removed. The amount of target
removed data is approximately 10\% of the total data taken.
In~\cref{sec:hadron:spec} we describe 
the procedure for the target removed subtraction.

The Monte Carlo simulation sets were generated by first generating
the primary interactions using hadronic interaction models and
then by passing the produced particles through a detailed
detector simulation based on \GeantThree package~\cite{Brun:1994aa}.
Three hadronic interaction models were used: \EposLong~\cite{\EposPaper},
\DPMJetLong~\cite{\DPMJetPaper} and \QGSJetLong~\cite{\QGSJetPaper}.
For each beam energy and hadronic interaction model,
a simulation set was produced with approximately the same event
number as the datasets. 
Both the data and the simulations were reconstructed by the standard
\NASixtyOne reconstruction chain~\cite{Abgrall:2011ae}. 

%%%%%%%%%%%%%%%%%%%%%%%%%%%%%%%%%%%%%%%%
\section{Event selection}
\label{sec:hadron:event}

\note{DONE}

The first step of the event selection is the upstream cuts,
which are based on the information from the beam detectors.  
The upstream cuts are:
\begin{enumerate}[label=(\roman*)]
\item CEDAR cut  to identify the beam particle type and then remove
  the contributions from non-pion particles.
\item WFA cut  that uses the time information from the S1 detector
  to exclude events in which a second beam particle was detected
  with a time difference shorter than 2 $\mu$s.
\item BPD cut that uses the information from the three BPD detectors
  to assure a good quality measurements of the beam position at the
  target plan. These measurements are important to contraint the
  main vertex position during the event reconstruction.
\end{enumerate}
More details about the upstream cuts can be found in Ref.~\cite{MartinThesis}.
\note{more refs}
Since the beam detectors are not implemented in the simulations,
the upstream cuts are applyied only to the data.

The second step is the event cuts, which are applyied both to data
and simulations. The three event cuts are:
\begin{enumerate}[label=(\roman*)]
\item Trigger cut  that selects events which are
  defined as T2 trigger type (see~\cref{sec:hadron:data}
  for the trigger definitions).
\item Main vertex cut to remove events in which the main vertex
  is not fitted during the event reconstruction.
\item Vertex Z cut  to remove events in which the z coordenate of the fitted
  main vertex is farer than 17 cm from the main vertex position measured
  by the BPD detectors. This cut is meant to reduce the constribution from
  out-of-target interactions. 
\end{enumerate}
In~\cref{tab:hadron:stat} we show the number of available events
after the event selection for the data and simulation sets.

\warning{include target removed in the table}

\begin{table}
  \begin{center}
    \begin{tabular}{|l|c|c|} \hline
                  & 158 GeV/c       & 350 GeV/c \\ \hline
      Data        & 3.46 $10^6$     & 3.04 $10^6$ \\
      \EposLong   & 3.71 $10^6$     & 3.12 $10^6$ \\
      \DPMJetLong & 3.92 $10^6$     & 3.47 $10^6$ \\
      \QGSJetLong & 3.71 $10^6$     & 3.06 $10^6$ \\ \hline
    \end{tabular}
    \caption{}
    \label{tab:hadron:stat}
  \end{center}
\end{table}

%%%%%%%%%%%%%%%%%%%%%%%%%%%%%%%%%%%%%%%%
\section{Track selection}
\label{sec:hadron:trackselection}

\note{DONE}

\note{define labels indentified spectra and \vzero analysis}

The following selection criteria were applyied to the measured tracks
for the identified spectra analysis:
\begin{enumerate}[label=(\roman*)]
\item The reconstructed track must be contained in the detector acceptance,
  that is basically defined as regions in ($\phi$,\p,\pT) phase space
  in which the selection efficiency is larger than 90\%. A second effect
  that is also accounted in the definition of the acceptance is the tracks that
  hit directly the S4 detector and are not removed by the T2 trigger selection.
  While the selection efficiency contribution is estimated with Monte Carlo simulation,
  the directly hits on S4 are removed based on the measured tracks.
  The full description of the acceptance selection can be found in Ref.\cite{MartinThesis}.
\item The total number of clusters on the track must be greater than or equal to 25.
\item The sum of clusters on both VTPCs must be greater than or equal to 12, or
  the number of cluster on the GTPC must be greater than of equal to 6.
\item The distance between the extrapolated track to the interaction plane and the
  interaction point, that is called impact parameter, must be smaller than 4 cm
  in the both horizontal and vertical plane.
\end{enumerate}
\note{some explanations on the idea of the acceptance cut}
More details about the track selection
can be found in Ref.~\cite{MartinThesis}.

%%%%%%%%%%%%%%%%%%%%%%%%%%%%%%%%%%%%%%%%
\section{\vzero selection}
\label{sec:hadron:vzeroselection}

\note{DONE}

The \vzero selection criteria used for the \vzero analysis
is the following:
\begin{enumerate}[label=(\roman*)]
\item The selected vertex must be identified as a \vzero type vertex.
\item The number of daughter tracks of the vertex must be equal to 2.
\item Both daughter tracks must be of opposite charges.
\item The total number of cluster has to be greater than 30 for both tracks
\item At least one track has to have more than 15 clusters in the VTPCs
\end{enumerate}
These selection criteria are standard ones in \NASixtyOne analysis.
Further cuts on the \vzeros will be applyied at the signal extraction
step (see~\cref{sec:hadron:signal:cuts}).

Since it is not possible to define the detector acceptance for
the \vzeros analogously to what is done for the tracks, the possible
discrepancies between data and simulations on the bourders of the acceptance
will be accounted on the systematic uncertainties (see~\cref{sec:hadron:spec:syst}).
Because these tracks on the bourders have small number of clusters,
the systematic uncertanty will be estimatede by changing the minimum number
of clusters on both tracks from 30 to 20. 


%%%%%%%%%%%%%%%%%%%%%%%%%%%%%%%%%%%%%%%%
\section{Phase space binning}

\note{DONE}

Both identified spectra and \vzero analysis were done
by splitting the data in bins in a 2-dimensional
phase space of the \pp and \pT variables. For the
identified spectra analysis only one phase space
binning configuration is defined. The intervals in \pp
are nearly uniform in $\log p$, with small adjustments in a way
that moves the crossing points of the energy deposit function
of different particles closer to the center of the bins.
Since some of these bins in the crossing regions
will be removed from the analysis (see~\cref{sec:hadron:dedx:sde}),
this strategy has shown effective to reduce the number of
removed bins. The average width of the \pp intervals is
$\Delta\log\pp=0.1$. Concerning the \pT intervals, the bin width
increases with \pT, being the width of the shorter and the longer one
$\Delta \pT=0.1$ and  $\Delta\pT=0.5$, respectively.  
In~\cref{fig:hadron:binning:dedx} we show the binning configuration
for the identified spectra analysis.


%%%%%%%%%%%%%% BINNING DEDX %%%%%%%%%%%%%%%
\begin{figure}[!ht]
  \centering
  \includegraphics[clip, rviewport=0 0 1 1,width=0.5\textwidth]{DedxBinning}
  \caption{}
  \label{fig:hadron:binning:dedx}
\end{figure}

Since the \vzero analysis is done independently for the three target particles,
the phase space binning is not required to be unique. Because of statistics,
the number of bins defined for the \lamb and \antilamb is the same and
for \kzeros is larger than for the former ones.
In~\cref{fig:hadron:binning:vzero} we show the two
binning configuration for the \vzero analysis.

%%%%%%%%%%%%%% BINNING V0 %%%%%%%%%%%%%%%
\begin{figure}[!ht]
  \centering
  \includegraphics[clip, rviewport=0 0 1 1,width=0.8\textwidth]{V0Binning}
  \caption{}
  \label{fig:hadron:binning:vzero}
\end{figure}


%%%%%%%%%%%%%%%%%%%%%%%%%%%%%%%%%%%%%%%%
\section{Particle identification for the identified spectra}

\note{DONE}

In this section we present the particle identification
analysis for the identified spectra of \pions, \kaons and \protons.
This step is done in a track basis through the \dedx measurements,
being the aim here to determine the fraction of tracks which
correspond to each particle type on every phase space bin.
A brief overview of the \dedx measurements is first given
in~\cref{sec:hadron:dedx:meas}.

The \dedx measurements only allow the particle identification
to be done statistically by fitting
the \dedx distributions with a combination of particle
types. Because of the complicated dependence of the \dedx
distributions on the particle momentum,
features of the measured track (e.g. number of clusters) and
detector properties (e.g. resolution and calibration),
the \dedx fit turns to be very challenging. The first
requirement to perform this step is the development
of a appropriate \dedx model,
which is shown in~\cref{sec:hadron:dedx:model}.

Having in hands the \dedx model, the measured \dedx
distributions can be fitted to determine the particle
fractions. However, the usual large number of
model parameters, added to the
overlap of the \dedx distributions
of different particles in certain regions of momentum,
can make the this fit very hard to perform.
Our fit strategy to overcome these difficulties is shown
in~\cref{sec:hadron:dedx:fit}. A new tool
developed in this work to evaluate the fit performance
and estimated bias and statistical uncertainties of the fit
is presented in~\cref{sec:hadron:dedx:sde}. 
Finally, in~\cref{sec:hadron:dedx:results} we show the results
of the particle identification analysis.


%%%---------------------------------%%%%
\subsection{\dedx measurements}
\label{sec:hadron:dedx:meas}

\cite{Alt:2005zq}

\cite{BlumBook}

\warning{rewrite}

The \dedx associated to each track is defined as the energy lost by the charged
particle per unit of length.
In \NASixtyOne the \dedx is measured by the TPCs, which collect the number of
freed electrons from the ionization of the gas by the passage of the charged particles.
The determination of the \dedx from the signal recorded at the TPCs requires a complex and
detailed procedure, which has been very well established by the \NAFortyNine and \NASixtyOne
experiment along the last decades. Since the detailed description of this procedure
is out of the scope of this text, only the general idea and the most important aspects
will be presented in the next paragraphs. More complete and detailed approaches
can be found in Refs.\cite{LeeuwenThesis,GaborVeresThesis}.

Several processes can contribute to the energy loss of charged particle due to
its interaction with atoms of the gas in the TPCs, being the emission of
electrons by ionization the most relevant one. The electrons emitted are
drifted through the chamber and collected in the readout pads, which records
the signal as ADC charges. A set of consecutive charges defines a cluster.
The 3-dimensional position of the cluster is determined by the position
and time distributions in which the charges reaches the readout pad. This
position gives the crossing point of the particle track inside the TPCs.

The total charge measured in each cluster is related to the \dedx of each track.
However, numerous detector effects have to be corrected at the cluster level before
grouping the cluster in one unique \dedx value. The simplest correction accounts for
the geometrical differences due to the incident angle of the track in the pad and
the pad widths. More complicated corrections account for differences in the electronic
gain and gas pressure/temperature of the pads, differences in the sector gains and
losses of electrons during the drift in the chambers and in the readout pad.
A detailed description of the correction procedure can be found in Ref.~\cite{AntoniMThesis}.

The track \dedx is then determined by the combination of the corrected 
charges in all clusters. The well known Landau-like shape of the
energy loss probability distribution makes the simplest approach,
based on the average over all clusters, not suitable. Because of the
long tail of the probability distribution, the average and the variance
of the measured charges are not well behaved for typical number of clusters
($\sim$ 20-150). To overcome this issue and obtain a satisfactory \dedx resolution,
the method of the truncated mean is applied, in which only a subset of the clusters
is selected to compute the average. The selected clusters are defined by ordering
the values of the charge and the selecting the ones inside a given percentage interval.
For the \NASixtyOne experiment, it was found the optimal interval being the smallest 50\%
of the clusters~\cite{GaborVeresThesis}.


%%%---------------------------------%%%%
\subsection{\dedx model}
\label{sec:hadron:dedx:model}

\warning{rewrite}

To perform the particle identification by fitting the
measured \dedx distribution, a model that describe
the \dedx distributions of different particle types as a function
of their momentum \p is required. Once there is no universal choice
of this model, several different alternatives have been
used in previous analysis. Although the model chosen here 
is based on previous studies developed for
\NAFortyNine and \NASixtyOne experiment, it contain
particular features which were found to be the most suitable
for the present analysis.

First, the notation adopted in this text has to be presented
for clarification. The particle types are represented by
the index \ipart, and it can assume one of the five particle types
treated here, $\ipart=e, \pi, K, p, d$. The charges are represented by
the index \ich, being that $\ich = +$ or $\ich=-$. Also, the number of
cluster measured in a track is represented by \ncl and the \dedx
is replaced by \eps for simplicity.

Because the \dedx is obtained by averaging the measured charge
over a certain number of cluster, it is natural to assume that
the shape of the \eps distribution depends on the \ncl.
To be more precise, the \eps resolution should be 
larger for smaller \ncl and vice-versa. 
Additionally, it is obviously expected the mean of the distribution
to change with the momentum of the particle \p and the particle type.

Since the shape of the \eps distribution, for a given \ncl and \p,
can be well described by an asymmetric Gaussian function, the
probability density function of \eps for a particle type
\ipart and charge \ich is written as
\begin{equation}
  f_{\ipart,\ich}(\eps|\p,\ncl) = \frac{1}{\sqrt{2\pi}\sigma_{\ipart,\ich}} \;
  \exp\left[-\frac{1}{2}\left(\frac{\eps-\mu_{\ipart,\ich}}{\delta \; \sigma_{\ipart,\ich}}\right)^2\right],
  \label{eq:dedx:model:pdf}
\end{equation}
with
\begin{equation}
  \delta =
  \begin{cases}
    & 1-d, \ \ \ \eps \le \mu_{\ipart,\ich} \\
    & 1+d, \ \ \ \eps > \mu_{\ipart,\ich}, \\
  \end{cases}
  \label{eq:dedx:model:asymm}
\end{equation}
where the parameter $\mu$ is the mode of the distribution, $\sigma$ is the resolution
and $d$ is the asymmetry parameter. The \p and \ncl dependence is implicit
on the parameters $\mu$ and $\sigma$, as will be explained next.
The mode $\mu$ is related to the mean of the distribution, \meaneps, by
\begin{equation}
  \mu_{\ipart,\ich} = \meaneps_{\ipart,\ich} - \frac{\sigma_{\ipart,\ich}}{\sqrt{2\pi}}
  \left[\left(1+d\right)^2 - \left(1-d\right)^2 \right].
  \label{eq:dedx:model:mu}
\end{equation}

The \p evolution of \meaneps is expected to follow
a Bethe-Bloch-like function. In this model, a reference
\meaneps(\p) curve is defined by a data-based
parametrization using a generic function which is a
variation of the Bethe-Bloch function. The reference value
of \meaneps for a given \p is denoted as \meanepsbb.
To account for deviations from the reference \meaneps,
the present model includes a set of parameters called
\textit{calibration constants}, which are denoted by $X$.
These parameters act as logarithmic shifts of the \meaneps
around \meanepsbb and they can in principle be applied
to each particle and charge separately. To reduce the complexity
of the model, it is assumed here one global calibration constant
for each charge that follows the \meaneps of the $\pi$ distribution
and individual calibration constants for the other particles,
but being common for both charges. In the end, the \meaneps for a
given particle type \ipart and charge \ich is given by
\begin{equation}
  \meaneps_{\ipart,\ich} =
  \begin{cases}
    & \meaneps_{\ipart}^\text{BB} \; e^{X_{\ipart}^{\ich}} \ \ \ \ \ \ \ (i=\pi) \\
    & \meaneps_{\ipart}^\text{BB} \; e^{X_{\pi}^{\ich}} \; e^{X_{\ipart}^{\ich}} \ \ \ (i\neq\pi).
  \end{cases}
  \label{eq:dedx:model:cal}
\end{equation}
In total, 6 calibration constants are included in the model:
$X_{\pi}^{+}$, $X_{\pi}^{-}$, $X_{e}$, $X_{K}$, $X_{p}$ and $X_{d}$.

The dependence of the resolution $\sigma$ on \ncl is assumed to be of the form
$\sigma \sim 1/\sqrt{\ncl}$. Besides that, $\sigma$ is assumed to depend
on the \meaneps by a power law relation and a normalization parameter for
each charge is also included ($\sigma_0^{\ich}$). The final expression for the resolution is,
\begin{equation}
  \sigma_{\ipart,\ich} = \frac{\sigma_0^{\ich}}{\sqrt{\ncl}} \meaneps_{\ipart,\ich}^{\alpha},
  \label{eq:dedx:model:sigma}
\end{equation}
in which 3 more parameters are included: $\sigma_0^+$, $\sigma_0^-$ and $\alpha$. 

By combining
the~\cref{eq:dedx:model:asymm,eq:dedx:model:mu,eq:dedx:model:cal,eq:dedx:model:sigma}
with the~\cref{eq:dedx:model:pdf}, we obtain the probability density
function of \eps for each particle \ipart and charge \ich. Besides the 6 calibration constants,
the model includes 4 \textit{shape parameters}: $\sigma_0^+$, $\sigma_0^-$, $\alpha$ and $d$.
Altogether there are 10 parameters that can be set free to fit the model
to the measured \eps distributions.


%%%---------------------------------%%%%
\subsection{\dedx fit strategy}
\label{sec:hadron:dedx:fit}


%%%---------------------------------%%%%
\subsection{Simulated data ensembles, cuts and corrections}
\label{sec:hadron:dedx:sde}


%%%---------------------------------%%%%
\subsection{Particle identification results}
\label{sec:hadron:dedx:results}



%%%%%%%%%%%%%%%%%%%%%%%%%%%%%%%%%%%%%%%%
\section{\vzero analysis}


%%%---------------------------------%%%%
\subsection{Signal extraction}


\subsubsection{\vzero cuts}
\label{sec:hadron:signal:cuts}

%%%%%%%%%%%%%%%%%%%%%%%%%%%%%%%%%%%%%%%%
\section{Corrections}


%%%%%%%%%%%%%%%%%%%%%%%%%%%%%%%%%%%%%%%%
\section{Spectra}


%%%---------------------------------%%%%
\subsection{Statistical uncertainties}


%%%---------------------------------%%%%
\subsection{Systematic uncertainties}
\label{sec:hadron:spec:syst}



%%%%%%%%%%%%%%%%%%%%%%%%%%%%%%%%%%%%%%%%
\section{Results}

%%%%%%%%%%%%%%%%%%%%%%%%%%%%%%%%%%%%%%%%
\section{Summary and conclusions}



\clearpage

%%%%%%%%%%% DIST %%%%%%%%%%%%%%%%%%%
\begin{figure}
  \centering
  \includegraphics[clip, rviewport=0 0 1 1,width=0.4\textwidth]{dedx/dist_350_v0_c0_x13_y3}
  \includegraphics[clip, rviewport=0 0 1 1,width=0.4\textwidth]{dedx/dist_350_v0_c1_x13_y3}

  \vspace{0.5cm}
  
  \includegraphics[clip, rviewport=0 0 1 1,width=0.4\textwidth]{dedx/dist_350_v1_c0_x29_y5}
  \includegraphics[clip, rviewport=0 0 1 1,width=0.4\textwidth]{dedx/dist_350_v1_c1_x29_y5}
  
  \caption{Examples of the fitted \dedx distributions from the 350 \GeVc dataset.}
  \label{fig:hadron:dedx:fit:dist350}
\end{figure}

%%%%%%%%%% CAL %%%%%%%%%%%%%%
\begin{figure}
  \centering
  \includegraphics[clip, rviewport=0 0 1 0.94,width=0.4\textwidth]{dedx/model_158_v1_m0}
  \includegraphics[clip, rviewport=0 0 1 0.94,width=0.4\textwidth]{dedx/model_158_v1_m1}

  \includegraphics[clip, rviewport=0 0 1 0.94,width=0.4\textwidth]{dedx/model_158_v1_m2}
  \includegraphics[clip, rviewport=0 0 1 0.94,width=0.4\textwidth]{dedx/model_158_v1_m3}

  \includegraphics[clip, rviewport=0 0 1 0.94,width=0.4\textwidth]{dedx/model_158_v1_m4}
  \includegraphics[clip, rviewport=0 0 1 0.94,width=0.4\textwidth]{dedx/model_158_v1_m5}
  \caption{Calibration constants obtained from the fit of the WST dataset at 158 \GeVc.}
  \label{fig:hadron:dedx:fit:cal158w}
\end{figure}

%%%%%%%%%% CAL %%%%%%%%%%%%%%
\begin{figure}
  \centering
  \includegraphics[clip, rviewport=0 0 1 0.94,width=0.4\textwidth]{dedx/model_350_v0_m0}
  \includegraphics[clip, rviewport=0 0 1 0.94,width=0.4\textwidth]{dedx/model_350_v0_m1}

  \includegraphics[clip, rviewport=0 0 1 0.94,width=0.4\textwidth]{dedx/model_350_v0_m2}
  \includegraphics[clip, rviewport=0 0 1 0.94,width=0.4\textwidth]{dedx/model_350_v0_m3}

  \includegraphics[clip, rviewport=0 0 1 0.94,width=0.4\textwidth]{dedx/model_350_v0_m4}
  \includegraphics[clip, rviewport=0 0 1 0.94,width=0.4\textwidth]{dedx/model_350_v0_m5}
  \caption{Calibration constants obtained from the fit of the RST dataset at 350 \GeVc.}
  \label{fig:hadron:dedx:fit:cal350r}
\end{figure}

%%%%%%%%%% CAL %%%%%%%%%%%%%%
\begin{figure}
  \centering
  \includegraphics[clip, rviewport=0 0 1 0.94,width=0.4\textwidth]{dedx/model_350_v1_m0}
  \includegraphics[clip, rviewport=0 0 1 0.94,width=0.4\textwidth]{dedx/model_350_v1_m1}

  \includegraphics[clip, rviewport=0 0 1 0.94,width=0.4\textwidth]{dedx/model_350_v1_m2}
  \includegraphics[clip, rviewport=0 0 1 0.94,width=0.4\textwidth]{dedx/model_350_v1_m3}

  \includegraphics[clip, rviewport=0 0 1 0.94,width=0.4\textwidth]{dedx/model_350_v1_m4}
  \includegraphics[clip, rviewport=0 0 1 0.94,width=0.4\textwidth]{dedx/model_350_v1_m5}
  \caption{Calibration constants obtained from the fit of the WST dataset at 350 \GeVc.}
  \label{fig:hadron:dedx:fit:cal350w}
\end{figure}

\clearpage

%%%%%%%%%% SHAPE %%%%%%%%%%%%%%
\begin{figure}
  \centering
  \includegraphics[clip, rviewport=0 0 1 0.94,width=0.4\textwidth]{dedx/model_158_v1_m6}
  \includegraphics[clip, rviewport=0 0 1 0.94,width=0.4\textwidth]{dedx/model_158_v1_m7}

  \includegraphics[clip, rviewport=0 0 1 0.94,width=0.4\textwidth]{dedx/model_158_v1_m9}
  \includegraphics[clip, rviewport=0 0 1 0.94,width=0.4\textwidth]{dedx/model_158_v1_m10}
  \caption{Shape parameters obtained from the fit of the WST dataset at 158 \GeVc.}
  \label{fig:hadron:dedx:fit:shape158w}
\end{figure}

%%%%%%%%%% SHAPE %%%%%%%%%%%%%%
\begin{figure}
  \centering
  \includegraphics[clip, rviewport=0 0 1 0.94,width=0.4\textwidth]{dedx/model_350_v0_m6}
  \includegraphics[clip, rviewport=0 0 1 0.94,width=0.4\textwidth]{dedx/model_350_v0_m7}

  \includegraphics[clip, rviewport=0 0 1 0.94,width=0.4\textwidth]{dedx/model_350_v0_m9}
  \includegraphics[clip, rviewport=0 0 1 0.94,width=0.4\textwidth]{dedx/model_350_v0_m10}
  \caption{Shape parameters obtained from the fit of the RST dataset at 350 \GeVc.}
  \label{fig:hadron:dedx:fit:shape350r}
\end{figure}

%%%%%%%%%% SHAPE %%%%%%%%%%%%%%
\begin{figure}
  \centering
  \includegraphics[clip, rviewport=0 0 1 0.94,width=0.4\textwidth]{dedx/model_350_v1_m6}
  \includegraphics[clip, rviewport=0 0 1 0.94,width=0.4\textwidth]{dedx/model_350_v1_m7}

  \includegraphics[clip, rviewport=0 0 1 0.94,width=0.4\textwidth]{dedx/model_350_v1_m9}
  \includegraphics[clip, rviewport=0 0 1 0.94,width=0.4\textwidth]{dedx/model_350_v1_m10}
  \caption{Shape parameters obtained from the fit of the WST dataset at 350 \GeVc.}
  \label{fig:hadron:dedx:fit:shape350w}
\end{figure}

\clearpage


%%%%%%%%%% FRACTIONS %%%%%%%%%%%%%%
\begin{figure}
  \centering
  \includegraphics[clip, rviewport=0 0.13 1 0.94,width=0.4\textwidth]{dedx/fraction_158_fl0_v1_c0_p0}
  \includegraphics[clip, rviewport=0 0.13 1 0.94,width=0.4\textwidth]{dedx/fraction_158_fl0_v1_c1_p0}

  \includegraphics[clip, rviewport=0 0.13 1 0.94,width=0.4\textwidth]{dedx/fraction_158_fl0_v1_c0_p1}
  \includegraphics[clip, rviewport=0 0.13 1 0.94,width=0.4\textwidth]{dedx/fraction_158_fl0_v1_c1_p1}

  \includegraphics[clip, rviewport=0 0.13 1 0.94,width=0.4\textwidth]{dedx/fraction_158_fl0_v1_c0_p2}
  \includegraphics[clip, rviewport=0 0.13 1 0.94,width=0.4\textwidth]{dedx/fraction_158_fl0_v1_c1_p2}


  \includegraphics[clip, rviewport=0 0.13 1 0.94,width=0.4\textwidth]{dedx/fraction_158_fl0_v1_c0_p3}
  \includegraphics[clip, rviewport=0 0.13 1 0.94,width=0.4\textwidth]{dedx/fraction_158_fl0_v1_c1_p3}

  \includegraphics[clip, rviewport=0 0 1 0.94,width=0.4\textwidth]{dedx/fraction_158_fl0_v1_c0_p4}
  \includegraphics[clip, rviewport=0 0 1 0.94,width=0.4\textwidth]{dedx/fraction_158_fl0_v1_c1_p4}

  \caption{Particle fractions obtained from the fit of the WST dataset at 158 \GeVc.}
  \label{fig:hadron:dedx:fit:frac158w}
\end{figure}


%%%%%%%%%% FRACTIONS %%%%%%%%%%%%%%
\begin{figure}
  \centering
  \includegraphics[clip, rviewport=0 0.13 1 0.94,width=0.4\textwidth]{dedx/fraction_350_fl0_v0_c0_p0}
  \includegraphics[clip, rviewport=0 0.13 1 0.94,width=0.4\textwidth]{dedx/fraction_350_fl0_v0_c1_p0}

  \includegraphics[clip, rviewport=0 0.13 1 0.94,width=0.4\textwidth]{dedx/fraction_350_fl0_v0_c0_p1}
  \includegraphics[clip, rviewport=0 0.13 1 0.94,width=0.4\textwidth]{dedx/fraction_350_fl0_v0_c1_p1}

  \includegraphics[clip, rviewport=0 0.13 1 0.94,width=0.4\textwidth]{dedx/fraction_350_fl0_v0_c0_p2}
  \includegraphics[clip, rviewport=0 0.13 1 0.94,width=0.4\textwidth]{dedx/fraction_350_fl0_v0_c1_p2}


  \includegraphics[clip, rviewport=0 0.13 1 0.94,width=0.4\textwidth]{dedx/fraction_350_fl0_v0_c0_p3}
  \includegraphics[clip, rviewport=0 0.13 1 0.94,width=0.4\textwidth]{dedx/fraction_350_fl0_v0_c1_p3}

  \includegraphics[clip, rviewport=0 0 1 0.94,width=0.4\textwidth]{dedx/fraction_350_fl0_v0_c0_p4}
  \includegraphics[clip, rviewport=0 0 1 0.94,width=0.4\textwidth]{dedx/fraction_350_fl0_v0_c1_p4}

  \caption{Particle fractions obtained from the fit of the RST dataset at 350 \GeVc.}
  \label{fig:hadron:dedx:fit:frac350r}
\end{figure}

%%%%%%%%%% FRACTIONS %%%%%%%%%%%%%%
\begin{figure}
  \centering
  \includegraphics[clip, rviewport=0 0.13 1 0.94,width=0.4\textwidth]{dedx/fraction_350_fl0_v1_c0_p0}
  \includegraphics[clip, rviewport=0 0.13 1 0.94,width=0.4\textwidth]{dedx/fraction_350_fl0_v1_c1_p0}

  \includegraphics[clip, rviewport=0 0.13 1 0.94,width=0.4\textwidth]{dedx/fraction_350_fl0_v1_c0_p1}
  \includegraphics[clip, rviewport=0 0.13 1 0.94,width=0.4\textwidth]{dedx/fraction_350_fl0_v1_c1_p1}

  \includegraphics[clip, rviewport=0 0.13 1 0.94,width=0.4\textwidth]{dedx/fraction_350_fl0_v1_c0_p2}
  \includegraphics[clip, rviewport=0 0.13 1 0.94,width=0.4\textwidth]{dedx/fraction_350_fl0_v1_c1_p2}


  \includegraphics[clip, rviewport=0 0.13 1 0.94,width=0.4\textwidth]{dedx/fraction_350_fl0_v1_c0_p3}
  \includegraphics[clip, rviewport=0 0.13 1 0.94,width=0.4\textwidth]{dedx/fraction_350_fl0_v1_c1_p3}

  \includegraphics[clip, rviewport=0 0 1 0.94,width=0.4\textwidth]{dedx/fraction_350_fl0_v1_c0_p4}
  \includegraphics[clip, rviewport=0 0 1 0.94,width=0.4\textwidth]{dedx/fraction_350_fl0_v1_c1_p4}

  \caption{Particle fractions obtained from the fit of the WST dataset at 350 \GeVc.}
  \label{fig:hadron:dedx:fit:frac350w}
\end{figure}

\clearpage



%%%%%%%%%% FAKE REL SIG %%%%%%%%%%%%%%
\begin{figure}
  \centering
  \includegraphics[clip, rviewport=0 0.13 1 0.94,width=0.4\textwidth]{dedx/fake_rel_sig_158_fl0_v1_c0_p1}
  \includegraphics[clip, rviewport=0 0.13 1 0.94,width=0.4\textwidth]{dedx/fake_rel_sig_158_fl0_v1_c1_p1}

  \includegraphics[clip, rviewport=0 0.13 1 0.94,width=0.4\textwidth]{dedx/fake_rel_sig_158_fl0_v1_c0_p2}
  \includegraphics[clip, rviewport=0 0.13 1 0.94,width=0.4\textwidth]{dedx/fake_rel_sig_158_fl0_v1_c1_p2}

  \includegraphics[clip, rviewport=0 0 1 0.94,width=0.4\textwidth]{dedx/fake_rel_sig_158_fl0_v1_c0_p3}
  \includegraphics[clip, rviewport=0 0 1 0.94,width=0.4\textwidth]{dedx/fake_rel_sig_158_fl0_v1_c1_p3}

  \caption{Relative standard deviation of the particle fractions obtained with SDEs for WST and 158 \GeVc dataset.}
  \label{fig:hadron:dedx:fit:fake:relsig158w}
\end{figure}



%%%%%%%%%% FAKE REL SIG %%%%%%%%%%%%%%
\begin{figure}
  \centering
  \includegraphics[clip, rviewport=0 0.13 1 0.94,width=0.4\textwidth]{dedx/fake_rel_sig_350_fl0_v0_c0_p1}
  \includegraphics[clip, rviewport=0 0.13 1 0.94,width=0.4\textwidth]{dedx/fake_rel_sig_350_fl0_v0_c1_p1}

  \includegraphics[clip, rviewport=0 0.13 1 0.94,width=0.4\textwidth]{dedx/fake_rel_sig_350_fl0_v0_c0_p2}
  \includegraphics[clip, rviewport=0 0.13 1 0.94,width=0.4\textwidth]{dedx/fake_rel_sig_350_fl0_v0_c1_p2}

  \includegraphics[clip, rviewport=0 0 1 0.94,width=0.4\textwidth]{dedx/fake_rel_sig_350_fl0_v0_c0_p3}
  \includegraphics[clip, rviewport=0 0 1 0.94,width=0.4\textwidth]{dedx/fake_rel_sig_350_fl0_v0_c1_p3}


  \caption{Relative standard deviation of the particle fractions obtained with SDEs for RST and 350 \GeVc dataset.}
  \label{fig:hadron:dedx:fit:fake:relsig350r}
\end{figure}

%%%%%%%%%% FAKE REL SIG %%%%%%%%%%%%%%
\begin{figure}
  \centering
  \includegraphics[clip, rviewport=0 0.13 1 0.94,width=0.4\textwidth]{dedx/fake_rel_sig_350_fl0_v1_c0_p1}
  \includegraphics[clip, rviewport=0 0.13 1 0.94,width=0.4\textwidth]{dedx/fake_rel_sig_350_fl0_v1_c1_p1}

  \includegraphics[clip, rviewport=0 0.13 1 0.94,width=0.4\textwidth]{dedx/fake_rel_sig_350_fl0_v1_c0_p2}
  \includegraphics[clip, rviewport=0 0.13 1 0.94,width=0.4\textwidth]{dedx/fake_rel_sig_350_fl0_v1_c1_p2}

  \includegraphics[clip, rviewport=0 0 1 0.94,width=0.4\textwidth]{dedx/fake_rel_sig_350_fl0_v1_c0_p3}
  \includegraphics[clip, rviewport=0 0 1 0.94,width=0.4\textwidth]{dedx/fake_rel_sig_350_fl0_v1_c1_p3}

  \caption{Relative standard deviation of the particle fractions obtained with SDEs for WST and 350 \GeVc dataset.}
  \label{fig:hadron:dedx:fit:fake:relsig350w}
\end{figure}


%%%%%%%%%% FAKE REL DEV %%%%%%%%%%%%%%
\begin{figure}
  \centering
  \includegraphics[clip, rviewport=0 0.13 1 0.94,width=0.4\textwidth]{dedx/fake_rel_dev_158_fl0_v1_c0_p1}
  \includegraphics[clip, rviewport=0 0.13 1 0.94,width=0.4\textwidth]{dedx/fake_rel_dev_158_fl0_v1_c1_p1}

  \includegraphics[clip, rviewport=0 0.13 1 0.94,width=0.4\textwidth]{dedx/fake_rel_dev_158_fl0_v1_c0_p2}
  \includegraphics[clip, rviewport=0 0.13 1 0.94,width=0.4\textwidth]{dedx/fake_rel_dev_158_fl0_v1_c1_p2}

  \includegraphics[clip, rviewport=0 0 1 0.94,width=0.4\textwidth]{dedx/fake_rel_dev_158_fl0_v1_c0_p3}
  \includegraphics[clip, rviewport=0 0 1 0.94,width=0.4\textwidth]{dedx/fake_rel_dev_158_fl0_v1_c1_p3}

  \caption{Average relative bias of the particle fractions obtained with SDEs for WST and 158 \GeVc dataset.}
  \label{fig:hadron:dedx:fit:fake:reldev158w}
\end{figure}



%%%%%%%%%% FAKE REL DEV %%%%%%%%%%%%%%
\begin{figure}
  \centering
  \includegraphics[clip, rviewport=0 0.13 1 0.94,width=0.4\textwidth]{dedx/fake_rel_dev_350_fl0_v0_c0_p1}
  \includegraphics[clip, rviewport=0 0.13 1 0.94,width=0.4\textwidth]{dedx/fake_rel_dev_350_fl0_v0_c1_p1}

  \includegraphics[clip, rviewport=0 0.13 1 0.94,width=0.4\textwidth]{dedx/fake_rel_dev_350_fl0_v0_c0_p2}
  \includegraphics[clip, rviewport=0 0.13 1 0.94,width=0.4\textwidth]{dedx/fake_rel_dev_350_fl0_v0_c1_p2}

  \includegraphics[clip, rviewport=0 0 1 0.94,width=0.4\textwidth]{dedx/fake_rel_dev_350_fl0_v0_c0_p3}
  \includegraphics[clip, rviewport=0 0 1 0.94,width=0.4\textwidth]{dedx/fake_rel_dev_350_fl0_v0_c1_p3}


  \caption{Average relative bias of the particle fractions obtained with SDEs for RST and 350 \GeVc dataset.}
  \label{fig:hadron:dedx:fit:fake:reldev350r}
\end{figure}

%%%%%%%%%% FAKE REL DEV %%%%%%%%%%%%%%
\begin{figure}
  \centering
  \includegraphics[clip, rviewport=0 0.13 1 0.94,width=0.4\textwidth]{dedx/fake_rel_dev_350_fl0_v1_c0_p1}
  \includegraphics[clip, rviewport=0 0.13 1 0.94,width=0.4\textwidth]{dedx/fake_rel_dev_350_fl0_v1_c1_p1}

  \includegraphics[clip, rviewport=0 0.13 1 0.94,width=0.4\textwidth]{dedx/fake_rel_dev_350_fl0_v1_c0_p2}
  \includegraphics[clip, rviewport=0 0.13 1 0.94,width=0.4\textwidth]{dedx/fake_rel_dev_350_fl0_v1_c1_p2}

  \includegraphics[clip, rviewport=0 0 1 0.94,width=0.4\textwidth]{dedx/fake_rel_dev_350_fl0_v1_c0_p3}
  \includegraphics[clip, rviewport=0 0 1 0.94,width=0.4\textwidth]{dedx/fake_rel_dev_350_fl0_v1_c1_p3}

  \caption{Average relative bias of the particle fractions obtained with SDEs for WST and 350 \GeVc dataset.}
  \label{fig:hadron:dedx:fit:fake:reldev350w}
\end{figure}

\clearpage


%%%%%%%%%% COR %%%%%%%%%%%%%%
\begin{figure}
  \centering
  \includegraphics[clip, rviewport=0 0.13 1 0.94,width=0.4\textwidth]{dedx/cor_158_v1_c0_p1}
  \includegraphics[clip, rviewport=0 0.13 1 0.94,width=0.4\textwidth]{dedx/cor_158_v1_c1_p1}

  \includegraphics[clip, rviewport=0 0.13 1 0.94,width=0.4\textwidth]{dedx/cor_158_v1_c0_p2}
  \includegraphics[clip, rviewport=0 0.13 1 0.94,width=0.4\textwidth]{dedx/cor_158_v1_c1_p2}

  \includegraphics[clip, rviewport=0 0 1 0.94,width=0.4\textwidth]{dedx/cor_158_v1_c0_p3}
  \includegraphics[clip, rviewport=0 0 1 0.94,width=0.4\textwidth]{dedx/cor_158_v1_c1_p3}

  \caption{Correction factor for WST and 158 \GeVc dataset.}
  \label{fig:hadron:dedx:fit:fake:cor158w}
\end{figure}


%%%%%%%%%% COR %%%%%%%%%%%%%%
\begin{figure}
  \centering
  \includegraphics[clip, rviewport=0 0.13 1 0.94,width=0.4\textwidth]{dedx/cor_350_v0_c0_p1}
  \includegraphics[clip, rviewport=0 0.13 1 0.94,width=0.4\textwidth]{dedx/cor_350_v0_c1_p1}

  \includegraphics[clip, rviewport=0 0.13 1 0.94,width=0.4\textwidth]{dedx/cor_350_v0_c0_p2}
  \includegraphics[clip, rviewport=0 0.13 1 0.94,width=0.4\textwidth]{dedx/cor_350_v0_c1_p2}

  \includegraphics[clip, rviewport=0 0 1 0.94,width=0.4\textwidth]{dedx/cor_350_v0_c0_p3}
  \includegraphics[clip, rviewport=0 0 1 0.94,width=0.4\textwidth]{dedx/cor_350_v0_c1_p3}

  \caption{Correction factor for RST and 350 \GeVc dataset.}
  \label{fig:hadron:dedx:fit:fake:cor350r}
\end{figure}

%%%%%%%%%% COR %%%%%%%%%%%%%%
\begin{figure}
  \centering
  \includegraphics[clip, rviewport=0 0.13 1 0.94,width=0.4\textwidth]{dedx/cor_350_v1_c0_p1}
  \includegraphics[clip, rviewport=0 0.13 1 0.94,width=0.4\textwidth]{dedx/cor_350_v1_c1_p1}

  \includegraphics[clip, rviewport=0 0.13 1 0.94,width=0.4\textwidth]{dedx/cor_350_v1_c0_p2}
  \includegraphics[clip, rviewport=0 0.13 1 0.94,width=0.4\textwidth]{dedx/cor_350_v1_c1_p2}

  \includegraphics[clip, rviewport=0 0 1 0.94,width=0.4\textwidth]{dedx/cor_350_v1_c0_p3}
  \includegraphics[clip, rviewport=0 0 1 0.94,width=0.4\textwidth]{dedx/cor_350_v1_c1_p3}

  \caption{Correction factor for WST and 350 \GeVc dataset.}
  \label{fig:hadron:dedx:fit:fake:cor350w}
\end{figure}


\clearpage

%%%%%%%%%% FRACTION %%%%%%%%%%%%%%

\begin{figure}
  \centering
  \includegraphics[clip, rviewport=0 0 1 1,width=1.00\textwidth]{dedx/fraction_pt_158_fl2_v1}
  \caption{Particle fractions obtained from the \dedx fit of the WST and 158 \GeVc dataset, with target inserted.}
  \label{fig:hadron:dedx:fit:final158w}
\end{figure}

\begin{figure}
  \centering
  \includegraphics[clip, rviewport=0 0 1 1,width=1.00\textwidth]{dedx/fraction_pt_350_fl2_v0}
  \caption{Particle fractions obtained from the \dedx fit of the RST and 350 \GeVc dataset, with target inserted.}
  \label{fig:hadron:dedx:fit:final350r}
\end{figure}

\begin{figure}
  \centering
  \includegraphics[clip, rviewport=0 0 1 1,width=1.00\textwidth]{dedx/fraction_pt_350_fl2_v1}
  \caption{Particle fractions obtained from the \dedx fit of the WST and 350 \GeVc dataset, with target inserted.}
  \label{fig:hadron:dedx:fit:final350w}
\end{figure}

%%%%%%%%%% FRACTION OUT%%%%%%%%%%%%%%
\begin{figure}
  \centering
  \includegraphics[clip, rviewport=0 0 1 1,width=1.00\textwidth]{dedx/fraction_out_pt_158_v0}
  \caption{Particle fractions obtained from the \dedx fit of the RST and 158 \GeVc dataset, with target removed.}
  \label{fig:hadron:dedx:fit:out158r}
\end{figure}

\begin{figure}
  \centering
  \includegraphics[clip, rviewport=0 0 1 1,width=1.00\textwidth]{dedx/fraction_out_pt_158_v1}
  \caption{Particle fractions obtained from the \dedx fit of the WST and 158 \GeVc dataset, with target inserted.}
  \label{fig:hadron:dedx:fit:out158w}
\end{figure}

\begin{figure}
  \centering
  \includegraphics[clip, rviewport=0 0 1 1,width=1.00\textwidth]{dedx/fraction_out_pt_350_v0}
  \caption{Particle fractions obtained from the \dedx fit of the RST and 350 \GeVc dataset, with target inserted.}
  \label{fig:hadron:dedx:fit:out350r}
\end{figure}

\begin{figure}
  \centering
  \includegraphics[clip, rviewport=0 0 1 1,width=1.00\textwidth]{dedx/fraction_out_pt_350_v1}
  \caption{Particle fractions obtained from the \dedx fit of the WST and 350 \GeVc dataset, with target inserted.}
  \label{fig:hadron:dedx:fit:out350w}
\end{figure}

\clearpage


%%%%%%%%%% BETA %%%%%%%%%%%%%%
\begin{figure}
  \centering
  \includegraphics[clip, rviewport=0 0.13 1 0.94,width=0.4\textwidth]{dedx/fac_350_All_beta_c0_p1}
  \includegraphics[clip, rviewport=0 0.13 1 0.94,width=0.4\textwidth]{dedx/fac_350_All_beta_c1_p1}

  \includegraphics[clip, rviewport=0 0.13 1 0.94,width=0.4\textwidth]{dedx/fac_350_All_beta_c0_p2}
  \includegraphics[clip, rviewport=0 0.13 1 0.94,width=0.4\textwidth]{dedx/fac_350_All_beta_c1_p2}

  \includegraphics[clip, rviewport=0 0.13 1 0.94,width=0.4\textwidth]{dedx/fac_350_All_beta_c0_p3}
  \includegraphics[clip, rviewport=0 0.13 1 0.94,width=0.4\textwidth]{dedx/fac_350_All_beta_c1_p3}

  \caption{$\beta$ correction factor for the 350 \GeVc dataset.}
  \label{fig:hadron:correction:beta:dedx350}
\end{figure}


%%%%%%%%%%% BETA %%%%%%%%%%%%%%%%%%
\begin{figure}
  \centering
  \includegraphics[clip, rviewport=0 0 1 1,width=0.95\textwidth]{vzero/beta350}
  
  \caption{$\beta$ correction factor for the 350 \GeVc dataset.}
  \label{fig:hadron:correction:beta:vzero350}
\end{figure}



\chapter[Summary]{Summary}
\label{sec:conclusions}

In this thesis the muon deficit problem was approached in three different fronts.
In~\cref{sec:interpretation,sec:observable} we presented simulation studies aiming
to the development of methods to be applied in the analysis of measurements of
future experiments with the presence of muon detectors, which includes AugerPrime.
In~\cref{sec:hadron} we presented the measurements of hadron production spectra
in pion-carbon interactions by \NASixtyOne experiment. The importance of these type
of interaction for muon production in air showers were pointed
out in~\cref{sec:showers:phen:had}. \newline

The main idea of the method presented in~\cref{sec:interpretation}
is to explore the energy evolution of the moments of the \nmu
to interpret the measurements in terms of composition in despite
of the \nmu misprediction by the simulations. The first step of the method
is the development of a model to describe the energy and the primary mass evolution
of the moments of \lgnmu (\lgnmumean and \lgnmurms)
which capture the common behavior to the simulations with different
hadronic interaction models (Sec. 2 of \cref{sec:interpretation}).
This model also contains free parameters that represent
the features which are discrepant with relation to the models ($\alpha$ and $\beta$).
Second step is the selection of a set of composition scenarios that can be
motivated by astrophysical models or by the measurements of other observables, like \xmax.
These scenarios should predict the energy evolution of the abundance of different
groups of primaries (Sec. 4 of \cref{sec:interpretation}).
The next and final step is to compare the measurements of \lgnmumean and \lgnmurms
with the prediction of the model considering different composition scenarios. The comparison
is performed by a \cchi parameter for \lgnmumean and \lgnmurms separately and the free parameters of the model
are used then to minimize the \cchi. The minimum values of the \cchi (\chiminalpha and \chiminbeta)
can be compared for different composition scenarios and the true scenario can be identified by the smallest
values (Sec. 5 of \cref{sec:interpretation}).

To illustrate and test the method, a large set of simulated showers were used.
Two simulation codes were applied, \Conex and \Corsika. An algorithm to convert
the \nmu from \Conex, which counts all the muons at ground above 1 \GeV, to the
one that takes into account a limited lateral distance range (between 500 and 2000 m)
and a different muon energy threshold of 0.2 \GeV,
was developed by using \Corsika showers to parametrize their relation
(Sec. 3 of \cref{sec:interpretation}).

A consistency test of the method was performed by using these large simulation set.
As a result, it was shown that the method can provide a very efficient identification
of the true composition scenario. The muon deficit problem was considered by testing
the method using simulations in which the \nmu was artificially scaled by a given
factor. Furthermore, the uncertainties due to the energy scale were also considered
by scaling the shower energies in the simulations. The result of the method was shown
to be stable under these both tests (Sec. 5 of \cref{sec:interpretation}). \newline

In the analysis presented in~\cref{sec:observable}, a set of \Corsika simulations
were first used to characterize the muon energy spectrum at ground and to study
its dependencies on the energy and primary mass of the primary and on the hadronic
interaction model. It was observed that the most relevant feature of
the muon energy spectrum at ground to discriminate between hadronic models
is given by its mode parameter ($\eta$), which evolves strongly with the \dx of the shower.
Next step was the proposal of a new observable (\rmu) which correlates with $\eta$ and which
can be measured by an experimental setup including two types of muon detectors,
one at the surface and one buried few meters below the ground. These two types of detectors
were based on th designed of the AugerPrime detectors.

By using \Corsika simulations it was possible to test the applicability of \rmu measurements
under realistic experimental conditions. For that, it was considered an array of the area
of Auger infill and the experimental resolution on the measurements of the muon signal
were considered based on simulation studies from the literature.
As a final result, it was shown that the evaluation of the measured \rmumean as a function of \dx
can be efficiently used to constrain hadronic interaction models. \newline


The last part of this thesis, presented in~\cref{sec:hadron},
contains the measurements of hadron production spectra in pion-carbon
interactions by by \NASixtyOne experiment. The main steps of the analysis
are the event selections, particle identification, Monte Carlo correction and
the computation of the spectra. The particle identification step was done separately
for charged hadrons, by means of the fit of the \dedx distributions, and for
\vzero particles, by means of the signal extraction from the \minv spectra.

The final results include the production spectra of \pions, \kaons, \protons, \lambs and \kzeros
at two collision energies, 158 and 350 \GeV. The spectra of \protons are of particular interest
for muon production in air showers since one of the hypothesis to explain the muon deficit problem
is that the current hadronic interaction models under-produce (anti)baryon in hadron-air interaction.
From our results, it can be seen that a few current models actually predict well the \proton spectra,
which is an indication that the underproduction of (anti)baryons is not the reason of the
muon deficit problem. However, the strongest impact of our hadron production measurements
on the muon production in air shower by simulations will be seen by the performance of the next
generation of hadronic interaction models.
















































% ----------------------------------------------------------
% ELEMENTOS PÓS-TEXTUAIS
% ----------------------------------------------------------
\postextual
% ----------------------------------------------------------

% -----------------------------------------------------------
% Referências bibliográficas
% ----------------------------------------------------------
\bibliography{lib}


% ----------------------------------------------------------
% Glossário
% ----------------------------------------------------------
%
% Consulte o manual da classe abntex2 para orientações sobre o glossário.
%
%\glossary

% ----------------------------------------------------------
% Apêndices
% ----------------------------------------------------------
%\include{USPSC-Apendices}

% ----------------------------------------------------------
% Anexos
% ----------------------------------------------------------
%\include{USPSC-Anexos}

%---------------------------------------------------------------------
% INDICE REMISSIVO
%--------------------------------------------------------------------
%\include{USPSC-IndicesRemissivos}

%---------------------------------------------------------------------

\end{document}
