\chapter[Summary]{Summary}
\label{sec:conclusions}

In this thesis the muon deficit problem was approached by three different fronts.
In~\cref{sec:interpretation,sec:observable} we presented simulation studies aiming
the development of methods to be applied in the analysis of measurements of
future experiments with the presence of muon detectors, which includes AugerPrime.
In~\cref{sec:hadron} we presented the measurements of hadron production spectra
in pion-carbon interactions by \NASixtyOne experiment. The importance of these type
of interaction for muon production in air showers were pointed
out in~\cref{sec:showers:phen:had}. \newline

The main idea of the method presented in~\cref{sec:interpretation}
is to explore the energy evolution of the moments of the \nmu
to interpret the measurements in terms of composition in despite
of the \nmu misprediction by the simulations. The first step of the method
is the development of a model to describe the energy and the primary mass evolution
of the moments of \lgnmu (\lgnmumean and \lgnmurms)
which capture the common behavior to the simulations with different
hadronic interaction models (Sec. 2 of \cref{sec:interpretation}).
This model also contains free parameters that represent
the features which are discrepant with relation to the models ($\alpha$ and $\beta$).
Second step is the selection of a set of composition scenarios that can be
motivated by astrophysical models or by the measurements of other observables, like \xmax.
These scenarios should predict the energy evolution of the abundance of different
groups of primaries (Sec. 4 of \cref{sec:interpretation}).
The next and final step is to compare the measurements of \lgnmumean and \lgnmurms
with the prediction of the model considering different composition scenarios. The comparison
is performed by a \cchi parameter for \lgnmumean and \lgnmurms separately an the free parameters of the model
are used then to minimize the \cchi. The minimum values of the \cchi (\chiminalpha and \chiminbeta)
can be compared for different composition scenarios and the true scenario can be identified by the smallest
values (Sec. 5 of \cref{sec:interpretation}).

To illustrate and test the method, a large set of simulated showers were used.
Two simulation codes were applied, \Conex and \Corsika. An algorithm to convert
the \nmu from \Conex, which counts all the muons at ground above 1 \GeV, to the
one that takes into account a limited lateral distance range (between 500 and 2000 m)
and a different muon energy threshold of 0.2 \GeV,
was developed by using \Corsika showers to parametrize their relation
(Sec. 3 of \cref{sec:interpretation}).

A consistency test of the method was performed by using these large simulation set.
As a result, it was shown that the method can provide a very efficient identification
of the true composition scenario. The muon deficit problem was considered by testing
the method using simulations in which the \nmu was artificially scaled by a given
factor. Furthermore, the uncertainties due to the energy scale were also considered
by scaling the shower energies in the simulations. The result of the method was shown
to be stable under these both tests (Sec. 5 of \cref{sec:interpretation}). \newline

In the analysis presented in~\cref{sec:observable}, a set of \Corsika simulations
were first used to characterize the muon energy spectrum at ground and to study
its dependencies on the energy and primary mass of the primary and on the hadronic
interaction model. It was observed that the most relevant feature of
the muon energy spectrum at ground to discriminate between hadronic models
is given by its mode parameter ($\eta$), which evolves strongly with the \dx of the shower.
Next step was the proposal of a new observable (\rmu) which correlates with $\eta$ and which
can be measured by an experimental setup including two types of muon detectors,
one at the surface and one buried few meters below the ground. These two types of detectors
were based on th designed of the AugerPrime detectors.

By using \Corsika simulations it was possible to test the applicability of \rmu measurements
under realistic experimental conditions. For that, it was considered an array of the area
of Auger infill and the experimental resolution on the measurements of the muon signal
were considered based on simulation studies from the literature.
As a final result, it was shown that the evaluation of the measured \rmumean as a function of \dx
can be efficiently used to constrain hadronic interaction models. \newline


The last part of this thesis, presented in~\cref{sec:hadron},
contains the measurements of hadron production spectra in pion-carbon
interactions by by \NASixtyOne experiment. The main steps of the analysis
are the event selections, particle identification, Monte Carlo correction and
the computation of the spectra. The particle identification step was done separately
for charged hadrons, by means of the fit of the \dedx distributions, and for
\vzero particles, by means of the signal extraction from the \minv spectra.

The final results include the production spectra of \pions, \kaons, \protons, \lambs and \kzeros
at two collision energies, 158 and 350 \GeV. The spectra of \protons are of particular interest
for muon production in air showers since one of the hypothesis to explain the muon deficit problem
is that the current hadronic interaction models under-produce (anti)baryon in hadron-air interaction.
From our results, it can be seen that a few current models actually predict well the \proton spectra,
which is an indication that the underproduction of (anti)baryons is not the reason of the
muon deficit problem. However, the strongest impact of our hadron production measurements
on the muon production in air shower by simulations will be seen by the performance of the next
generation of hadronic interaction models.













































