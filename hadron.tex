\chapter[Hadron production in pion-carbon interactions]{Hadron production in pion-carbon interactions}
\label{sec:hadron}

\note{Introduction}

%%%%%%%%%%%%%%%%%%%%%%%%%%%%%%%%%%%%%%%%
\section{Dataset and simulations}
\label{sec:hadron:data}


\note{DONE}

The $\pi^-$-C data were collected by \NASixtyOne in 2009 at two beam energies:
158 and 350 \GeVc. The $\pi^-$ beam was a secondary one
produced by the collisions of a 400 \GeVc proton beam against
a 10 cm long beryllium target. The carbon target consisted of
an isotropic graphite plate with 2 cm thickness along the beam axis.
For more details about the $\pi^-$-C dataset see Ref.\cite{\RhoPaper}.

Two trigger modes are relevant for the present analysis: the beam and
interaction trigger, which are denominated by T1 and T2 respectively.
The definition of T1 is
$\text{S1}\wedge\text{S2}\wedge\overline{\text{V0}}\wedge\overline{\text{V1}}\wedge\overline{\text{V1}'}$
and T2 is
$\text{T1}\wedge\overline{\text{S4}}$.
While the T1
assures that a beam particle reached the target position,
the T2 is supose to eliminate events in which a beam particle
crossed the target without interacting. Because of the position of
the S4 detector, it can also be reached 
by high energy particles produced by the inelastic interaction at the target,
causing the removal of events which are desirable for the analysis.
It was verified that the rate of this events is very small and they do not
produce a significant bias on our results.
For more details about the trigger modes see Ref.~\cite{MartinThesis}.
The standard calibration algorithm applied to \NASixtyOne
data is described in Ref.~\cite{Abgrall:2008zz}.

To estimate and remove from the particle spectra the contribution
of interactions that do not occur at the target, a set of data
were also taken with the target removed. The amount of target
removed data is approximately 10\% of the total data taken.
In~\cref{sec:hadron:spec} we describe 
the procedure for the target removed subtraction.

The Monte Carlo simulation sets were generated by first generating
the primary interactions using hadronic interaction models and
then by passing the produced particles through a detailed
detector simulation based on \GeantThree package~\cite{Brun:1994aa}.
Three hadronic interaction models were used: \EposLong~\cite{\EposPaper},
\DPMJetLong~\cite{\DPMJetPaper} and \QGSJetLong~\cite{\QGSJetPaper}.
For each beam energy and hadronic interaction model,
a simulation set was produced with approximately the same event
number as the datasets. 
Both the data and the simulations were reconstructed by the standard
\NASixtyOne reconstruction chain~\cite{Abgrall:2011ae}. 

%%%%%%%%%%%%%%%%%%%%%%%%%%%%%%%%%%%%%%%%
\section{Upstream and event selections}
\label{sec:hadron:selections}



The upstream and event selection applied in this analysis
are described in Ref.~\cite{\RhoPaper}. It is important
to mention the cut on the vertex position of
the main vertex. For this analysis, the events
in which the fitted Z position of the main vertex
is more than 17 cm distant from the nominal target
position (-580 cm) were removed. The motivation for that
is to reduce the number of out of target interactions.

In~\cref{tab:hadron:stat} we show the number of events available
for each data and simulation set after the event selection.

\begin{table}
  \begin{center}
    \begin{tabular}{|r|c|c|} \hline
      & 158 GeV/c & 350 GeV/c \\ \hline
      Data        & 3.46 M     & 3.04 M \\
      \EposLong   & 3.71 M     & 3.12 M \\
      \DPMJetLong & 3.92 M     & 3.47 M \\
      \QGSJetLong & 3.71 M     & 3.06 M \\ \hline
    \end{tabular}
    \caption{}
    \label{tab:hadron:stat}
  \end{center}
\end{table}

In addition to the standard dataset, a set of data were
recorded with the carbon target removed. This dataset
is used to estimate the contribution of the interactions
which do not occur in the target. The two dataset
will be refereed as target inserted and target removed along
the text.


%%%%%%%%%%%%%%%%%%%%%%%%%%%%%%%%%%%%%%%%
\section{Track selection}
\label{sec:hadron:trackselection}

\warning{rewrite}

The following section criteria were applied to assure
the selection of well reconstructed tracks:
\begin{enumerate}[label=(\roman*)]
\item The fitted track has to be inside the geometrical acceptance of the detector.
\item The total number of cluster on the track should be greater than or equal to 25.
\item The sum of clusters on both VTPCs has to be greater than or equal to 12, or
  the number of cluster on the GTPC has to be greater than of equal to 6.
\item The distance between the track extrapolated to the interaction plane and the
  interaction point (impact parameter) should be smaller than 4 cm in the both horizontal
  and vertical plane.
\end{enumerate}


%%%%%%%%%%%%%%%%%%%%%%%%%%%%%%%%%%%%%%%%
\section{\vzero selection}
\label{sec:hadron:vzeroselection}

The following criteria were applied to select the \vzeros:
\begin{enumerate}
\item The found vertex has to be identified as a \vzero type vertex
\item The number of daughter tracks has to be equal to 2
\item Both daughter tracks have to be opposite charges
\item The total number of cluster has to be greater than 30 for both tracks
\item At least one track has to have more than 15 clusters in the VTPCs
\end{enumerate}


%%%%%%%%%%%%%%%%%%%%%%%%%%%%%%%%%%%%%%%%
\section{Phase space binning}

\warning{rewrite}

For the present analysis, the phase space were divided
in bins of \p and \pT. In~\cref{}
we show the (\p,\pT) configuration used in this analysis.

The phase space was split in bins of \p and \pT.
In~\cref{} we show the binning configuration, which
is the same for \lamb and \antilamb but is different
for \kzeros.


%%%%%%%%%%%%%%%%%%%%%%%%%%%%%%%%%%%%%%%%
\section{\dedx analysis}

\warning{rewrite}

In this section we present the analysis and the results
of the spectra of \pions, \kaons and \protons.
The analysis is done in track basis and it starts
with the track selection that is shown in~\cref{sec:app:proj:dedx:selection}.
Second step, after splitting the measured tracks in bins of the phase
space, is to perform the particle identification analysis
aiming to determine the number of tracks which correspond
to each particle type. This is done here by using the
energy deposited by the tracks in the TPCs, the \dedx,
which is briefly explored in~\cref{sec:app:proj:dedx:meas}.


The particle identification (\cref{sec:app:proj:dedx:pid})
is done statistically by fitting
the \dedx distributions with a combination of particle
types. Because of the complicated dependence of the \dedx
on the particle momentum and properties of the measured track
(like number of clusters), the \dedx fit turns to be
a complex step, requiring the development of a suitable
\dedx model and an efficient \dedx fit strategy.
In addition to that, the overlapping of the \dedx distributions
of different particles in certain regions of momentum
makes the \dedx fit even more complicated for these
regions of the phase space.
Because of these difficulties, the particle identification
step turns to be the most challenging one and several
original ideas were proposed and applied to overcome them. 


The following steps are the Monte Carlo correction (\cref{sec:app:proj:dedx:correction})
and the computation of the spectra (\cref{sec:app:proj:dedx:spectra}).
Finally the derived spectra are shown in~\cref{sec:app:proj:dedx:results}. 

In general, the \dedx measurements can provide a very precise
particle identification in a statistical basis. For a given
momentum interval, the \dedx distributions originated by
different charged particles can be well separated by their
mean value and shape. Applying a proper model to describe
the \dedx distribution, it is possible to perform a fitting
of the measured distributions and determine the fractions of
tracks associated to each particle. This procedure can only fail
for a restricted region of momentum in which the average of
the energy deposit function of two particles are too close to
each other. In the next section, the particle
identification procedure will be described. 


%%%---------------------------------%%%%
\subsection{\dedx measurements}

\warning{rewrite}

The \dedx associated to each track is defined as the energy lost by the charged
particle per unit of length.
In \NASixtyOne the \dedx is measured by the TPCs, which collect the number of
freed electrons from the ionization of the gas by the passage of the charged particles.
The determination of the \dedx from the signal recorded at the TPCs requires a complex and
detailed procedure, which has been very well established by the \NAFortyNine and \NASixtyOne
experiment along the last decades. Since the detailed description of this procedure
is out of the scope of this text, only the general idea and the most important aspects
will be presented in the next paragraphs. More complete and detailed approaches
can be found in Refs.\cite{GaborVeresThesis}.

Several processes can contribute to the energy loss of charged particle due to
its interaction with atoms of the gas in the TPCs, being the emission of
electrons by ionization the most relevant one. The electrons emitted are
drifted through the chamber and collected in the readout pads, which records
the signal as ADC charges. A set of consecutive charges defines a cluster.
The 3-dimensional position of the cluster is determined by the position
and time distributions in which the charges reaches the readout pad. This
position gives the crossing point of the particle track inside the TPCs.

The total charge measured in each cluster is related to the \dedx of each track.
However, numerous detector effects have to be corrected at the cluster level before
grouping the cluster in one unique \dedx value. The simplest correction accounts for
the geometrical differences due to the incident angle of the track in the pad and
the pad widths. More complicated corrections account for differences in the electronic
gain and gas pressure/temperature of the pads, differences in the sector gains and
losses of electrons during the drift in the chambers and in the readout pad.
A detailed description of the correction procedure can be found in Ref.~\cite{AntoniMThesis}.

The track \dedx is then determined by the combination of the corrected 
charges in all clusters. The well known Landau-like shape of the
energy loss probability distribution makes the simplest approach,
based on the average over all clusters, not suitable. Because of the
long tail of the probability distribution, the average and the variance
of the measured charges are not well behaved for typical number of clusters
($\sim$ 20-150). To overcome this issue and obtain a satisfactory \dedx resolution,
the method of the truncated mean is applied, in which only a subset of the clusters
is selected to compute the average. The selected clusters are defined by ordering
the values of the charge and the selecting the ones inside a given percentage interval.
For the \NASixtyOne experiment, it was found the optimal interval being the smallest 50\%
of the clusters~\cite{GaborVeresThesis}.


%%%---------------------------------%%%%
\subsection{\dedx model}

\warning{rewrite}

To perform the particle identification by fitting the
measured \dedx distribution, a model that describe
the \dedx distributions of different particle types as a function
of their momentum \p is required. Once there is no universal choice
of this model, several different alternatives have been
used in previous analysis. Although the model chosen here 
is based on previous studies developed for
\NAFortyNine and \NASixtyOne experiment, it contain
particular features which were found to be the most suitable
for the present analysis.

First, the notation adopted in this text has to be presented
for clarification. The particle types are represented by
the index \ipart, and it can assume one of the five particle types
treated here, $\ipart=e, \pi, K, p, d$. The charges are represented by
the index \ich, being that $\ich = +$ or $\ich=-$. Also, the number of
cluster measured in a track is represented by \ncl and the \dedx
is replaced by \eps for simplicity.

Because the \dedx is obtained by averaging the measured charge
over a certain number of cluster, it is natural to assume that
the shape of the \eps distribution depends on the \ncl.
To be more precise, the \eps resolution should be 
larger for smaller \ncl and vice-versa. 
Additionally, it is obviously expected the mean of the distribution
to change with the momentum of the particle \p and the particle type.

Since the shape of the \eps distribution, for a given \ncl and \p,
can be well described by an asymmetric Gaussian function, the
probability density function of \eps for a particle type
\ipart and charge \ich is written as
\begin{equation}
  f_{\ipart,\ich}(\eps|\p,\ncl) = \frac{1}{\sqrt{2\pi}\sigma_{\ipart,\ich}} \;
  \exp\left[-\frac{1}{2}\left(\frac{\eps-\mu_{\ipart,\ich}}{\delta \; \sigma_{\ipart,\ich}}\right)^2\right],
  \label{eq:dedx:model:pdf}
\end{equation}
with
\begin{equation}
  \delta =
  \begin{cases}
    & 1-d, \ \ \ \eps \le \mu_{\ipart,\ich} \\
    & 1+d, \ \ \ \eps > \mu_{\ipart,\ich}, \\
  \end{cases}
  \label{eq:dedx:model:asymm}
\end{equation}
where the parameter $\mu$ is the mode of the distribution, $\sigma$ is the resolution
and $d$ is the asymmetry parameter. The \p and \ncl dependence is implicit
on the parameters $\mu$ and $\sigma$, as will be explained next.
The mode $\mu$ is related to the mean of the distribution, \meaneps, by
\begin{equation}
  \mu_{\ipart,\ich} = \meaneps_{\ipart,\ich} - \frac{\sigma_{\ipart,\ich}}{\sqrt{2\pi}}
  \left[\left(1+d\right)^2 - \left(1-d\right)^2 \right].
  \label{eq:dedx:model:mu}
\end{equation}

The \p evolution of \meaneps is expected to follow
a Bethe-Bloch-like function. In this model, a reference
\meaneps(\p) curve is defined by a data-based
parametrization using a generic function which is a
variation of the Bethe-Bloch function. The reference value
of \meaneps for a given \p is denoted as \meanepsbb.
To account for deviations from the reference \meaneps,
the present model includes a set of parameters called
\textit{calibration constants}, which are denoted by $X$.
These parameters act as logarithmic shifts of the \meaneps
around \meanepsbb and they can in principle be applied
to each particle and charge separately. To reduce the complexity
of the model, it is assumed here one global calibration constant
for each charge that follows the \meaneps of the $\pi$ distribution
and individual calibration constants for the other particles,
but being common for both charges. In the end, the \meaneps for a
given particle type \ipart and charge \ich is given by
\begin{equation}
  \meaneps_{\ipart,\ich} =
  \begin{cases}
    & \meaneps_{\ipart}^\text{BB} \; e^{X_{\ipart}^{\ich}} \ \ \ \ \ \ \ (i=\pi) \\
    & \meaneps_{\ipart}^\text{BB} \; e^{X_{\pi}^{\ich}} \; e^{X_{\ipart}^{\ich}} \ \ \ (i\neq\pi).
  \end{cases}
  \label{eq:dedx:model:cal}
\end{equation}
In total, 6 calibration constants are included in the model:
$X_{\pi}^{+}$, $X_{\pi}^{-}$, $X_{e}$, $X_{K}$, $X_{p}$ and $X_{d}$.

The dependence of the resolution $\sigma$ on \ncl is assumed to be of the form
$\sigma \sim 1/\sqrt{\ncl}$. Besides that, $\sigma$ is assumed to depend
on the \meaneps by a power law relation and a normalization parameter for
each charge is also included ($\sigma_0^{\ich}$). The final expression for the resolution is,
\begin{equation}
  \sigma_{\ipart,\ich} = \frac{\sigma_0^{\ich}}{\sqrt{\ncl}} \meaneps_{\ipart,\ich}^{\alpha},
  \label{eq:dedx:model:sigma}
\end{equation}
in which 3 more parameters are included: $\sigma_0^+$, $\sigma_0^-$ and $\alpha$. 

By combining
the~\cref{eq:dedx:model:asymm,eq:dedx:model:mu,eq:dedx:model:cal,eq:dedx:model:sigma}
with the~\cref{eq:dedx:model:pdf}, we obtain the probability density
function of \eps for each particle \ipart and charge \ich. Besides the 6 calibration constants,
the model includes 4 \textit{shape parameters}: $\sigma_0^+$, $\sigma_0^-$, $\alpha$ and $d$.
Altogether there are 10 parameters that can be set free to fit the model
to the measured \eps distributions.


%%%---------------------------------%%%%
\subsection{\dedx fit strategy}



%%%%%%%%%%%%%%%%%%%%%%%%%%%%%%%%%%%%%%%%
\section{\vzero analysis}


%%%---------------------------------%%%%
\subsection{Signal extraction}




%%%%%%%%%%%%%%%%%%%%%%%%%%%%%%%%%%%%%%%%
\section{Corrections}


%%%%%%%%%%%%%%%%%%%%%%%%%%%%%%%%%%%%%%%%
\section{Spectra}


%%%---------------------------------%%%%
\subsection{Statistical uncertainties}


%%%---------------------------------%%%%
\subsection{Systematic uncertainties}




%%%%%%%%%%%%%%%%%%%%%%%%%%%%%%%%%%%%%%%%
\section{Results}

%%%%%%%%%%%%%%%%%%%%%%%%%%%%%%%%%%%%%%%%
\section{Summary and conclusions}

